%#!pdflatex
\documentclass{ddltxtyp}
\begin{document}

\begin{package}{cmr10}{amsfonts}{{\TeX} fonts from the American Mathematical Society.}
An extended set of fonts for use in mathematics, including:
extra mathematical symbols; blackboard bold letters (uppercase
only); fraktur letters; subscript sizes of bold math italic and
bold Greek letters; subscript sizes of large symbols such as
sum and product; added sizes of the Computer Modern small caps
font; cyrillic fonts (from the University of Washington); Euler
mathematical fonts. All fonts are provided as Adobe Type~1
files, and all except the Euler fonts are provided as Metafont
source. The distribution also includes the canonical Type~1
versions of the Computer Modern family of fonts. Plain {\TeX} and
{\LaTeX} macros for using the fonts are provided.
\end{package}

\begin{package}{cmb10}{amsfonts}{{\TeX} fonts from the American Mathematical Society.}
An extended set of fonts for use in mathematics, including:
extra mathematical symbols; blackboard bold letters (uppercase
only); fraktur letters; subscript sizes of bold math italic and
bold Greek letters; subscript sizes of large symbols such as
sum and product; added sizes of the Computer Modern small caps
font; cyrillic fonts (from the University of Washington); Euler
mathematical fonts. All fonts are provided as Adobe Type~1
files, and all except the Euler fonts are provided as Metafont
source. The distribution also includes the canonical Type~1
versions of the Computer Modern family of fonts. Plain {\TeX} and
{\LaTeX} macros for using the fonts are provided.
\end{package}
\begin{package}{cmbx10}{amsfonts}{{\TeX} fonts from the American Mathematical Society.}
An extended set of fonts for use in mathematics, including:
extra mathematical symbols; blackboard bold letters (uppercase
only); fraktur letters; subscript sizes of bold math italic and
bold Greek letters; subscript sizes of large symbols such as
sum and product; added sizes of the Computer Modern small caps
font; cyrillic fonts (from the University of Washington); Euler
mathematical fonts. All fonts are provided as Adobe Type~1
files, and all except the Euler fonts are provided as Metafont
source. The distribution also includes the canonical Type~1
versions of the Computer Modern family of fonts. Plain {\TeX} and
{\LaTeX} macros for using the fonts are provided.
\end{package}
\begin{package}{cmbxsl10}{amsfonts}{{\TeX} fonts from the American Mathematical Society.}
An extended set of fonts for use in mathematics, including:
extra mathematical symbols; blackboard bold letters (uppercase
only); fraktur letters; subscript sizes of bold math italic and
bold Greek letters; subscript sizes of large symbols such as
sum and product; added sizes of the Computer Modern small caps
font; cyrillic fonts (from the University of Washington); Euler
mathematical fonts. All fonts are provided as Adobe Type~1
files, and all except the Euler fonts are provided as Metafont
source. The distribution also includes the canonical Type~1
versions of the Computer Modern family of fonts. Plain {\TeX} and
{\LaTeX} macros for using the fonts are provided.
\end{package}
\begin{package}{cmbxti10}{amsfonts}{{\TeX} fonts from the American Mathematical Society.}
An extended set of fonts for use in mathematics, including:
extra mathematical symbols; blackboard bold letters (uppercase
only); fraktur letters; subscript sizes of bold math italic and
bold Greek letters; subscript sizes of large symbols such as
sum and product; added sizes of the Computer Modern small caps
font; cyrillic fonts (from the University of Washington); Euler
mathematical fonts. All fonts are provided as Adobe Type~1
files, and all except the Euler fonts are provided as Metafont
source. The distribution also includes the canonical Type~1
versions of the Computer Modern family of fonts. Plain {\TeX} and
{\LaTeX} macros for using the fonts are provided.
\end{package}
\begin{package}{cmcsc10}{amsfonts}{{\TeX} fonts from the American Mathematical Society.}
An extended set of fonts for use in mathematics, including:
extra mathematical symbols; blackboard bold letters (uppercase
only); fraktur letters; subscript sizes of bold math italic and
bold Greek letters; subscript sizes of large symbols such as
sum and product; added sizes of the Computer Modern small caps
font; cyrillic fonts (from the University of Washington); Euler
mathematical fonts. All fonts are provided as Adobe Type~1
files, and all except the Euler fonts are provided as Metafont
source. The distribution also includes the canonical Type~1
versions of the Computer Modern family of fonts. Plain {\TeX} and
{\LaTeX} macros for using the fonts are provided.
\end{package}
\begin{package}{cmdunh10}{amsfonts}{{\TeX} fonts from the American Mathematical Society.}
An extended set of fonts for use in mathematics, including:
extra mathematical symbols; blackboard bold letters (uppercase
only); fraktur letters; subscript sizes of bold math italic and
bold Greek letters; subscript sizes of large symbols such as
sum and product; added sizes of the Computer Modern small caps
font; cyrillic fonts (from the University of Washington); Euler
mathematical fonts. All fonts are provided as Adobe Type~1
files, and all except the Euler fonts are provided as Metafont
source. The distribution also includes the canonical Type~1
versions of the Computer Modern family of fonts. Plain {\TeX} and
{\LaTeX} macros for using the fonts are provided.
\end{package}
\begin{package}{cmff10}{amsfonts}{{\TeX} fonts from the American Mathematical Society.}
An extended set of fonts for use in mathematics, including:
extra mathematical symbols; blackboard bold letters (uppercase
only); fraktur letters; subscript sizes of bold math italic and
bold Greek letters; subscript sizes of large symbols such as
sum and product; added sizes of the Computer Modern small caps
font; cyrillic fonts (from the University of Washington); Euler
mathematical fonts. All fonts are provided as Adobe Type~1
files, and all except the Euler fonts are provided as Metafont
source. The distribution also includes the canonical Type~1
versions of the Computer Modern family of fonts. Plain {\TeX} and
{\LaTeX} macros for using the fonts are provided.
\end{package}
\begin{package}{cmfi10}{amsfonts}{{\TeX} fonts from the American Mathematical Society.}
An extended set of fonts for use in mathematics, including:
extra mathematical symbols; blackboard bold letters (uppercase
only); fraktur letters; subscript sizes of bold math italic and
bold Greek letters; subscript sizes of large symbols such as
sum and product; added sizes of the Computer Modern small caps
font; cyrillic fonts (from the University of Washington); Euler
mathematical fonts. All fonts are provided as Adobe Type~1
files, and all except the Euler fonts are provided as Metafont
source. The distribution also includes the canonical Type~1
versions of the Computer Modern family of fonts. Plain {\TeX} and
{\LaTeX} macros for using the fonts are provided.
\end{package}
\begin{package}{cmfib8}{amsfonts}{{\TeX} fonts from the American Mathematical Society.}
An extended set of fonts for use in mathematics, including:
extra mathematical symbols; blackboard bold letters (uppercase
only); fraktur letters; subscript sizes of bold math italic and
bold Greek letters; subscript sizes of large symbols such as
sum and product; added sizes of the Computer Modern small caps
font; cyrillic fonts (from the University of Washington); Euler
mathematical fonts. All fonts are provided as Adobe Type~1
files, and all except the Euler fonts are provided as Metafont
source. The distribution also includes the canonical Type~1
versions of the Computer Modern family of fonts. Plain {\TeX} and
{\LaTeX} macros for using the fonts are provided.
\end{package}
\begin{package}{cmitt10}{amsfonts}{{\TeX} fonts from the American Mathematical Society.}
An extended set of fonts for use in mathematics, including:
extra mathematical symbols; blackboard bold letters (uppercase
only); fraktur letters; subscript sizes of bold math italic and
bold Greek letters; subscript sizes of large symbols such as
sum and product; added sizes of the Computer Modern small caps
font; cyrillic fonts (from the University of Washington); Euler
mathematical fonts. All fonts are provided as Adobe Type~1
files, and all except the Euler fonts are provided as Metafont
source. The distribution also includes the canonical Type~1
versions of the Computer Modern family of fonts. Plain {\TeX} and
{\LaTeX} macros for using the fonts are provided.
\end{package}
\begin{package}{cmsl10}{amsfonts}{{\TeX} fonts from the American Mathematical Society.}
An extended set of fonts for use in mathematics, including:
extra mathematical symbols; blackboard bold letters (uppercase
only); fraktur letters; subscript sizes of bold math italic and
bold Greek letters; subscript sizes of large symbols such as
sum and product; added sizes of the Computer Modern small caps
font; cyrillic fonts (from the University of Washington); Euler
mathematical fonts. All fonts are provided as Adobe Type~1
files, and all except the Euler fonts are provided as Metafont
source. The distribution also includes the canonical Type~1
versions of the Computer Modern family of fonts. Plain {\TeX} and
{\LaTeX} macros for using the fonts are provided.
\end{package}
\begin{package}{cmsltt10}{amsfonts}{{\TeX} fonts from the American Mathematical Society.}
An extended set of fonts for use in mathematics, including:
extra mathematical symbols; blackboard bold letters (uppercase
only); fraktur letters; subscript sizes of bold math italic and
bold Greek letters; subscript sizes of large symbols such as
sum and product; added sizes of the Computer Modern small caps
font; cyrillic fonts (from the University of Washington); Euler
mathematical fonts. All fonts are provided as Adobe Type~1
files, and all except the Euler fonts are provided as Metafont
source. The distribution also includes the canonical Type~1
versions of the Computer Modern family of fonts. Plain {\TeX} and
{\LaTeX} macros for using the fonts are provided.
\end{package}
\begin{package}{cmss10}{amsfonts}{{\TeX} fonts from the American Mathematical Society.}
An extended set of fonts for use in mathematics, including:
extra mathematical symbols; blackboard bold letters (uppercase
only); fraktur letters; subscript sizes of bold math italic and
bold Greek letters; subscript sizes of large symbols such as
sum and product; added sizes of the Computer Modern small caps
font; cyrillic fonts (from the University of Washington); Euler
mathematical fonts. All fonts are provided as Adobe Type~1
files, and all except the Euler fonts are provided as Metafont
source. The distribution also includes the canonical Type~1
versions of the Computer Modern family of fonts. Plain {\TeX} and
{\LaTeX} macros for using the fonts are provided.
\end{package}
\begin{package}{cmssbx10}{amsfonts}{{\TeX} fonts from the American Mathematical Society.}
An extended set of fonts for use in mathematics, including:
extra mathematical symbols; blackboard bold letters (uppercase
only); fraktur letters; subscript sizes of bold math italic and
bold Greek letters; subscript sizes of large symbols such as
sum and product; added sizes of the Computer Modern small caps
font; cyrillic fonts (from the University of Washington); Euler
mathematical fonts. All fonts are provided as Adobe Type~1
files, and all except the Euler fonts are provided as Metafont
source. The distribution also includes the canonical Type~1
versions of the Computer Modern family of fonts. Plain {\TeX} and
{\LaTeX} macros for using the fonts are provided.
\end{package}
\begin{package}{cmssdc10}{amsfonts}{{\TeX} fonts from the American Mathematical Society.}
An extended set of fonts for use in mathematics, including:
extra mathematical symbols; blackboard bold letters (uppercase
only); fraktur letters; subscript sizes of bold math italic and
bold Greek letters; subscript sizes of large symbols such as
sum and product; added sizes of the Computer Modern small caps
font; cyrillic fonts (from the University of Washington); Euler
mathematical fonts. All fonts are provided as Adobe Type~1
files, and all except the Euler fonts are provided as Metafont
source. The distribution also includes the canonical Type~1
versions of the Computer Modern family of fonts. Plain {\TeX} and
{\LaTeX} macros for using the fonts are provided.
\end{package}
\begin{package}{cmssi10}{amsfonts}{{\TeX} fonts from the American Mathematical Society.}
An extended set of fonts for use in mathematics, including:
extra mathematical symbols; blackboard bold letters (uppercase
only); fraktur letters; subscript sizes of bold math italic and
bold Greek letters; subscript sizes of large symbols such as
sum and product; added sizes of the Computer Modern small caps
font; cyrillic fonts (from the University of Washington); Euler
mathematical fonts. All fonts are provided as Adobe Type~1
files, and all except the Euler fonts are provided as Metafont
source. The distribution also includes the canonical Type~1
versions of the Computer Modern family of fonts. Plain {\TeX} and
{\LaTeX} macros for using the fonts are provided.
\end{package}
\begin{package}{cmssq8}{amsfonts}{{\TeX} fonts from the American Mathematical Society.}
An extended set of fonts for use in mathematics, including:
extra mathematical symbols; blackboard bold letters (uppercase
only); fraktur letters; subscript sizes of bold math italic and
bold Greek letters; subscript sizes of large symbols such as
sum and product; added sizes of the Computer Modern small caps
font; cyrillic fonts (from the University of Washington); Euler
mathematical fonts. All fonts are provided as Adobe Type~1
files, and all except the Euler fonts are provided as Metafont
source. The distribution also includes the canonical Type~1
versions of the Computer Modern family of fonts. Plain {\TeX} and
{\LaTeX} macros for using the fonts are provided.
\end{package}
\begin{package}{cmssqi8}{amsfonts}{{\TeX} fonts from the American Mathematical Society.}
An extended set of fonts for use in mathematics, including:
extra mathematical symbols; blackboard bold letters (uppercase
only); fraktur letters; subscript sizes of bold math italic and
bold Greek letters; subscript sizes of large symbols such as
sum and product; added sizes of the Computer Modern small caps
font; cyrillic fonts (from the University of Washington); Euler
mathematical fonts. All fonts are provided as Adobe Type~1
files, and all except the Euler fonts are provided as Metafont
source. The distribution also includes the canonical Type~1
versions of the Computer Modern family of fonts. Plain {\TeX} and
{\LaTeX} macros for using the fonts are provided.
\end{package}
\begin{package}{cmtcsc10}{amsfonts}{{\TeX} fonts from the American Mathematical Society.}
An extended set of fonts for use in mathematics, including:
extra mathematical symbols; blackboard bold letters (uppercase
only); fraktur letters; subscript sizes of bold math italic and
bold Greek letters; subscript sizes of large symbols such as
sum and product; added sizes of the Computer Modern small caps
font; cyrillic fonts (from the University of Washington); Euler
mathematical fonts. All fonts are provided as Adobe Type~1
files, and all except the Euler fonts are provided as Metafont
source. The distribution also includes the canonical Type~1
versions of the Computer Modern family of fonts. Plain {\TeX} and
{\LaTeX} macros for using the fonts are provided.
\end{package}
\begin{package}{cmtex10}{amsfonts}{{\TeX} fonts from the American Mathematical Society.}
An extended set of fonts for use in mathematics, including:
extra mathematical symbols; blackboard bold letters (uppercase
only); fraktur letters; subscript sizes of bold math italic and
bold Greek letters; subscript sizes of large symbols such as
sum and product; added sizes of the Computer Modern small caps
font; cyrillic fonts (from the University of Washington); Euler
mathematical fonts. All fonts are provided as Adobe Type~1
files, and all except the Euler fonts are provided as Metafont
source. The distribution also includes the canonical Type~1
versions of the Computer Modern family of fonts. Plain {\TeX} and
{\LaTeX} macros for using the fonts are provided.
\end{package}
\begin{package}{cmti10}{amsfonts}{{\TeX} fonts from the American Mathematical Society.}
An extended set of fonts for use in mathematics, including:
extra mathematical symbols; blackboard bold letters (uppercase
only); fraktur letters; subscript sizes of bold math italic and
bold Greek letters; subscript sizes of large symbols such as
sum and product; added sizes of the Computer Modern small caps
font; cyrillic fonts (from the University of Washington); Euler
mathematical fonts. All fonts are provided as Adobe Type~1
files, and all except the Euler fonts are provided as Metafont
source. The distribution also includes the canonical Type~1
versions of the Computer Modern family of fonts. Plain {\TeX} and
{\LaTeX} macros for using the fonts are provided.
\end{package}
\begin{package}{cmtt10}{amsfonts}{{\TeX} fonts from the American Mathematical Society.}
An extended set of fonts for use in mathematics, including:
extra mathematical symbols; blackboard bold letters (uppercase
only); fraktur letters; subscript sizes of bold math italic and
bold Greek letters; subscript sizes of large symbols such as
sum and product; added sizes of the Computer Modern small caps
font; cyrillic fonts (from the University of Washington); Euler
mathematical fonts. All fonts are provided as Adobe Type~1
files, and all except the Euler fonts are provided as Metafont
source. The distribution also includes the canonical Type~1
versions of the Computer Modern family of fonts. Plain {\TeX} and
{\LaTeX} macros for using the fonts are provided.
\end{package}

% \begin{package}{}{cm}{Computer Modern fonts.}
% Knuth's final iteration of his re-interpretation of a c.19
% Modern-style font from Monotype. The family is comprehensive,
% offering both sans and roman styles, and a monospaced font,
% together with mathematics fonts closely integrated with the
% mathematical facilities of {\TeX} itself. The base fonts are
% distributed as Metafont source, but autotraced PostScript Type
% 1 versions are available (one version in the AMS fonts
% distribution, and also the BaKoMa distribution). The Computer
% Modern fonts have inspired many later families, notably the
% European Computer Modern and the Latin Modern families.
% \end{package}

\begin{package}{manfnt}{manfnt}{LaTeX support for the TeX book symbols.}
A LaTeX package for easy access to the symbols of the Knuth's
'manual' font, such as the Dangerous Bend and Manual-errata
Arrow.
\end{package}

% \begin{package}{pagd8r}{avantgar: URWGothicL-Demi}{URW ``Base 35'' font pack for {\LaTeX}.}
% A set of fonts for use as ``drop-in'' replacements for Adobe's
% basic set, comprising: - Century Schoolbook (substituting for
% Adobe's New Century Schoolbook); - Dingbats (substituting for
% Adobe's Zapf Dingbats); - Nimbus Mono L (substituting for
% Abobe's Courier); - Nimbus Roman No9 L (substituting for
% Adobe's Times); - Nimbus Sans L (substituting for Adobe's
% Helvetica); - Standard Symbols L (substituting for Adobe's
% Symbol); - URW Bookman; - URW Chancery L Medium Italic
% (substituting for Adobe's Zapf Chancery); - URW Gothic L Book
% (substituting for Adobe's Avant Garde); and - URW Palladio L
% (substituting for Adobe's Palatino).
% \end{package}

% \begin{package}{pagdo8r}{avantgar: URWGothicL-DemiObli}{URW ``Base 35'' font pack for {\LaTeX}.}
% A set of fonts for use as ``drop-in'' replacements for Adobe's
% basic set, comprising: - Century Schoolbook (substituting for
% Adobe's New Century Schoolbook); - Dingbats (substituting for
% Adobe's Zapf Dingbats); - Nimbus Mono L (substituting for
% Abobe's Courier); - Nimbus Roman No9 L (substituting for
% Adobe's Times); - Nimbus Sans L (substituting for Adobe's
% Helvetica); - Standard Symbols L (substituting for Adobe's
% Symbol); - URW Bookman; - URW Chancery L Medium Italic
% (substituting for Adobe's Zapf Chancery); - URW Gothic L Book
% (substituting for Adobe's Avant Garde); and - URW Palladio L
% (substituting for Adobe's Palatino).
% \end{package}

\begin{package}{pagk8r}{avantgar: URWGothicL-Book}{URW ``Base 35'' font pack for {\LaTeX}.}
A set of fonts for use as ``drop-in'' replacements for Adobe's
basic set, comprising: - Century Schoolbook (substituting for
Adobe's New Century Schoolbook); - Dingbats (substituting for
Adobe's Zapf Dingbats); - Nimbus Mono L (substituting for
Abobe's Courier); - Nimbus Roman No9 L (substituting for
Adobe's Times); - Nimbus Sans L (substituting for Adobe's
Helvetica); - Standard Symbols L (substituting for Adobe's
Symbol); - URW Bookman; - URW Chancery L Medium Italic
(substituting for Adobe's Zapf Chancery); - URW Gothic L Book
(substituting for Adobe's Avant Garde); and - URW Palladio L
(substituting for Adobe's Palatino).
\end{package}

% \begin{package}{pagko8r}{avantgar: URWGothicL-BookObli}{URW ``Base 35'' font pack for {\LaTeX}.}
% A set of fonts for use as ``drop-in'' replacements for Adobe's
% basic set, comprising: - Century Schoolbook (substituting for
% Adobe's New Century Schoolbook); - Dingbats (substituting for
% Adobe's Zapf Dingbats); - Nimbus Mono L (substituting for
% Abobe's Courier); - Nimbus Roman No9 L (substituting for
% Adobe's Times); - Nimbus Sans L (substituting for Adobe's
% Helvetica); - Standard Symbols L (substituting for Adobe's
% Symbol); - URW Bookman; - URW Chancery L Medium Italic
% (substituting for Adobe's Zapf Chancery); - URW Gothic L Book
% (substituting for Adobe's Avant Garde); and - URW Palladio L
% (substituting for Adobe's Palatino).
% \end{package}

% \begin{package}{pbkd8r}{bookman: URWBookmanL-DemiBold}{URW ``Base 35'' font pack for {\LaTeX}.}
% A set of fonts for use as ``drop-in'' replacements for Adobe's
% basic set, comprising: - Century Schoolbook (substituting for
% Adobe's New Century Schoolbook); - Dingbats (substituting for
% Adobe's Zapf Dingbats); - Nimbus Mono L (substituting for
% Abobe's Courier); - Nimbus Roman No9 L (substituting for
% Adobe's Times); - Nimbus Sans L (substituting for Adobe's
% Helvetica); - Standard Symbols L (substituting for Adobe's
% Symbol); - URW Bookman; - URW Chancery L Medium Italic
% (substituting for Adobe's Zapf Chancery); - URW Gothic L Book
% (substituting for Adobe's Avant Garde); and - URW Palladio L
% (substituting for Adobe's Palatino).
% \end{package}

% \begin{package}{pbkdo8r}{bookman: URWBookmanL-DemiBoldObli}{URW ``Base 35'' font pack for {\LaTeX}.}
% A set of fonts for use as ``drop-in'' replacements for Adobe's
% basic set, comprising: - Century Schoolbook (substituting for
% Adobe's New Century Schoolbook); - Dingbats (substituting for
% Adobe's Zapf Dingbats); - Nimbus Mono L (substituting for
% Abobe's Courier); - Nimbus Roman No9 L (substituting for
% Adobe's Times); - Nimbus Sans L (substituting for Adobe's
% Helvetica); - Standard Symbols L (substituting for Adobe's
% Symbol); - URW Bookman; - URW Chancery L Medium Italic
% (substituting for Adobe's Zapf Chancery); - URW Gothic L Book
% (substituting for Adobe's Avant Garde); and - URW Palladio L
% (substituting for Adobe's Palatino).
% \end{package}

% \begin{package}{pbkdi8r}{bookman: URWBookmanL-DemiBoldItal}{URW ``Base 35'' font pack for {\LaTeX}.}
% A set of fonts for use as ``drop-in'' replacements for Adobe's
% basic set, comprising: - Century Schoolbook (substituting for
% Adobe's New Century Schoolbook); - Dingbats (substituting for
% Adobe's Zapf Dingbats); - Nimbus Mono L (substituting for
% Abobe's Courier); - Nimbus Roman No9 L (substituting for
% Adobe's Times); - Nimbus Sans L (substituting for Adobe's
% Helvetica); - Standard Symbols L (substituting for Adobe's
% Symbol); - URW Bookman; - URW Chancery L Medium Italic
% (substituting for Adobe's Zapf Chancery); - URW Gothic L Book
% (substituting for Adobe's Avant Garde); and - URW Palladio L
% (substituting for Adobe's Palatino).
% \end{package}

\begin{package}{pbkl8r}{bookman: URWBookmanL-Ligh}{URW ``Base 35'' font pack for {\LaTeX}.}
A set of fonts for use as ``drop-in'' replacements for Adobe's
basic set, comprising: - Century Schoolbook (substituting for
Adobe's New Century Schoolbook); - Dingbats (substituting for
Adobe's Zapf Dingbats); - Nimbus Mono L (substituting for
Abobe's Courier); - Nimbus Roman No9 L (substituting for
Adobe's Times); - Nimbus Sans L (substituting for Adobe's
Helvetica); - Standard Symbols L (substituting for Adobe's
Symbol); - URW Bookman; - URW Chancery L Medium Italic
(substituting for Adobe's Zapf Chancery); - URW Gothic L Book
(substituting for Adobe's Avant Garde); and - URW Palladio L
(substituting for Adobe's Palatino).
\end{package}

% \begin{package}{pbklo8r}{bookman: URWBookmanL-LighObli}{URW ``Base 35'' font pack for {\LaTeX}.}
% A set of fonts for use as ``drop-in'' replacements for Adobe's
% basic set, comprising: - Century Schoolbook (substituting for
% Adobe's New Century Schoolbook); - Dingbats (substituting for
% Adobe's Zapf Dingbats); - Nimbus Mono L (substituting for
% Abobe's Courier); - Nimbus Roman No9 L (substituting for
% Adobe's Times); - Nimbus Sans L (substituting for Adobe's
% Helvetica); - Standard Symbols L (substituting for Adobe's
% Symbol); - URW Bookman; - URW Chancery L Medium Italic
% (substituting for Adobe's Zapf Chancery); - URW Gothic L Book
% (substituting for Adobe's Avant Garde); and - URW Palladio L
% (substituting for Adobe's Palatino).
% \end{package}

% \begin{package}{pbkli8r}{bookman: URWBookmanL-LighItal}{URW ``Base 35'' font pack for {\LaTeX}.}
% A set of fonts for use as ``drop-in'' replacements for Adobe's
% basic set, comprising: - Century Schoolbook (substituting for
% Adobe's New Century Schoolbook); - Dingbats (substituting for
% Adobe's Zapf Dingbats); - Nimbus Mono L (substituting for
% Abobe's Courier); - Nimbus Roman No9 L (substituting for
% Adobe's Times); - Nimbus Sans L (substituting for Adobe's
% Helvetica); - Standard Symbols L (substituting for Adobe's
% Symbol); - URW Bookman; - URW Chancery L Medium Italic
% (substituting for Adobe's Zapf Chancery); - URW Gothic L Book
% (substituting for Adobe's Avant Garde); and - URW Palladio L
% (substituting for Adobe's Palatino).
% \end{package}

\begin{package}{bchb8r}{charter: CharterBT-Bold}{Charter fonts.}
A commercial text font donated for the common good. Support for
use with {\LaTeX} is available in freenfss, part of psnfss.
\end{package}
% \begin{package}{bchbi8r}{charter: CharterBT-BoldItalic}{Charter fonts.}
% A commercial text font donated for the common good. Support for
% use with {\LaTeX} is available in freenfss, part of psnfss.
% \end{package}
\begin{package}{bchr8r}{charter: CharterBT-Roman}{Charter fonts.}
A commercial text font donated for the common good. Support for
use with {\LaTeX} is available in freenfss, part of psnfss.
\end{package}
\begin{package}{bchri8r}{charter: CharterBT-Italic}{Charter fonts.}
A commercial text font donated for the common good. Support for
use with {\LaTeX} is available in freenfss, part of psnfss.
\end{package}
% \begin{package}{bchro8r}{charter: CharterBT-Oblique}{Charter fonts.}
% A commercial text font donated for the common good. Support for
% use with {\LaTeX} is available in freenfss, part of psnfss.
% \end{package}

% \begin{package}{ecff1000}{cm-super}{CM-Super family of fonts}
% CM-Super family of fonts are Adobe Type~1 fonts that replace
% the T1/TS1-encoded Computer Modern (EC/TC), T1/TS1-encoded
% Concrete, T1/TS1-encoded CM bright and LH fonts (thus
% supporting all European languages except Greek, and all
% Cyrillic-based languages), and bringing many ameliorations in
% typesetting quality. The fonts exhibit the same metrics as the
% MetaFont-encoded originals.
% \end{package}

% \begin{package}{pcrb8r}{courier: NimbusMonL-Bold}{URW ``Base 35'' font pack for {\LaTeX}.}
% A set of fonts for use as ``drop-in'' replacements for Adobe's
% basic set, comprising: - Century Schoolbook (substituting for
% Adobe's New Century Schoolbook); - Dingbats (substituting for
% Adobe's Zapf Dingbats); - Nimbus Mono L (substituting for
% Abobe's Courier); - Nimbus Roman No9 L (substituting for
% Adobe's Times); - Nimbus Sans L (substituting for Adobe's
% Helvetica); - Standard Symbols L (substituting for Adobe's
% Symbol); - URW Bookman; - URW Chancery L Medium Italic
% (substituting for Adobe's Zapf Chancery); - URW Gothic L Book
% (substituting for Adobe's Avant Garde); and - URW Palladio L
% (substituting for Adobe's Palatino).
% \end{package}
% \begin{package}{pcrbo8r}{courier: NimbusMonL-BoldObli}{URW ``Base 35'' font pack for {\LaTeX}.}
% A set of fonts for use as ``drop-in'' replacements for Adobe's
% basic set, comprising: - Century Schoolbook (substituting for
% Adobe's New Century Schoolbook); - Dingbats (substituting for
% Adobe's Zapf Dingbats); - Nimbus Mono L (substituting for
% Abobe's Courier); - Nimbus Roman No9 L (substituting for
% Adobe's Times); - Nimbus Sans L (substituting for Adobe's
% Helvetica); - Standard Symbols L (substituting for Adobe's
% Symbol); - URW Bookman; - URW Chancery L Medium Italic
% (substituting for Adobe's Zapf Chancery); - URW Gothic L Book
% (substituting for Adobe's Avant Garde); and - URW Palladio L
% (substituting for Adobe's Palatino).
% \end{package}
\begin{package}{pcrr8r}{courier: NimbusMonL-Regu}{URW ``Base 35'' font pack for {\LaTeX}.}
A set of fonts for use as ``drop-in'' replacements for Adobe's
basic set, comprising: - Century Schoolbook (substituting for
Adobe's New Century Schoolbook); - Dingbats (substituting for
Adobe's Zapf Dingbats); - Nimbus Mono L (substituting for
Abobe's Courier); - Nimbus Roman No9 L (substituting for
Adobe's Times); - Nimbus Sans L (substituting for Adobe's
Helvetica); - Standard Symbols L (substituting for Adobe's
Symbol); - URW Bookman; - URW Chancery L Medium Italic
(substituting for Adobe's Zapf Chancery); - URW Gothic L Book
(substituting for Adobe's Avant Garde); and - URW Palladio L
(substituting for Adobe's Palatino).
\end{package}
% \begin{package}{pcrro8r}{courier: NimbusMonL-ReguObli}{URW ``Base 35'' font pack for {\LaTeX}.}
% A set of fonts for use as ``drop-in'' replacements for Adobe's
% basic set, comprising: - Century Schoolbook (substituting for
% Adobe's New Century Schoolbook); - Dingbats (substituting for
% Adobe's Zapf Dingbats); - Nimbus Mono L (substituting for
% Abobe's Courier); - Nimbus Roman No9 L (substituting for
% Adobe's Times); - Nimbus Sans L (substituting for Adobe's
% Helvetica); - Standard Symbols L (substituting for Adobe's
% Symbol); - URW Bookman; - URW Chancery L Medium Italic
% (substituting for Adobe's Zapf Chancery); - URW Gothic L Book
% (substituting for Adobe's Avant Garde); and - URW Palladio L
% (substituting for Adobe's Palatino).
% \end{package}

% \begin{package}{}{euro}{Provide Euro values for national currency amounts.}
% Converts arbitrary national currency amounts using the Euro as
% base unit, and typesets monetary amounts in almost any desired
% way. Write, e.g., ATS{17.6} to get something like '17,60 oS
% (1,28 Euro)' automatically. Conversion rates for the initial
% Euro-zone countries are already built-in. Further rates can be
% added easily. The package uses the fp package to do its sums.
% \end{package}

%\begin{package}{}{euro-ce}{Euro and CE sign font.}
%Metafont source for the symbols in several variants, designed
%to fit with Computer Modern-set text.
%\end{package}

%\begin{package}{feymr10}{eurosym}{MetaFont and macros for Euro sign.}
%The new European currency symbol for the Euro implemented in
%MetaFont, using the official European Commission dimensions,
%and providing several shapes (normal, slanted, bold, outline).
%The package also includes a {\LaTeX} package which defines the
%macro, pre-compiled tfm files, and documentation.
%\end{package}

\begin{package}{pplbj8r}{fpl: TeXPalladioL-BoldOsF}{SC and OsF fonts for URW Palladio L}
The FPL Fonts provide a set of SC/OsF fonts for URW Palladio L
which are compatible with respect to metrics with the Palatino
SC/OsF fonts from Adobe. Note that it is not my aim to exactly
reproduce the outlines of the original Adobe fonts. The SC and
OsF in the FPL Fonts were designed with the glyphs from URW
Palladio L as starting point. For some glyphs (e.g. 'o') I got
the best result by scaling and boldening. For others (e.g. 'h')
shifting selected portions of the character gave more
satisfying results. All this was done using the free font
editor FontForge. The kerning data in these fonts comes from
Walter Schmidt's improved Palatino metrics. {\LaTeX} use is
enabled by the mathpazo package, which is part of the psnfss
distribution.
\end{package}
\begin{package}{pplrc8r}{fpl: TeXPalladioL-SC}{SC and OsF fonts for URW Palladio L}
The FPL Fonts provide a set of SC/OsF fonts for URW Palladio L
which are compatible with respect to metrics with the Palatino
SC/OsF fonts from Adobe. Note that it is not my aim to exactly
reproduce the outlines of the original Adobe fonts. The SC and
OsF in the FPL Fonts were designed with the glyphs from URW
Palladio L as starting point. For some glyphs (e.g. 'o') I got
the best result by scaling and boldening. For others (e.g. 'h')
shifting selected portions of the character gave more
satisfying results. All this was done using the free font
editor FontForge. The kerning data in these fonts comes from
Walter Schmidt's improved Palatino metrics. {\LaTeX} use is
enabled by the mathpazo package, which is part of the psnfss
distribution.
\end{package}

% \begin{package}{phvb8r}{helvetic: NimbusSanL-Bold}{URW ``Base 35'' font pack for {\LaTeX}.}
% A set of fonts for use as ``drop-in'' replacements for Adobe's
% basic set, comprising: - Century Schoolbook (substituting for
% Adobe's New Century Schoolbook); - Dingbats (substituting for
% Adobe's Zapf Dingbats); - Nimbus Mono L (substituting for
% Abobe's Courier); - Nimbus Roman No9 L (substituting for
% Adobe's Times); - Nimbus Sans L (substituting for Adobe's
% Helvetica); - Standard Symbols L (substituting for Adobe's
% Symbol); - URW Bookman; - URW Chancery L Medium Italic
% (substituting for Adobe's Zapf Chancery); - URW Gothic L Book
% (substituting for Adobe's Avant Garde); and - URW Palladio L
% (substituting for Adobe's Palatino).
% \end{package}
% \begin{package}{phvb8rn}{helvetic: NimbusSanL-BoldCond}{URW ``Base 35'' font pack for {\LaTeX}.}
% A set of fonts for use as ``drop-in'' replacements for Adobe's
% basic set, comprising: - Century Schoolbook (substituting for
% Adobe's New Century Schoolbook); - Dingbats (substituting for
% Adobe's Zapf Dingbats); - Nimbus Mono L (substituting for
% Abobe's Courier); - Nimbus Roman No9 L (substituting for
% Adobe's Times); - Nimbus Sans L (substituting for Adobe's
% Helvetica); - Standard Symbols L (substituting for Adobe's
% Symbol); - URW Bookman; - URW Chancery L Medium Italic
% (substituting for Adobe's Zapf Chancery); - URW Gothic L Book
% (substituting for Adobe's Avant Garde); and - URW Palladio L
% (substituting for Adobe's Palatino).
% \end{package}
% \begin{package}{phvbo8r}{helvetic: NimbusSanL-BoldObli}{URW ``Base 35'' font pack for {\LaTeX}.}
% A set of fonts for use as ``drop-in'' replacements for Adobe's
% basic set, comprising: - Century Schoolbook (substituting for
% Adobe's New Century Schoolbook); - Dingbats (substituting for
% Adobe's Zapf Dingbats); - Nimbus Mono L (substituting for
% Abobe's Courier); - Nimbus Roman No9 L (substituting for
% Adobe's Times); - Nimbus Sans L (substituting for Adobe's
% Helvetica); - Standard Symbols L (substituting for Adobe's
% Symbol); - URW Bookman; - URW Chancery L Medium Italic
% (substituting for Adobe's Zapf Chancery); - URW Gothic L Book
% (substituting for Adobe's Avant Garde); and - URW Palladio L
% (substituting for Adobe's Palatino).
% \end{package}
% \begin{package}{phvbo8rn}{helvetic: NimbusSanL-BoldCondObli}{URW ``Base 35'' font pack for {\LaTeX}.}
% A set of fonts for use as ``drop-in'' replacements for Adobe's
% basic set, comprising: - Century Schoolbook (substituting for
% Adobe's New Century Schoolbook); - Dingbats (substituting for
% Adobe's Zapf Dingbats); - Nimbus Mono L (substituting for
% Abobe's Courier); - Nimbus Roman No9 L (substituting for
% Adobe's Times); - Nimbus Sans L (substituting for Adobe's
% Helvetica); - Standard Symbols L (substituting for Adobe's
% Symbol); - URW Bookman; - URW Chancery L Medium Italic
% (substituting for Adobe's Zapf Chancery); - URW Gothic L Book
% (substituting for Adobe's Avant Garde); and - URW Palladio L
% (substituting for Adobe's Palatino).
% \end{package}
\begin{package}{phvr8r}{helvetic: NimbusSanL-Regu}{URW ``Base 35'' font pack for {\LaTeX}.}
A set of fonts for use as ``drop-in'' replacements for Adobe's
basic set, comprising: - Century Schoolbook (substituting for
Adobe's New Century Schoolbook); - Dingbats (substituting for
Adobe's Zapf Dingbats); - Nimbus Mono L (substituting for
Abobe's Courier); - Nimbus Roman No9 L (substituting for
Adobe's Times); - Nimbus Sans L (substituting for Adobe's
Helvetica); - Standard Symbols L (substituting for Adobe's
Symbol); - URW Bookman; - URW Chancery L Medium Italic
(substituting for Adobe's Zapf Chancery); - URW Gothic L Book
(substituting for Adobe's Avant Garde); and - URW Palladio L
(substituting for Adobe's Palatino).
\end{package}
% \begin{package}{phvr8rn}{helvetic: NimbusSanL-ReguCond}{URW ``Base 35'' font pack for {\LaTeX}.}
% A set of fonts for use as ``drop-in'' replacements for Adobe's
% basic set, comprising: - Century Schoolbook (substituting for
% Adobe's New Century Schoolbook); - Dingbats (substituting for
% Adobe's Zapf Dingbats); - Nimbus Mono L (substituting for
% Abobe's Courier); - Nimbus Roman No9 L (substituting for
% Adobe's Times); - Nimbus Sans L (substituting for Adobe's
% Helvetica); - Standard Symbols L (substituting for Adobe's
% Symbol); - URW Bookman; - URW Chancery L Medium Italic
% (substituting for Adobe's Zapf Chancery); - URW Gothic L Book
% (substituting for Adobe's Avant Garde); and - URW Palladio L
% (substituting for Adobe's Palatino).
% \end{package}
% \begin{package}{phvro8r}{helvetic: NimbusSanL-ReguObli}{URW ``Base 35'' font pack for {\LaTeX}.}
% A set of fonts for use as ``drop-in'' replacements for Adobe's
% basic set, comprising: - Century Schoolbook (substituting for
% Adobe's New Century Schoolbook); - Dingbats (substituting for
% Adobe's Zapf Dingbats); - Nimbus Mono L (substituting for
% Abobe's Courier); - Nimbus Roman No9 L (substituting for
% Adobe's Times); - Nimbus Sans L (substituting for Adobe's
% Helvetica); - Standard Symbols L (substituting for Adobe's
% Symbol); - URW Bookman; - URW Chancery L Medium Italic
% (substituting for Adobe's Zapf Chancery); - URW Gothic L Book
% (substituting for Adobe's Avant Garde); and - URW Palladio L
% (substituting for Adobe's Palatino).
% \end{package}
% \begin{package}{phvro8rn}{helvetic: NimbusSanL-ReguCondObli}{URW ``Base 35'' font pack for {\LaTeX}.}
% A set of fonts for use as ``drop-in'' replacements for Adobe's
% basic set, comprising: - Century Schoolbook (substituting for
% Adobe's New Century Schoolbook); - Dingbats (substituting for
% Adobe's Zapf Dingbats); - Nimbus Mono L (substituting for
% Abobe's Courier); - Nimbus Roman No9 L (substituting for
% Adobe's Times); - Nimbus Sans L (substituting for Adobe's
% Helvetica); - Standard Symbols L (substituting for Adobe's
% Symbol); - URW Bookman; - URW Chancery L Medium Italic
% (substituting for Adobe's Zapf Chancery); - URW Gothic L Book
% (substituting for Adobe's Avant Garde); and - URW Palladio L
% (substituting for Adobe's Palatino).
% \end{package}

\begin{package}{ec-lmb10}{lm: LMRomanDemi10-Regular}{Latin modern fonts in outline formats.}
The Latin Modern family of fonts consists of 72 text fonts and
20 mathematics fonts, and is based on the Computer Modern fonts
released into public domain by AMS (copyright (c) 1997 AMS).
The lm font set contains a lot of additional characters, mainly
accented ones, but not exclusively. There is one set of fonts,
available both in Adobe Type~1 format (*.pfb) and in OpenType
format (*.otf). There are five sets of {\TeX} Font Metric files,
corresponding to: Cork encoding (cork-*.tfm); QX encoding (qx-
*.tfm); {\TeX}'n'ANSI aka LY1 encoding (texnansi-*.tfm); T5
(Vietnamese) encoding (t5-*.tfm); and Text Companion for EC
fonts aka TS1 (ts1-*.tfm).
\end{package}
% \begin{package}{ec-lmbo10}{lm: LMRomanDemi10-Oblique}{Latin modern fonts in outline formats.}
% The Latin Modern family of fonts consists of 72 text fonts and
% 20 mathematics fonts, and is based on the Computer Modern fonts
% released into public domain by AMS (copyright (c) 1997 AMS).
% The lm font set contains a lot of additional characters, mainly
% accented ones, but not exclusively. There is one set of fonts,
% available both in Adobe Type~1 format (*.pfb) and in OpenType
% format (*.otf). There are five sets of {\TeX} Font Metric files,
% corresponding to: Cork encoding (cork-*.tfm); QX encoding (qx-
% *.tfm); {\TeX}'n'ANSI aka LY1 encoding (texnansi-*.tfm); T5
% (Vietnamese) encoding (t5-*.tfm); and Text Companion for EC
% fonts aka TS1 (ts1-*.tfm).
% \end{package}
\begin{package}{ec-lmbx10}{lm: LMRoman10-Bold}{Latin modern fonts in outline formats.}
The Latin Modern family of fonts consists of 72 text fonts and
20 mathematics fonts, and is based on the Computer Modern fonts
released into public domain by AMS (copyright (c) 1997 AMS).
The lm font set contains a lot of additional characters, mainly
accented ones, but not exclusively. There is one set of fonts,
available both in Adobe Type~1 format (*.pfb) and in OpenType
format (*.otf). There are five sets of {\TeX} Font Metric files,
corresponding to: Cork encoding (cork-*.tfm); QX encoding (qx-
*.tfm); {\TeX}'n'ANSI aka LY1 encoding (texnansi-*.tfm); T5
(Vietnamese) encoding (t5-*.tfm); and Text Companion for EC
fonts aka TS1 (ts1-*.tfm).
\end{package}
% \begin{package}{ec-lmbxi10}{lm: LMRoman10-BoldItalic}{Latin modern fonts in outline formats.}
% The Latin Modern family of fonts consists of 72 text fonts and
% 20 mathematics fonts, and is based on the Computer Modern fonts
% released into public domain by AMS (copyright (c) 1997 AMS).
% The lm font set contains a lot of additional characters, mainly
% accented ones, but not exclusively. There is one set of fonts,
% available both in Adobe Type~1 format (*.pfb) and in OpenType
% format (*.otf). There are five sets of {\TeX} Font Metric files,
% corresponding to: Cork encoding (cork-*.tfm); QX encoding (qx-
% *.tfm); {\TeX}'n'ANSI aka LY1 encoding (texnansi-*.tfm); T5
% (Vietnamese) encoding (t5-*.tfm); and Text Companion for EC
% fonts aka TS1 (ts1-*.tfm).
% \end{package}
\begin{package}{ec-lmbxo10}{lm: LMRomanSlant10-Bold}{Latin modern fonts in outline formats.}
The Latin Modern family of fonts consists of 72 text fonts and
20 mathematics fonts, and is based on the Computer Modern fonts
released into public domain by AMS (copyright (c) 1997 AMS).
The lm font set contains a lot of additional characters, mainly
accented ones, but not exclusively. There is one set of fonts,
available both in Adobe Type~1 format (*.pfb) and in OpenType
format (*.otf). There are five sets of {\TeX} Font Metric files,
corresponding to: Cork encoding (cork-*.tfm); QX encoding (qx-
*.tfm); {\TeX}'n'ANSI aka LY1 encoding (texnansi-*.tfm); T5
(Vietnamese) encoding (t5-*.tfm); and Text Companion for EC
fonts aka TS1 (ts1-*.tfm).
\end{package}
\begin{package}{ec-lmcsc10}{lm: LMRomanCaps10-Regular}{Latin modern fonts in outline formats.}
The Latin Modern family of fonts consists of 72 text fonts and
20 mathematics fonts, and is based on the Computer Modern fonts
released into public domain by AMS (copyright (c) 1997 AMS).
The lm font set contains a lot of additional characters, mainly
accented ones, but not exclusively. There is one set of fonts,
available both in Adobe Type~1 format (*.pfb) and in OpenType
format (*.otf). There are five sets of {\TeX} Font Metric files,
corresponding to: Cork encoding (cork-*.tfm); QX encoding (qx-
*.tfm); {\TeX}'n'ANSI aka LY1 encoding (texnansi-*.tfm); T5
(Vietnamese) encoding (t5-*.tfm); and Text Companion for EC
fonts aka TS1 (ts1-*.tfm).
\end{package}
% \begin{package}{ec-lmcsco10}{lm: LMRomanCaps10-Oblique}{Latin modern fonts in outline formats.}
% The Latin Modern family of fonts consists of 72 text fonts and
% 20 mathematics fonts, and is based on the Computer Modern fonts
% released into public domain by AMS (copyright (c) 1997 AMS).
% The lm font set contains a lot of additional characters, mainly
% accented ones, but not exclusively. There is one set of fonts,
% available both in Adobe Type~1 format (*.pfb) and in OpenType
% format (*.otf). There are five sets of {\TeX} Font Metric files,
% corresponding to: Cork encoding (cork-*.tfm); QX encoding (qx-
% *.tfm); {\TeX}'n'ANSI aka LY1 encoding (texnansi-*.tfm); T5
% (Vietnamese) encoding (t5-*.tfm); and Text Companion for EC
% fonts aka TS1 (ts1-*.tfm).
% \end{package}
\begin{package}{ec-lmdunh10}{lm: LMRomanDunh10-Regular}{Latin modern fonts in outline formats.}
The Latin Modern family of fonts consists of 72 text fonts and
20 mathematics fonts, and is based on the Computer Modern fonts
released into public domain by AMS (copyright (c) 1997 AMS).
The lm font set contains a lot of additional characters, mainly
accented ones, but not exclusively. There is one set of fonts,
available both in Adobe Type~1 format (*.pfb) and in OpenType
format (*.otf). There are five sets of {\TeX} Font Metric files,
corresponding to: Cork encoding (cork-*.tfm); QX encoding (qx-
*.tfm); {\TeX}'n'ANSI aka LY1 encoding (texnansi-*.tfm); T5
(Vietnamese) encoding (t5-*.tfm); and Text Companion for EC
fonts aka TS1 (ts1-*.tfm).
\end{package}
% \begin{package}{ec-lmduno10}{lm: LMRomanDunh10-Oblique}{Latin modern fonts in outline formats.}
% The Latin Modern family of fonts consists of 72 text fonts and
% 20 mathematics fonts, and is based on the Computer Modern fonts
% released into public domain by AMS (copyright (c) 1997 AMS).
% The lm font set contains a lot of additional characters, mainly
% accented ones, but not exclusively. There is one set of fonts,
% available both in Adobe Type~1 format (*.pfb) and in OpenType
% format (*.otf). There are five sets of {\TeX} Font Metric files,
% corresponding to: Cork encoding (cork-*.tfm); QX encoding (qx-
% *.tfm); {\TeX}'n'ANSI aka LY1 encoding (texnansi-*.tfm); T5
% (Vietnamese) encoding (t5-*.tfm); and Text Companion for EC
% fonts aka TS1 (ts1-*.tfm).
% \end{package}
\begin{package}{ec-lmr10}{lm: LMRoman10-Regular}{Latin modern fonts in outline formats.}
The Latin Modern family of fonts consists of 72 text fonts and
20 mathematics fonts, and is based on the Computer Modern fonts
released into public domain by AMS (copyright (c) 1997 AMS).
The lm font set contains a lot of additional characters, mainly
accented ones, but not exclusively. There is one set of fonts,
available both in Adobe Type~1 format (*.pfb) and in OpenType
format (*.otf). There are five sets of {\TeX} Font Metric files,
corresponding to: Cork encoding (cork-*.tfm); QX encoding (qx-
*.tfm); {\TeX}'n'ANSI aka LY1 encoding (texnansi-*.tfm); T5
(Vietnamese) encoding (t5-*.tfm); and Text Companion for EC
fonts aka TS1 (ts1-*.tfm).
\end{package}
\begin{package}{ec-lmri10}{lm: LMRoman10-Italic}{Latin modern fonts in outline formats.}
The Latin Modern family of fonts consists of 72 text fonts and
20 mathematics fonts, and is based on the Computer Modern fonts
released into public domain by AMS (copyright (c) 1997 AMS).
The lm font set contains a lot of additional characters, mainly
accented ones, but not exclusively. There is one set of fonts,
available both in Adobe Type~1 format (*.pfb) and in OpenType
format (*.otf). There are five sets of {\TeX} Font Metric files,
corresponding to: Cork encoding (cork-*.tfm); QX encoding (qx-
*.tfm); {\TeX}'n'ANSI aka LY1 encoding (texnansi-*.tfm); T5
(Vietnamese) encoding (t5-*.tfm); and Text Companion for EC
fonts aka TS1 (ts1-*.tfm).
\end{package}
\begin{package}{ec-lmro10}{lm: LMRomanSlant10-Regular}{Latin modern fonts in outline formats.}
The Latin Modern family of fonts consists of 72 text fonts and
20 mathematics fonts, and is based on the Computer Modern fonts
released into public domain by AMS (copyright (c) 1997 AMS).
The lm font set contains a lot of additional characters, mainly
accented ones, but not exclusively. There is one set of fonts,
available both in Adobe Type~1 format (*.pfb) and in OpenType
format (*.otf). There are five sets of {\TeX} Font Metric files,
corresponding to: Cork encoding (cork-*.tfm); QX encoding (qx-
*.tfm); {\TeX}'n'ANSI aka LY1 encoding (texnansi-*.tfm); T5
(Vietnamese) encoding (t5-*.tfm); and Text Companion for EC
fonts aka TS1 (ts1-*.tfm).
\end{package}
\begin{package}{ec-lmss10}{lm: LMSans10-Regular}{Latin modern fonts in outline formats.}
The Latin Modern family of fonts consists of 72 text fonts and
20 mathematics fonts, and is based on the Computer Modern fonts
released into public domain by AMS (copyright (c) 1997 AMS).
The lm font set contains a lot of additional characters, mainly
accented ones, but not exclusively. There is one set of fonts,
available both in Adobe Type~1 format (*.pfb) and in OpenType
format (*.otf). There are five sets of {\TeX} Font Metric files,
corresponding to: Cork encoding (cork-*.tfm); QX encoding (qx-
*.tfm); {\TeX}'n'ANSI aka LY1 encoding (texnansi-*.tfm); T5
(Vietnamese) encoding (t5-*.tfm); and Text Companion for EC
fonts aka TS1 (ts1-*.tfm).
\end{package}
% \begin{package}{ec-lmssbo10}{lm: LMSans10-BoldOblique}{Latin modern fonts in outline formats.}
% The Latin Modern family of fonts consists of 72 text fonts and
% 20 mathematics fonts, and is based on the Computer Modern fonts
% released into public domain by AMS (copyright (c) 1997 AMS).
% The lm font set contains a lot of additional characters, mainly
% accented ones, but not exclusively. There is one set of fonts,
% available both in Adobe Type~1 format (*.pfb) and in OpenType
% format (*.otf). There are five sets of {\TeX} Font Metric files,
% corresponding to: Cork encoding (cork-*.tfm); QX encoding (qx-
% *.tfm); {\TeX}'n'ANSI aka LY1 encoding (texnansi-*.tfm); T5
% (Vietnamese) encoding (t5-*.tfm); and Text Companion for EC
% fonts aka TS1 (ts1-*.tfm).
% \end{package}
\begin{package}{ec-lmssbx10}{lm: LMSans10-Bold}{Latin modern fonts in outline formats.}
The Latin Modern family of fonts consists of 72 text fonts and
20 mathematics fonts, and is based on the Computer Modern fonts
released into public domain by AMS (copyright (c) 1997 AMS).
The lm font set contains a lot of additional characters, mainly
accented ones, but not exclusively. There is one set of fonts,
available both in Adobe Type~1 format (*.pfb) and in OpenType
format (*.otf). There are five sets of {\TeX} Font Metric files,
corresponding to: Cork encoding (cork-*.tfm); QX encoding (qx-
*.tfm); {\TeX}'n'ANSI aka LY1 encoding (texnansi-*.tfm); T5
(Vietnamese) encoding (t5-*.tfm); and Text Companion for EC
fonts aka TS1 (ts1-*.tfm).
\end{package}
\begin{package}{ec-lmssdc10}{lm: LMSansDemiCond10-Regular}{Latin modern fonts in outline formats.}
The Latin Modern family of fonts consists of 72 text fonts and
20 mathematics fonts, and is based on the Computer Modern fonts
released into public domain by AMS (copyright (c) 1997 AMS).
The lm font set contains a lot of additional characters, mainly
accented ones, but not exclusively. There is one set of fonts,
available both in Adobe Type~1 format (*.pfb) and in OpenType
format (*.otf). There are five sets of {\TeX} Font Metric files,
corresponding to: Cork encoding (cork-*.tfm); QX encoding (qx-
*.tfm); {\TeX}'n'ANSI aka LY1 encoding (texnansi-*.tfm); T5
(Vietnamese) encoding (t5-*.tfm); and Text Companion for EC
fonts aka TS1 (ts1-*.tfm).
\end{package}
% \begin{package}{ec-lmssdo10}{lm: LMSansDemiCond10-Oblique}{Latin modern fonts in outline formats.}
% The Latin Modern family of fonts consists of 72 text fonts and
% 20 mathematics fonts, and is based on the Computer Modern fonts
% released into public domain by AMS (copyright (c) 1997 AMS).
% The lm font set contains a lot of additional characters, mainly
% accented ones, but not exclusively. There is one set of fonts,
% available both in Adobe Type~1 format (*.pfb) and in OpenType
% format (*.otf). There are five sets of {\TeX} Font Metric files,
% corresponding to: Cork encoding (cork-*.tfm); QX encoding (qx-
% *.tfm); {\TeX}'n'ANSI aka LY1 encoding (texnansi-*.tfm); T5
% (Vietnamese) encoding (t5-*.tfm); and Text Companion for EC
% fonts aka TS1 (ts1-*.tfm).
% \end{package}
% \begin{package}{ec-lmsso10}{lm: LMSans10-Oblique}{Latin modern fonts in outline formats.}
% The Latin Modern family of fonts consists of 72 text fonts and
% 20 mathematics fonts, and is based on the Computer Modern fonts
% released into public domain by AMS (copyright (c) 1997 AMS).
% The lm font set contains a lot of additional characters, mainly
% accented ones, but not exclusively. There is one set of fonts,
% available both in Adobe Type~1 format (*.pfb) and in OpenType
% format (*.otf). There are five sets of {\TeX} Font Metric files,
% corresponding to: Cork encoding (cork-*.tfm); QX encoding (qx-
% *.tfm); {\TeX}'n'ANSI aka LY1 encoding (texnansi-*.tfm); T5
% (Vietnamese) encoding (t5-*.tfm); and Text Companion for EC
% fonts aka TS1 (ts1-*.tfm).
% \end{package}
\begin{package}{ec-lmssq8}{lm: LMSansQuot8-Regular}{Latin modern fonts in outline formats.}
The Latin Modern family of fonts consists of 72 text fonts and
20 mathematics fonts, and is based on the Computer Modern fonts
released into public domain by AMS (copyright (c) 1997 AMS).
The lm font set contains a lot of additional characters, mainly
accented ones, but not exclusively. There is one set of fonts,
available both in Adobe Type~1 format (*.pfb) and in OpenType
format (*.otf). There are five sets of {\TeX} Font Metric files,
corresponding to: Cork encoding (cork-*.tfm); QX encoding (qx-
*.tfm); {\TeX}'n'ANSI aka LY1 encoding (texnansi-*.tfm); T5
(Vietnamese) encoding (t5-*.tfm); and Text Companion for EC
fonts aka TS1 (ts1-*.tfm).
\end{package}
% \begin{package}{ec-lmssqbo8}{lm: LMSansQuot8-BoldOblique}{Latin modern fonts in outline formats.}
% The Latin Modern family of fonts consists of 72 text fonts and
% 20 mathematics fonts, and is based on the Computer Modern fonts
% released into public domain by AMS (copyright (c) 1997 AMS).
% The lm font set contains a lot of additional characters, mainly
% accented ones, but not exclusively. There is one set of fonts,
% available both in Adobe Type~1 format (*.pfb) and in OpenType
% format (*.otf). There are five sets of {\TeX} Font Metric files,
% corresponding to: Cork encoding (cork-*.tfm); QX encoding (qx-
% *.tfm); {\TeX}'n'ANSI aka LY1 encoding (texnansi-*.tfm); T5
% (Vietnamese) encoding (t5-*.tfm); and Text Companion for EC
% fonts aka TS1 (ts1-*.tfm).
% \end{package}
\begin{package}{ec-lmssqbx8}{lm: LMSansQuot8-Bold}{Latin modern fonts in outline formats.}
The Latin Modern family of fonts consists of 72 text fonts and
20 mathematics fonts, and is based on the Computer Modern fonts
released into public domain by AMS (copyright (c) 1997 AMS).
The lm font set contains a lot of additional characters, mainly
accented ones, but not exclusively. There is one set of fonts,
available both in Adobe Type~1 format (*.pfb) and in OpenType
format (*.otf). There are five sets of {\TeX} Font Metric files,
corresponding to: Cork encoding (cork-*.tfm); QX encoding (qx-
*.tfm); {\TeX}'n'ANSI aka LY1 encoding (texnansi-*.tfm); T5
(Vietnamese) encoding (t5-*.tfm); and Text Companion for EC
fonts aka TS1 (ts1-*.tfm).
\end{package}
% \begin{package}{ec-lmssqo8}{lm: LMSansQuot8-Oblique}{Latin modern fonts in outline formats.}
% The Latin Modern family of fonts consists of 72 text fonts and
% 20 mathematics fonts, and is based on the Computer Modern fonts
% released into public domain by AMS (copyright (c) 1997 AMS).
% The lm font set contains a lot of additional characters, mainly
% accented ones, but not exclusively. There is one set of fonts,
% available both in Adobe Type~1 format (*.pfb) and in OpenType
% format (*.otf). There are five sets of {\TeX} Font Metric files,
% corresponding to: Cork encoding (cork-*.tfm); QX encoding (qx-
% *.tfm); {\TeX}'n'ANSI aka LY1 encoding (texnansi-*.tfm); T5
% (Vietnamese) encoding (t5-*.tfm); and Text Companion for EC
% fonts aka TS1 (ts1-*.tfm).
% \end{package}
\begin{package}{ec-lmtcsc10}{lm: LMMonoCaps10-Regular}{Latin modern fonts in outline formats.}
The Latin Modern family of fonts consists of 72 text fonts and
20 mathematics fonts, and is based on the Computer Modern fonts
released into public domain by AMS (copyright (c) 1997 AMS).
The lm font set contains a lot of additional characters, mainly
accented ones, but not exclusively. There is one set of fonts,
available both in Adobe Type~1 format (*.pfb) and in OpenType
format (*.otf). There are five sets of {\TeX} Font Metric files,
corresponding to: Cork encoding (cork-*.tfm); QX encoding (qx-
*.tfm); {\TeX}'n'ANSI aka LY1 encoding (texnansi-*.tfm); T5
(Vietnamese) encoding (t5-*.tfm); and Text Companion for EC
fonts aka TS1 (ts1-*.tfm).
\end{package}
% \begin{package}{ec-lmtcso10}{lm: LMMonoCaps10-Oblique}{Latin modern fonts in outline formats.}
% The Latin Modern family of fonts consists of 72 text fonts and
% 20 mathematics fonts, and is based on the Computer Modern fonts
% released into public domain by AMS (copyright (c) 1997 AMS).
% The lm font set contains a lot of additional characters, mainly
% accented ones, but not exclusively. There is one set of fonts,
% available both in Adobe Type~1 format (*.pfb) and in OpenType
% format (*.otf). There are five sets of {\TeX} Font Metric files,
% corresponding to: Cork encoding (cork-*.tfm); QX encoding (qx-
% *.tfm); {\TeX}'n'ANSI aka LY1 encoding (texnansi-*.tfm); T5
% (Vietnamese) encoding (t5-*.tfm); and Text Companion for EC
% fonts aka TS1 (ts1-*.tfm).
% \end{package}
\begin{package}{ec-lmtk10}{lm: LMMonoLt10-Bold}{Latin modern fonts in outline formats.}
The Latin Modern family of fonts consists of 72 text fonts and
20 mathematics fonts, and is based on the Computer Modern fonts
released into public domain by AMS (copyright (c) 1997 AMS).
The lm font set contains a lot of additional characters, mainly
accented ones, but not exclusively. There is one set of fonts,
available both in Adobe Type~1 format (*.pfb) and in OpenType
format (*.otf). There are five sets of {\TeX} Font Metric files,
corresponding to: Cork encoding (cork-*.tfm); QX encoding (qx-
*.tfm); {\TeX}'n'ANSI aka LY1 encoding (texnansi-*.tfm); T5
(Vietnamese) encoding (t5-*.tfm); and Text Companion for EC
fonts aka TS1 (ts1-*.tfm).
\end{package}
% \begin{package}{ec-lmtko10}{lm: LMMonoLt10-BoldOblique}{Latin modern fonts in outline formats.}
% The Latin Modern family of fonts consists of 72 text fonts and
% 20 mathematics fonts, and is based on the Computer Modern fonts
% released into public domain by AMS (copyright (c) 1997 AMS).
% The lm font set contains a lot of additional characters, mainly
% accented ones, but not exclusively. There is one set of fonts,
% available both in Adobe Type~1 format (*.pfb) and in OpenType
% format (*.otf). There are five sets of {\TeX} Font Metric files,
% corresponding to: Cork encoding (cork-*.tfm); QX encoding (qx-
% *.tfm); {\TeX}'n'ANSI aka LY1 encoding (texnansi-*.tfm); T5
% (Vietnamese) encoding (t5-*.tfm); and Text Companion for EC
% fonts aka TS1 (ts1-*.tfm).
% \end{package}
\begin{package}{ec-lmtl10}{lm: LMMonoLt10-Regular}{Latin modern fonts in outline formats.}
The Latin Modern family of fonts consists of 72 text fonts and
20 mathematics fonts, and is based on the Computer Modern fonts
released into public domain by AMS (copyright (c) 1997 AMS).
The lm font set contains a lot of additional characters, mainly
accented ones, but not exclusively. There is one set of fonts,
available both in Adobe Type~1 format (*.pfb) and in OpenType
format (*.otf). There are five sets of {\TeX} Font Metric files,
corresponding to: Cork encoding (cork-*.tfm); QX encoding (qx-
*.tfm); {\TeX}'n'ANSI aka LY1 encoding (texnansi-*.tfm); T5
(Vietnamese) encoding (t5-*.tfm); and Text Companion for EC
fonts aka TS1 (ts1-*.tfm).
\end{package}
\begin{package}{ec-lmtlc10}{lm: LMMonoLtCond10-Regular}{Latin modern fonts in outline formats.}
The Latin Modern family of fonts consists of 72 text fonts and
20 mathematics fonts, and is based on the Computer Modern fonts
released into public domain by AMS (copyright (c) 1997 AMS).
The lm font set contains a lot of additional characters, mainly
accented ones, but not exclusively. There is one set of fonts,
available both in Adobe Type~1 format (*.pfb) and in OpenType
format (*.otf). There are five sets of {\TeX} Font Metric files,
corresponding to: Cork encoding (cork-*.tfm); QX encoding (qx-
*.tfm); {\TeX}'n'ANSI aka LY1 encoding (texnansi-*.tfm); T5
(Vietnamese) encoding (t5-*.tfm); and Text Companion for EC
fonts aka TS1 (ts1-*.tfm).
\end{package}
% \begin{package}{ec-lmtlco10}{lm: LMMonoLtCond10-Oblique}{Latin modern fonts in outline formats.}
% The Latin Modern family of fonts consists of 72 text fonts and
% 20 mathematics fonts, and is based on the Computer Modern fonts
% released into public domain by AMS (copyright (c) 1997 AMS).
% The lm font set contains a lot of additional characters, mainly
% accented ones, but not exclusively. There is one set of fonts,
% available both in Adobe Type~1 format (*.pfb) and in OpenType
% format (*.otf). There are five sets of {\TeX} Font Metric files,
% corresponding to: Cork encoding (cork-*.tfm); QX encoding (qx-
% *.tfm); {\TeX}'n'ANSI aka LY1 encoding (texnansi-*.tfm); T5
% (Vietnamese) encoding (t5-*.tfm); and Text Companion for EC
% fonts aka TS1 (ts1-*.tfm).
% \end{package}
% \begin{package}{ec-lmtlo10}{lm: LMMonoLt10-Oblique}{Latin modern fonts in outline formats.}
% The Latin Modern family of fonts consists of 72 text fonts and
% 20 mathematics fonts, and is based on the Computer Modern fonts
% released into public domain by AMS (copyright (c) 1997 AMS).
% The lm font set contains a lot of additional characters, mainly
% accented ones, but not exclusively. There is one set of fonts,
% available both in Adobe Type~1 format (*.pfb) and in OpenType
% format (*.otf). There are five sets of {\TeX} Font Metric files,
% corresponding to: Cork encoding (cork-*.tfm); QX encoding (qx-
% *.tfm); {\TeX}'n'ANSI aka LY1 encoding (texnansi-*.tfm); T5
% (Vietnamese) encoding (t5-*.tfm); and Text Companion for EC
% fonts aka TS1 (ts1-*.tfm).
% \end{package}
\begin{package}{ec-lmtt10}{lm: LMMono10-Regular}{Latin modern fonts in outline formats.}
The Latin Modern family of fonts consists of 72 text fonts and
20 mathematics fonts, and is based on the Computer Modern fonts
released into public domain by AMS (copyright (c) 1997 AMS).
The lm font set contains a lot of additional characters, mainly
accented ones, but not exclusively. There is one set of fonts,
available both in Adobe Type~1 format (*.pfb) and in OpenType
format (*.otf). There are five sets of {\TeX} Font Metric files,
corresponding to: Cork encoding (cork-*.tfm); QX encoding (qx-
*.tfm); {\TeX}'n'ANSI aka LY1 encoding (texnansi-*.tfm); T5
(Vietnamese) encoding (t5-*.tfm); and Text Companion for EC
fonts aka TS1 (ts1-*.tfm).
\end{package}
\begin{package}{ec-lmtti10}{lm: LMMono10-Italic}{Latin modern fonts in outline formats.}
The Latin Modern family of fonts consists of 72 text fonts and
20 mathematics fonts, and is based on the Computer Modern fonts
released into public domain by AMS (copyright (c) 1997 AMS).
The lm font set contains a lot of additional characters, mainly
accented ones, but not exclusively. There is one set of fonts,
available both in Adobe Type~1 format (*.pfb) and in OpenType
format (*.otf). There are five sets of {\TeX} Font Metric files,
corresponding to: Cork encoding (cork-*.tfm); QX encoding (qx-
*.tfm); {\TeX}'n'ANSI aka LY1 encoding (texnansi-*.tfm); T5
(Vietnamese) encoding (t5-*.tfm); and Text Companion for EC
fonts aka TS1 (ts1-*.tfm).
\end{package}
\begin{package}{ec-lmtto10}{lm: LMMonoSlant10-Regular}{Latin modern fonts in outline formats.}
The Latin Modern family of fonts consists of 72 text fonts and
20 mathematics fonts, and is based on the Computer Modern fonts
released into public domain by AMS (copyright (c) 1997 AMS).
The lm font set contains a lot of additional characters, mainly
accented ones, but not exclusively. There is one set of fonts,
available both in Adobe Type~1 format (*.pfb) and in OpenType
format (*.otf). There are five sets of {\TeX} Font Metric files,
corresponding to: Cork encoding (cork-*.tfm); QX encoding (qx-
*.tfm); {\TeX}'n'ANSI aka LY1 encoding (texnansi-*.tfm); T5
(Vietnamese) encoding (t5-*.tfm); and Text Companion for EC
fonts aka TS1 (ts1-*.tfm).
\end{package}
\begin{package}{ec-lmu10}{lm: LMRomanUnsl10-Regular}{Latin modern fonts in outline formats.}
The Latin Modern family of fonts consists of 72 text fonts and
20 mathematics fonts, and is based on the Computer Modern fonts
released into public domain by AMS (copyright (c) 1997 AMS).
The lm font set contains a lot of additional characters, mainly
accented ones, but not exclusively. There is one set of fonts,
available both in Adobe Type~1 format (*.pfb) and in OpenType
format (*.otf). There are five sets of {\TeX} Font Metric files,
corresponding to: Cork encoding (cork-*.tfm); QX encoding (qx-
*.tfm); {\TeX}'n'ANSI aka LY1 encoding (texnansi-*.tfm); T5
(Vietnamese) encoding (t5-*.tfm); and Text Companion for EC
fonts aka TS1 (ts1-*.tfm).
\end{package}
\begin{package}{ec-lmvtk10}{lm: LMMonoPropLt10-Bold}{Latin modern fonts in outline formats.}
The Latin Modern family of fonts consists of 72 text fonts and
20 mathematics fonts, and is based on the Computer Modern fonts
released into public domain by AMS (copyright (c) 1997 AMS).
The lm font set contains a lot of additional characters, mainly
accented ones, but not exclusively. There is one set of fonts,
available both in Adobe Type~1 format (*.pfb) and in OpenType
format (*.otf). There are five sets of {\TeX} Font Metric files,
corresponding to: Cork encoding (cork-*.tfm); QX encoding (qx-
*.tfm); {\TeX}'n'ANSI aka LY1 encoding (texnansi-*.tfm); T5
(Vietnamese) encoding (t5-*.tfm); and Text Companion for EC
fonts aka TS1 (ts1-*.tfm).
\end{package}
% \begin{package}{ec-lmvtko10}{lm: LMMonoPropLt10-BoldOblique}{Latin modern fonts in outline formats.}
% The Latin Modern family of fonts consists of 72 text fonts and
% 20 mathematics fonts, and is based on the Computer Modern fonts
% released into public domain by AMS (copyright (c) 1997 AMS).
% The lm font set contains a lot of additional characters, mainly
% accented ones, but not exclusively. There is one set of fonts,
% available both in Adobe Type~1 format (*.pfb) and in OpenType
% format (*.otf). There are five sets of {\TeX} Font Metric files,
% corresponding to: Cork encoding (cork-*.tfm); QX encoding (qx-
% *.tfm); {\TeX}'n'ANSI aka LY1 encoding (texnansi-*.tfm); T5
% (Vietnamese) encoding (t5-*.tfm); and Text Companion for EC
% fonts aka TS1 (ts1-*.tfm).
% \end{package}
\begin{package}{ec-lmvtl10}{lm: LMMonoPropLt10-Regular}{Latin modern fonts in outline formats.}
The Latin Modern family of fonts consists of 72 text fonts and
20 mathematics fonts, and is based on the Computer Modern fonts
released into public domain by AMS (copyright (c) 1997 AMS).
The lm font set contains a lot of additional characters, mainly
accented ones, but not exclusively. There is one set of fonts,
available both in Adobe Type~1 format (*.pfb) and in OpenType
format (*.otf). There are five sets of {\TeX} Font Metric files,
corresponding to: Cork encoding (cork-*.tfm); QX encoding (qx-
*.tfm); {\TeX}'n'ANSI aka LY1 encoding (texnansi-*.tfm); T5
(Vietnamese) encoding (t5-*.tfm); and Text Companion for EC
fonts aka TS1 (ts1-*.tfm).
\end{package}
% \begin{package}{ec-lmvtlo10}{lm: LMMonoPropLt10-Oblique}{Latin modern fonts in outline formats.}
% The Latin Modern family of fonts consists of 72 text fonts and
% 20 mathematics fonts, and is based on the Computer Modern fonts
% released into public domain by AMS (copyright (c) 1997 AMS).
% The lm font set contains a lot of additional characters, mainly
% accented ones, but not exclusively. There is one set of fonts,
% available both in Adobe Type~1 format (*.pfb) and in OpenType
% format (*.otf). There are five sets of {\TeX} Font Metric files,
% corresponding to: Cork encoding (cork-*.tfm); QX encoding (qx-
% *.tfm); {\TeX}'n'ANSI aka LY1 encoding (texnansi-*.tfm); T5
% (Vietnamese) encoding (t5-*.tfm); and Text Companion for EC
% fonts aka TS1 (ts1-*.tfm).
% \end{package}
\begin{package}{ec-lmvtt10}{lm: LMMonoProp10-Regular}{Latin modern fonts in outline formats.}
The Latin Modern family of fonts consists of 72 text fonts and
20 mathematics fonts, and is based on the Computer Modern fonts
released into public domain by AMS (copyright (c) 1997 AMS).
The lm font set contains a lot of additional characters, mainly
accented ones, but not exclusively. There is one set of fonts,
available both in Adobe Type~1 format (*.pfb) and in OpenType
format (*.otf). There are five sets of {\TeX} Font Metric files,
corresponding to: Cork encoding (cork-*.tfm); QX encoding (qx-
*.tfm); {\TeX}'n'ANSI aka LY1 encoding (texnansi-*.tfm); T5
(Vietnamese) encoding (t5-*.tfm); and Text Companion for EC
fonts aka TS1 (ts1-*.tfm).
\end{package}
% \begin{package}{ec-lmvtto10}{lm: LMMonoProp10-Oblique}{Latin modern fonts in outline formats.}
% The Latin Modern family of fonts consists of 72 text fonts and
% 20 mathematics fonts, and is based on the Computer Modern fonts
% released into public domain by AMS (copyright (c) 1997 AMS).
% The lm font set contains a lot of additional characters, mainly
% accented ones, but not exclusively. There is one set of fonts,
% available both in Adobe Type~1 format (*.pfb) and in OpenType
% format (*.otf). There are five sets of {\TeX} Font Metric files,
% corresponding to: Cork encoding (cork-*.tfm); QX encoding (qx-
% *.tfm); {\TeX}'n'ANSI aka LY1 encoding (texnansi-*.tfm); T5
% (Vietnamese) encoding (t5-*.tfm); and Text Companion for EC
% fonts aka TS1 (ts1-*.tfm).
% \end{package}

%\begin{package}{}{lm-math}{OpenType maths fonts for Latin Modern.}
%Latin Modern Math is a maths companion for the Latin Modern
%family of fonts, in OpenType format. For use with {\LuaLaTeX} or
%{\XeLaTeX}, support is available from the unicode-math package.
%\end{package}

\begin{package}{umvs}{marvosym}{Martin Vogel's Symbols (marvosym) font.}
Martin Vogel's Symbol font (marvosym) contains the Euro
currency symbol as defined by the European commission, along
with symbols for structural engineering; symbols for steel
cross-sections; astronomy signs (sun, moon, planets); the 12
signs of the zodiac; scissor symbols; CE sign and others. The
package contains both the original TrueType font and the
derived Type~1 font, together with support files for {\TeX}
({\LaTeX}).
\end{package}

%\begin{package}{}{mathpazo}{Fonts to typeset mathematics to match Palatino.}
%The Pazo Math fonts are a family of PostScript fonts suitable
%for typesetting mathematics in combination with the Palatino
%family of text fonts. The Pazo Math family is made up of five
%fonts provided in Adobe Type~1 format (PazoMath, PazoMath-
%Italic, PazoMath-Bold, PazoMath-BoldItalic, and
%PazoMathBlackboardBold). These contain, in designs that match
%Palatino, glyphs that are usually not available in Palatino and
%for which Computer Modern looks odd when combined with
%Palatino. These glyphs include the uppercase Greek alphabet in
%upright and slanted shapes in regular and bold weights, the
%lowercase Greek alphabet in slanted shape in regular and bold
%weights, several mathematical glyphs (partialdiff, summation,
%product, coproduct, emptyset, infinity, and proportional) in
%regular and bold weights, other glyphs (Euro and dotlessj) in
%upright and slanted shapes in regular and bold weights, and the
%uppercase letters commonly used to represent various number
%sets (C, I, N, Q, R, and Z) in blackboard bold. The set also
%includes a set of 'true' small-caps fonts, also suitable for
%use with Palatino (or one of its clones). {\LaTeX} macro support
%(using package mathpazo.sty) is provided in psnfss (a required
%part of any {\LaTeX} distribution).
%\end{package}

% \begin{package}{pncb8r}{ncntrsbk: CenturySchL-Bold}{URW ``Base 35'' font pack for {\LaTeX}.}
% A set of fonts for use as ``drop-in'' replacements for Adobe's
% basic set, comprising: - Century Schoolbook (substituting for
% Adobe's New Century Schoolbook); - Dingbats (substituting for
% Adobe's Zapf Dingbats); - Nimbus Mono L (substituting for
% Abobe's Courier); - Nimbus Roman No9 L (substituting for
% Adobe's Times); - Nimbus Sans L (substituting for Adobe's
% Helvetica); - Standard Symbols L (substituting for Adobe's
% Symbol); - URW Bookman; - URW Chancery L Medium Italic
% (substituting for Adobe's Zapf Chancery); - URW Gothic L Book
% (substituting for Adobe's Avant Garde); and - URW Palladio L
% (substituting for Adobe's Palatino).
% \end{package}
% \begin{package}{pncbi8r}{ncntrsbk: CenturySchL-BoldItal}{URW ``Base 35'' font pack for {\LaTeX}.}
% A set of fonts for use as ``drop-in'' replacements for Adobe's
% basic set, comprising: - Century Schoolbook (substituting for
% Adobe's New Century Schoolbook); - Dingbats (substituting for
% Adobe's Zapf Dingbats); - Nimbus Mono L (substituting for
% Abobe's Courier); - Nimbus Roman No9 L (substituting for
% Adobe's Times); - Nimbus Sans L (substituting for Adobe's
% Helvetica); - Standard Symbols L (substituting for Adobe's
% Symbol); - URW Bookman; - URW Chancery L Medium Italic
% (substituting for Adobe's Zapf Chancery); - URW Gothic L Book
% (substituting for Adobe's Avant Garde); and - URW Palladio L
% (substituting for Adobe's Palatino).
% \end{package}
% \begin{package}{pncbo8r}{ncntrsbk: CenturySchL-BoldObli}{URW ``Base 35'' font pack for {\LaTeX}.}
% A set of fonts for use as ``drop-in'' replacements for Adobe's
% basic set, comprising: - Century Schoolbook (substituting for
% Adobe's New Century Schoolbook); - Dingbats (substituting for
% Adobe's Zapf Dingbats); - Nimbus Mono L (substituting for
% Abobe's Courier); - Nimbus Roman No9 L (substituting for
% Adobe's Times); - Nimbus Sans L (substituting for Adobe's
% Helvetica); - Standard Symbols L (substituting for Adobe's
% Symbol); - URW Bookman; - URW Chancery L Medium Italic
% (substituting for Adobe's Zapf Chancery); - URW Gothic L Book
% (substituting for Adobe's Avant Garde); and - URW Palladio L
% (substituting for Adobe's Palatino).
% \end{package}
\begin{package}{pncr8r}{ncntrsbk: CenturySchL-Roma}{URW ``Base 35'' font pack for {\LaTeX}.}
A set of fonts for use as ``drop-in'' replacements for Adobe's
basic set, comprising: - Century Schoolbook (substituting for
Adobe's New Century Schoolbook); - Dingbats (substituting for
Adobe's Zapf Dingbats); - Nimbus Mono L (substituting for
Abobe's Courier); - Nimbus Roman No9 L (substituting for
Adobe's Times); - Nimbus Sans L (substituting for Adobe's
Helvetica); - Standard Symbols L (substituting for Adobe's
Symbol); - URW Bookman; - URW Chancery L Medium Italic
(substituting for Adobe's Zapf Chancery); - URW Gothic L Book
(substituting for Adobe's Avant Garde); and - URW Palladio L
(substituting for Adobe's Palatino).
\end{package}
% \begin{package}{pncri8r}{ncntrsbk: CenturySchL-Ital}{URW ``Base 35'' font pack for {\LaTeX}.}
% A set of fonts for use as ``drop-in'' replacements for Adobe's
% basic set, comprising: - Century Schoolbook (substituting for
% Adobe's New Century Schoolbook); - Dingbats (substituting for
% Adobe's Zapf Dingbats); - Nimbus Mono L (substituting for
% Abobe's Courier); - Nimbus Roman No9 L (substituting for
% Adobe's Times); - Nimbus Sans L (substituting for Adobe's
% Helvetica); - Standard Symbols L (substituting for Adobe's
% Symbol); - URW Bookman; - URW Chancery L Medium Italic
% (substituting for Adobe's Zapf Chancery); - URW Gothic L Book
% (substituting for Adobe's Avant Garde); and - URW Palladio L
% (substituting for Adobe's Palatino).
% \end{package}
% \begin{package}{pncro8r}{ncntrsbk: CenturySchL-Obli}{URW ``Base 35'' font pack for {\LaTeX}.}
% A set of fonts for use as ``drop-in'' replacements for Adobe's
% basic set, comprising: - Century Schoolbook (substituting for
% Adobe's New Century Schoolbook); - Dingbats (substituting for
% Adobe's Zapf Dingbats); - Nimbus Mono L (substituting for
% Abobe's Courier); - Nimbus Roman No9 L (substituting for
% Adobe's Times); - Nimbus Sans L (substituting for Adobe's
% Helvetica); - Standard Symbols L (substituting for Adobe's
% Symbol); - URW Bookman; - URW Chancery L Medium Italic
% (substituting for Adobe's Zapf Chancery); - URW Gothic L Book
% (substituting for Adobe's Avant Garde); and - URW Palladio L
% (substituting for Adobe's Palatino).
% \end{package}

% \begin{package}{pplb8r}{palatino: URWPalladioL-Bold}{URW ``Base 35'' font pack for {\LaTeX}.}
% A set of fonts for use as ``drop-in'' replacements for Adobe's
% basic set, comprising: - Century Schoolbook (substituting for
% Adobe's New Century Schoolbook); - Dingbats (substituting for
% Adobe's Zapf Dingbats); - Nimbus Mono L (substituting for
% Abobe's Courier); - Nimbus Roman No9 L (substituting for
% Adobe's Times); - Nimbus Sans L (substituting for Adobe's
% Helvetica); - Standard Symbols L (substituting for Adobe's
% Symbol); - URW Bookman; - URW Chancery L Medium Italic
% (substituting for Adobe's Zapf Chancery); - URW Gothic L Book
% (substituting for Adobe's Avant Garde); and - URW Palladio L
% (substituting for Adobe's Palatino).
% \end{package}
% \begin{package}{pplbi8r}{palatino: URWPalladioL-BoldItal}{URW ``Base 35'' font pack for {\LaTeX}.}
% A set of fonts for use as ``drop-in'' replacements for Adobe's
% basic set, comprising: - Century Schoolbook (substituting for
% Adobe's New Century Schoolbook); - Dingbats (substituting for
% Adobe's Zapf Dingbats); - Nimbus Mono L (substituting for
% Abobe's Courier); - Nimbus Roman No9 L (substituting for
% Adobe's Times); - Nimbus Sans L (substituting for Adobe's
% Helvetica); - Standard Symbols L (substituting for Adobe's
% Symbol); - URW Bookman; - URW Chancery L Medium Italic
% (substituting for Adobe's Zapf Chancery); - URW Gothic L Book
% (substituting for Adobe's Avant Garde); and - URW Palladio L
% (substituting for Adobe's Palatino).
% \end{package}
% \begin{package}{pplbo8r}{palatino: URWPalladioL-BoldObli}{URW ``Base 35'' font pack for {\LaTeX}.}
% A set of fonts for use as ``drop-in'' replacements for Adobe's
% basic set, comprising: - Century Schoolbook (substituting for
% Adobe's New Century Schoolbook); - Dingbats (substituting for
% Adobe's Zapf Dingbats); - Nimbus Mono L (substituting for
% Abobe's Courier); - Nimbus Roman No9 L (substituting for
% Adobe's Times); - Nimbus Sans L (substituting for Adobe's
% Helvetica); - Standard Symbols L (substituting for Adobe's
% Symbol); - URW Bookman; - URW Chancery L Medium Italic
% (substituting for Adobe's Zapf Chancery); - URW Gothic L Book
% (substituting for Adobe's Avant Garde); and - URW Palladio L
% (substituting for Adobe's Palatino).
% \end{package}
\begin{package}{pplr8r}{palatino: URWPalladioL-Roma}{URW ``Base 35'' font pack for {\LaTeX}.}
A set of fonts for use as ``drop-in'' replacements for Adobe's
basic set, comprising: - Century Schoolbook (substituting for
Adobe's New Century Schoolbook); - Dingbats (substituting for
Adobe's Zapf Dingbats); - Nimbus Mono L (substituting for
Abobe's Courier); - Nimbus Roman No9 L (substituting for
Adobe's Times); - Nimbus Sans L (substituting for Adobe's
Helvetica); - Standard Symbols L (substituting for Adobe's
Symbol); - URW Bookman; - URW Chancery L Medium Italic
(substituting for Adobe's Zapf Chancery); - URW Gothic L Book
(substituting for Adobe's Avant Garde); and - URW Palladio L
(substituting for Adobe's Palatino).
\end{package}
% \begin{package}{pplri8r}{palatino: URWPalladioL-Ital}{URW ``Base 35'' font pack for {\LaTeX}.}
% A set of fonts for use as ``drop-in'' replacements for Adobe's
% basic set, comprising: - Century Schoolbook (substituting for
% Adobe's New Century Schoolbook); - Dingbats (substituting for
% Adobe's Zapf Dingbats); - Nimbus Mono L (substituting for
% Abobe's Courier); - Nimbus Roman No9 L (substituting for
% Adobe's Times); - Nimbus Sans L (substituting for Adobe's
% Helvetica); - Standard Symbols L (substituting for Adobe's
% Symbol); - URW Bookman; - URW Chancery L Medium Italic
% (substituting for Adobe's Zapf Chancery); - URW Gothic L Book
% (substituting for Adobe's Avant Garde); and - URW Palladio L
% (substituting for Adobe's Palatino).
% \end{package}
% \begin{package}{pplro8r}{palatino: URWPalladioL-Obli}{URW ``Base 35'' font pack for {\LaTeX}.}
% A set of fonts for use as ``drop-in'' replacements for Adobe's
% basic set, comprising: - Century Schoolbook (substituting for
% Adobe's New Century Schoolbook); - Dingbats (substituting for
% Adobe's Zapf Dingbats); - Nimbus Mono L (substituting for
% Abobe's Courier); - Nimbus Roman No9 L (substituting for
% Adobe's Times); - Nimbus Sans L (substituting for Adobe's
% Helvetica); - Standard Symbols L (substituting for Adobe's
% Symbol); - URW Bookman; - URW Chancery L Medium Italic
% (substituting for Adobe's Zapf Chancery); - URW Gothic L Book
% (substituting for Adobe's Avant Garde); and - URW Palladio L
% (substituting for Adobe's Palatino).
% \end{package}


% \begin{package}{pxsy}{pxfonts}{Palatino-like fonts in support of mathematics.}
% Pxfonts supplies virtual text roman fonts using Adobe Palatino
% (or URWPalladioL) with some modified and additional text
% symbols in the OT1, T1, and TS1 encodings; maths alphabets
% using Palatino/Palladio; maths fonts providing all the symbols
% of the Computer Modern and AMS fonts, including all the Greek
% capital letters from CMR; and additional maths fonts of various
% other symbols. The set is complemented by a sans-serif set of
% text fonts, based on Helvetica/NimbusSanL, and a monospace set
% derived from the parallel TX font set. All the fonts are in
% Type~1 format (AFM and PFB files), and are supported by {\TeX}
% metrics (VF and TFM files) and macros for use with {\LaTeX}.
% \end{package}

% \begin{package}{rsfs5}{rsfs}{Ralph Smith's Formal Script font.}
% The fonts provide uppercase 'formal' script letters for use as
% symbols in scientific and mathematical typesetting (in contrast
% to the informal script fonts such as that used for the
% 'calligraphic' symbols in the {\TeX} maths symbol font). The fonts
% are provided as MetaFont source, and as derived Adobe Type~1
% format. {\LaTeX} support, for using these fonts in mathematics, is
% available via one of the packages calrsfs and mathrsfs.
% \end{package}

\begin{package}{psyr}{symbol: StandardSymL}{URW ``Base 35'' font pack for {\LaTeX}.}
A set of fonts for use as ``drop-in'' replacements for Adobe's
basic set, comprising: - Century Schoolbook (substituting for
Adobe's New Century Schoolbook); - Dingbats (substituting for
Adobe's Zapf Dingbats); - Nimbus Mono L (substituting for
Abobe's Courier); - Nimbus Roman No9 L (substituting for
Adobe's Times); - Nimbus Sans L (substituting for Adobe's
Helvetica); - Standard Symbols L (substituting for Adobe's
Symbol); - URW Bookman; - URW Chancery L Medium Italic
(substituting for Adobe's Zapf Chancery); - URW Gothic L Book
(substituting for Adobe's Avant Garde); and - URW Palladio L
(substituting for Adobe's Palatino).
\end{package}
% \begin{package}{psyro}{symbol: StandardSymL-Obli}{URW ``Base 35'' font pack for {\LaTeX}.}
% A set of fonts for use as ``drop-in'' replacements for Adobe's
% basic set, comprising: - Century Schoolbook (substituting for
% Adobe's New Century Schoolbook); - Dingbats (substituting for
% Adobe's Zapf Dingbats); - Nimbus Mono L (substituting for
% Abobe's Courier); - Nimbus Roman No9 L (substituting for
% Adobe's Times); - Nimbus Sans L (substituting for Adobe's
% Helvetica); - Standard Symbols L (substituting for Adobe's
% Symbol); - URW Bookman; - URW Chancery L Medium Italic
% (substituting for Adobe's Zapf Chancery); - URW Gothic L Book
% (substituting for Adobe's Avant Garde); and - URW Palladio L
% (substituting for Adobe's Palatino).
% \end{package}


\begin{package}{ec-qagb}{tex-gyre: TeXGyreAdventor-Bold}{{\TeX} Fonts extending freely available URW fonts.}
The {\TeX}-GYRE bundle consists of six font families: {\TeX} Gyre
Adventor is based on the URW Gothic L family of fonts (which is
derived from ITC Avant Garde Gothic, designed by Herb Lubalin
and Tom Carnase). {\TeX} Gyre Bonum is based on the URW Bookman L
family (from Bookman Old Style, designed by Alexander
Phemister). {\TeX} Gyre Chorus is based on URW Chancery L Medium
Italic (from ITC Zapf Chancery, designed by Hermann Zapf in
1979). {\TeX}-Gyre Cursor is based on URW Nimbus Mono L (based on
Courier, designed by Howard G. Kettler in 1955, for IBM). {\TeX}
Gyre Heros is based on URW Nimbus Sans L (from Helvetica,
prepared by Max Miedinger, with Eduard Hoffmann in 1957). {\TeX}
Gyre Pagella is based on URW Palladio L (from Palation,
designed by Hermann Zapf in the 1940s). {\TeX} Gyre Schola is
based on the URW Century Schoolbook L family (which was
designed by Morris Fuller Benton for the American Type
Founders). {\TeX} Gyre Termes % is based on the URW Nimbus Roman No9
% L family of fonts (whose original, Times, was designed by
% Stanley Morison together with Starling Burgess and Victor
% Lardent and first offered by Monotype). The constituent
% standard faces of each family have been greatly extended, and
% contain nearly 1200 glyphs each (though Chorus omits Greek
% support, has no small-caps family and has approximately 900
% glyphs). Each family is available in Adobe Type~1 and Open Type
% formats, and {\LaTeX} support (for use with a variety of
% encodings) is provided. Vietnamese and Cyrillic characters were
% added by Han The Thanh and Valek Filippov, respectively.
\end{package}

\begin{package}{ec-qagbi}{tex-gyre: TeXGyreAdventor-BoldItalic}{{\TeX} Fonts extending freely available URW fonts.}
The {\TeX}-GYRE bundle consists of six font families: {\TeX} Gyre
Adventor is based on the URW Gothic L family of fonts (which is
derived from ITC Avant Garde Gothic, designed by Herb Lubalin
and Tom Carnase). {\TeX} Gyre Bonum is based on the URW Bookman L
family (from Bookman Old Style, designed by Alexander
Phemister). {\TeX} Gyre Chorus is based on URW Chancery L Medium
Italic (from ITC Zapf Chancery, designed by Hermann Zapf in
1979). {\TeX}-Gyre Cursor is based on URW Nimbus Mono L (based on
Courier, designed by Howard G. Kettler in 1955, for IBM). {\TeX}
Gyre Heros is based on URW Nimbus Sans L (from Helvetica,
prepared by Max Miedinger, with Eduard Hoffmann in 1957). {\TeX}
Gyre Pagella is based on URW Palladio L (from Palation,
designed by Hermann Zapf in the 1940s). {\TeX} Gyre Schola is
based on the URW Century Schoolbook L family (which was
designed by Morris Fuller Benton for the American Type
Founders). {\TeX} Gyre Termes % is based on the URW Nimbus Roman No9
% L family of fonts (whose original, Times, was designed by
% Stanley Morison together with Starling Burgess and Victor
% Lardent and first offered by Monotype). The constituent
% standard faces of each family have been greatly extended, and
% contain nearly 1200 glyphs each (though Chorus omits Greek
% support, has no small-caps family and has approximately 900
% glyphs). Each family is available in Adobe Type~1 and Open Type
% formats, and {\LaTeX} support (for use with a variety of
% encodings) is provided. Vietnamese and Cyrillic characters were
% added by Han The Thanh and Valek Filippov, respectively.
\end{package}
\begin{package}{ec-qagr}{tex-gyre: TeXGyreAdventor-Regular}{{\TeX} Fonts extending freely available URW fonts.}
The {\TeX}-GYRE bundle consists of six font families: {\TeX} Gyre
Adventor is based on the URW Gothic L family of fonts (which is
derived from ITC Avant Garde Gothic, designed by Herb Lubalin
and Tom Carnase). {\TeX} Gyre Bonum is based on the URW Bookman L
family (from Bookman Old Style, designed by Alexander
Phemister). {\TeX} Gyre Chorus is based on URW Chancery L Medium
Italic (from ITC Zapf Chancery, designed by Hermann Zapf in
1979). {\TeX}-Gyre Cursor is based on URW Nimbus Mono L (based on
Courier, designed by Howard G. Kettler in 1955, for IBM). {\TeX}
Gyre Heros is based on URW Nimbus Sans L (from Helvetica,
prepared by Max Miedinger, with Eduard Hoffmann in 1957). {\TeX}
Gyre Pagella is based on URW Palladio L (from Palation,
designed by Hermann Zapf in the 1940s). {\TeX} Gyre Schola is
based on the URW Century Schoolbook L family (which was
designed by Morris Fuller Benton for the American Type
Founders). {\TeX} Gyre Termes % is based on the URW Nimbus Roman No9
% L family of fonts (whose original, Times, was designed by
% Stanley Morison together with Starling Burgess and Victor
% Lardent and first offered by Monotype). The constituent
% standard faces of each family have been greatly extended, and
% contain nearly 1200 glyphs each (though Chorus omits Greek
% support, has no small-caps family and has approximately 900
% glyphs). Each family is available in Adobe Type~1 and Open Type
% formats, and {\LaTeX} support (for use with a variety of
% encodings) is provided. Vietnamese and Cyrillic characters were
% added by Han The Thanh and Valek Filippov, respectively.
\end{package}
\begin{package}{ec-qagri}{tex-gyre: TeXGyreAdventor-Italic}{{\TeX} Fonts extending freely available URW fonts.}
The {\TeX}-GYRE bundle consists of six font families: {\TeX} Gyre
Adventor is based on the URW Gothic L family of fonts (which is
derived from ITC Avant Garde Gothic, designed by Herb Lubalin
and Tom Carnase). {\TeX} Gyre Bonum is based on the URW Bookman L
family (from Bookman Old Style, designed by Alexander
Phemister). {\TeX} Gyre Chorus is based on URW Chancery L Medium
Italic (from ITC Zapf Chancery, designed by Hermann Zapf in
1979). {\TeX}-Gyre Cursor is based on URW Nimbus Mono L (based on
Courier, designed by Howard G. Kettler in 1955, for IBM). {\TeX}
Gyre Heros is based on URW Nimbus Sans L (from Helvetica,
prepared by Max Miedinger, with Eduard Hoffmann in 1957). {\TeX}
Gyre Pagella is based on URW Palladio L (from Palation,
designed by Hermann Zapf in the 1940s). {\TeX} Gyre Schola is
based on the URW Century Schoolbook L family (which was
designed by Morris Fuller Benton for the American Type
Founders). {\TeX} Gyre Termes % is based on the URW Nimbus Roman No9
% L family of fonts (whose original, Times, was designed by
% Stanley Morison together with Starling Burgess and Victor
% Lardent and first offered by Monotype). The constituent
% standard faces of each family have been greatly extended, and
% contain nearly 1200 glyphs each (though Chorus omits Greek
% support, has no small-caps family and has approximately 900
% glyphs). Each family is available in Adobe Type~1 and Open Type
% formats, and {\LaTeX} support (for use with a variety of
% encodings) is provided. Vietnamese and Cyrillic characters were
% added by Han The Thanh and Valek Filippov, respectively.
\end{package}

\begin{package}{ec-qbkb}{tex-gyre: TeXGyreBonum-Bold}{{\TeX} Fonts extending freely available URW fonts.}
The {\TeX}-GYRE bundle consists of six font families: {\TeX} Gyre
Adventor is based on the URW Gothic L family of fonts (which is
derived from ITC Avant Garde Gothic, designed by Herb Lubalin
and Tom Carnase). {\TeX} Gyre Bonum is based on the URW Bookman L
family (from Bookman Old Style, designed by Alexander
Phemister). {\TeX} Gyre Chorus is based on URW Chancery L Medium
Italic (from ITC Zapf Chancery, designed by Hermann Zapf in
1979). {\TeX}-Gyre Cursor is based on URW Nimbus Mono L (based on
Courier, designed by Howard G. Kettler in 1955, for IBM). {\TeX}
Gyre Heros is based on URW Nimbus Sans L (from Helvetica,
prepared by Max Miedinger, with Eduard Hoffmann in 1957). {\TeX}
Gyre Pagella is based on URW Palladio L (from Palation,
designed by Hermann Zapf in the 1940s). {\TeX} Gyre Schola is
based on the URW Century Schoolbook L % family (which was
% designed by Morris Fuller Benton for the American Type
% Founders). {\TeX} Gyre Termes is based on the URW Nimbus Roman No9
% L family of fonts (whose original, Times, was designed by
% Stanley Morison together with Starling Burgess and Victor
% Lardent and first offered by Monotype). The constituent
% standard faces of each family have been greatly extended, and
% contain nearly 1200 glyphs each (though Chorus omits Greek
% support, has no small-caps family and has approximately 900
% glyphs). Each family is available in Adobe Type~1 and Open Type
% formats, and {\LaTeX} support (for use with a variety of
% encodings) is provided. Vietnamese and Cyrillic characters were
% added by Han The Thanh and Valek Filippov, respectively.
\end{package}
\begin{package}{ec-qbkbi}{tex-gyre: TeXGyreBonum-BoldItalic}{{\TeX} Fonts extending freely available URW fonts.}
The {\TeX}-GYRE bundle consists of six font families: {\TeX} Gyre
Adventor is based on the URW Gothic L family of fonts (which is
derived from ITC Avant Garde Gothic, designed by Herb Lubalin
and Tom Carnase). {\TeX} Gyre Bonum is based on the URW Bookman L
family (from Bookman Old Style, designed by Alexander
Phemister). {\TeX} Gyre Chorus is based on URW Chancery L Medium
Italic (from ITC Zapf Chancery, designed by Hermann Zapf in
1979). {\TeX}-Gyre Cursor is based on URW Nimbus Mono L (based on
Courier, designed by Howard G. Kettler in 1955, for IBM). {\TeX}
Gyre Heros is based on URW Nimbus Sans L (from Helvetica,
prepared by Max Miedinger, with Eduard Hoffmann in 1957). {\TeX}
Gyre Pagella is based on URW Palladio L (from Palation,
designed by Hermann Zapf in the 1940s). {\TeX} Gyre Schola is
based on the URW Century % Schoolbook L family (which was
% designed by Morris Fuller Benton for the American Type
% Founders). {\TeX} Gyre Termes is based on the URW Nimbus Roman No9
% L family of fonts (whose original, Times, was designed by
% Stanley Morison together with Starling Burgess and Victor
% Lardent and first offered by Monotype). The constituent
% standard faces of each family have been greatly extended, and
% contain nearly 1200 glyphs each (though Chorus omits Greek
% support, has no small-caps family and has approximately 900
% glyphs). Each family is available in Adobe Type~1 and Open Type
% formats, and {\LaTeX} support (for use with a variety of
% encodings) is provided. Vietnamese and Cyrillic characters were
% added by Han The Thanh and Valek Filippov, respectively.
\end{package}
\begin{package}{ec-qbkr}{tex-gyre: TeXGyreBonum-Regular}{{\TeX} Fonts extending freely available URW fonts.}
The {\TeX}-GYRE bundle consists of six font families: {\TeX} Gyre
Adventor is based on the URW Gothic L family of fonts (which is
derived from ITC Avant Garde Gothic, designed by Herb Lubalin
and Tom Carnase). {\TeX} Gyre Bonum is based on the URW Bookman L
family (from Bookman Old Style, designed by Alexander
Phemister). {\TeX} Gyre Chorus is based on URW Chancery L Medium
Italic (from ITC Zapf Chancery, designed by Hermann Zapf in
1979). {\TeX}-Gyre Cursor is based on URW Nimbus Mono L (based on
Courier, designed by Howard G. Kettler in 1955, for IBM). {\TeX}
Gyre Heros is based on URW Nimbus Sans L (from Helvetica,
prepared by Max Miedinger, with Eduard Hoffmann in 1957). {\TeX}
Gyre Pagella is based on URW Palladio L (from Palation,
designed by Hermann Zapf in the 1940s). {\TeX} Gyre Schola is
based on the URW Century Schoolbook L family (which was
designed by Morris Fuller Benton for the American % Type
% Founders). {\TeX} Gyre Termes is based on the URW Nimbus Roman No9
% L family of fonts (whose original, Times, was designed by
% Stanley Morison together with Starling Burgess and Victor
% Lardent and first offered by Monotype). The constituent
% standard faces of each family have been greatly extended, and
% contain nearly 1200 glyphs each (though Chorus omits Greek
% support, has no small-caps family and has approximately 900
% glyphs). Each family is available in Adobe Type~1 and Open Type
% formats, and {\LaTeX} support (for use with a variety of
% encodings) is provided. Vietnamese and Cyrillic characters were
% added by Han The Thanh and Valek Filippov, respectively.
\end{package}
\begin{package}{ec-qbkri}{tex-gyre: TeXGyreBonum-Italic}{{\TeX} Fonts extending freely available URW fonts.}
The {\TeX}-GYRE bundle consists of six font families: {\TeX} Gyre
Adventor is based on the URW Gothic L family of fonts (which is
derived from ITC Avant Garde Gothic, designed by Herb Lubalin
and Tom Carnase). {\TeX} Gyre Bonum is based on the URW Bookman L
family (from Bookman Old Style, designed by Alexander
Phemister). {\TeX} Gyre Chorus is based on URW Chancery L Medium
Italic (from ITC Zapf Chancery, designed by Hermann Zapf in
1979). {\TeX}-Gyre Cursor is based on URW Nimbus Mono L (based on
Courier, designed by Howard G. Kettler in 1955, for IBM). {\TeX}
Gyre Heros is based on URW Nimbus Sans L (from Helvetica,
prepared by Max Miedinger, with Eduard Hoffmann in 1957). {\TeX}
Gyre Pagella is based on URW Palladio L (from Palation,
designed by Hermann Zapf in the 1940s). {\TeX} Gyre Schola is
based on the URW Century Schoolbook L family (which was
designed by Morris Fuller Benton for the American Type
Founders). % {\TeX} Gyre Termes is based on the URW Nimbus Roman No9
% L family of fonts (whose original, Times, was designed by
% Stanley Morison together with Starling Burgess and Victor
% Lardent and first offered by Monotype). The constituent
% standard faces of each family have been greatly extended, and
% contain nearly 1200 glyphs each (though Chorus omits Greek
% support, has no small-caps family and has approximately 900
% glyphs). Each family is available in Adobe Type~1 and Open Type
% formats, and {\LaTeX} support (for use with a variety of
% encodings) is provided. Vietnamese and Cyrillic characters were
% added by Han The Thanh and Valek Filippov, respectively.
\end{package}

\begin{package}{ec-qcrb}{tex-gyre: TeXGyreCursor-Bold}{{\TeX} Fonts extending freely available URW fonts.}
The {\TeX}-GYRE bundle consists of six font families: {\TeX} Gyre
Adventor is based on the URW Gothic L family of fonts (which is
derived from ITC Avant Garde Gothic, designed by Herb Lubalin
and Tom Carnase). {\TeX} Gyre Bonum is based on the URW Bookman L
family (from Bookman Old Style, designed by Alexander
Phemister). {\TeX} Gyre Chorus is based on URW Chancery L Medium
Italic (from ITC Zapf Chancery, designed by Hermann Zapf in
1979). {\TeX}-Gyre Cursor is based on URW Nimbus Mono L (based on
Courier, designed by Howard G. Kettler in 1955, for IBM). {\TeX}
Gyre Heros is based on URW Nimbus Sans L (from Helvetica,
prepared by Max Miedinger, with Eduard Hoffmann in 1957). {\TeX}
Gyre Pagella is based on URW Palladio L (from Palation,
designed % by Hermann Zapf in the 1940s). {\TeX} Gyre Schola is
% based on the URW Century Schoolbook L family (which was
% designed by Morris Fuller Benton for the American Type
% Founders). {\TeX} Gyre Termes is based on the URW Nimbus Roman No9
% L family of fonts (whose original, Times, was designed by
% Stanley Morison together with Starling Burgess and Victor
% Lardent and first offered by Monotype). The constituent
% standard faces of each family have been greatly extended, and
% contain nearly 1200 glyphs each (though Chorus omits Greek
% support, has no small-caps family and has approximately 900
% glyphs). Each family is available in Adobe Type~1 and Open Type
% formats, and {\LaTeX} support (for use with a variety of
% encodings) is provided. Vietnamese and Cyrillic characters were
% added by Han The Thanh and Valek Filippov, respectively.
\end{package}
\begin{package}{ec-qcrbi}{tex-gyre: TeXGyreCursor-BoldItalic}{{\TeX} Fonts extending freely available URW fonts.}
The {\TeX}-GYRE bundle consists of six font families: {\TeX} Gyre
Adventor is based on the URW Gothic L family of fonts (which is
derived from ITC Avant Garde Gothic, designed by Herb Lubalin
and Tom Carnase). {\TeX} Gyre Bonum is based on the URW Bookman L
family (from Bookman Old Style, designed by Alexander
Phemister). {\TeX} Gyre Chorus is based on URW Chancery L Medium
Italic (from ITC Zapf Chancery, designed by Hermann Zapf in
1979). {\TeX}-Gyre Cursor is based on URW Nimbus Mono L (based on
Courier, designed by Howard G. Kettler in 1955, for IBM). {\TeX}
Gyre Heros is based on URW Nimbus Sans L (from Helvetica,
prepared by Max Miedinger, with Eduard Hoffmann in 1957). {\TeX}
Gyre Pagella is based on URW Palladio L (from Palation,
designed % by Hermann Zapf in the 1940s). {\TeX} Gyre Schola is
% based on the URW Century Schoolbook L family (which was
% designed by Morris Fuller Benton for the American Type
% Founders). {\TeX} Gyre Termes is based on the URW Nimbus Roman No9
% L family of fonts (whose original, Times, was designed by
% Stanley Morison together with Starling Burgess and Victor
% Lardent and first offered by Monotype). The constituent
% standard faces of each family have been greatly extended, and
% contain nearly 1200 glyphs each (though Chorus omits Greek
% support, has no small-caps family and has approximately 900
% glyphs). Each family is available in Adobe Type~1 and Open Type
% formats, and {\LaTeX} support (for use with a variety of
% encodings) is provided. Vietnamese and Cyrillic characters were
% added by Han The Thanh and Valek Filippov, respectively.
\end{package}
\begin{package}{ec-qcrr}{tex-gyre: TeXGyreCursor-Regular}{{\TeX} Fonts extending freely available URW fonts.}
The {\TeX}-GYRE bundle consists of six font families: {\TeX} Gyre
Adventor is based on the URW Gothic L family of fonts (which is
derived from ITC Avant Garde Gothic, designed by Herb Lubalin
and Tom Carnase). {\TeX} Gyre Bonum is based on the URW Bookman L
family (from Bookman Old Style, designed by Alexander
Phemister). {\TeX} Gyre Chorus is based on URW Chancery L Medium
Italic (from ITC Zapf Chancery, designed by Hermann Zapf in
1979). {\TeX}-Gyre Cursor is based on URW Nimbus Mono L (based on
Courier, designed by Howard G. Kettler in 1955, for IBM). {\TeX}
Gyre Heros is based on URW Nimbus Sans L (from Helvetica,
prepared by Max Miedinger, with Eduard Hoffmann in 1957). {\TeX}
Gyre Pagella is based on URW Palladio L (from Palation,
designed by Hermann Zapf in the 1940s). {\TeX} Gyre % Schola is
% based on the URW Century Schoolbook L family (which was
% designed by Morris Fuller Benton for the American Type
% Founders). {\TeX} Gyre Termes is based on the URW Nimbus Roman No9
% L family of fonts (whose original, Times, was designed by
% Stanley Morison together with Starling Burgess and Victor
% Lardent and first offered by Monotype). The constituent
% standard faces of each family have been greatly extended, and
% contain nearly 1200 glyphs each (though Chorus omits Greek
% support, has no small-caps family and has approximately 900
% glyphs). Each family is available in Adobe Type~1 and Open Type
% formats, and {\LaTeX} support (for use with a variety of
% encodings) is provided. Vietnamese and Cyrillic characters were
% added by Han The Thanh and Valek Filippov, respectively.
\end{package}
\begin{package}{ec-qcrri}{tex-gyre: TeXGyreCursor-Italic}{{\TeX} Fonts extending freely available URW fonts.}
The {\TeX}-GYRE bundle consists of six font families: {\TeX} Gyre
Adventor is based on the URW Gothic L family of fonts (which is
derived from ITC Avant Garde Gothic, designed by Herb Lubalin
and Tom Carnase). {\TeX} Gyre Bonum is based on the URW Bookman L
family (from Bookman Old Style, designed by Alexander
Phemister). {\TeX} Gyre Chorus is based on URW Chancery L Medium
Italic (from ITC Zapf Chancery, designed by Hermann Zapf in
1979). {\TeX}-Gyre Cursor is based on URW Nimbus Mono L (based on
Courier, designed by Howard G. Kettler in 1955, for IBM). {\TeX}
Gyre Heros is based on URW Nimbus Sans L (from Helvetica,
prepared by Max Miedinger, with Eduard Hoffmann in 1957). {\TeX}
Gyre Pagella is based on URW Palladio L (from Palation,
designed by Hermann Zapf in the 1940s). {\TeX} Gyre % Schola is
% based on the URW Century Schoolbook L family (which was
% designed by Morris Fuller Benton for the American Type
% Founders). {\TeX} Gyre Termes is based on the URW Nimbus Roman No9
% L family of fonts (whose original, Times, was designed by
% Stanley Morison together with Starling Burgess and Victor
% Lardent and first offered by Monotype). The constituent
% standard faces of each family have been greatly extended, and
% contain nearly 1200 glyphs each (though Chorus omits Greek
% support, has no small-caps family and has approximately 900
% glyphs). Each family is available in Adobe Type~1 and Open Type
% formats, and {\LaTeX} support (for use with a variety of
% encodings) is provided. Vietnamese and Cyrillic characters were
% added by Han The Thanh and Valek Filippov, respectively.
\end{package}
\begin{package}{ec-qcsb}{tex-gyre: TeXGyreSchola-Bold}{{\TeX} Fonts extending freely available URW fonts.}
The {\TeX}-GYRE bundle consists of six font families: {\TeX} Gyre
Adventor is based on the URW Gothic L family of fonts (which is
derived from ITC Avant Garde Gothic, designed by Herb Lubalin
and Tom Carnase). {\TeX} Gyre Bonum is based on the URW Bookman L
family (from Bookman Old Style, designed by Alexander
Phemister). {\TeX} Gyre Chorus is based on URW Chancery L Medium
Italic (from ITC Zapf Chancery, designed by Hermann Zapf in
1979). {\TeX}-Gyre Cursor is based on URW Nimbus Mono L (based on
Courier, designed by Howard G. Kettler in 1955, for IBM). {\TeX}
Gyre Heros is based on URW Nimbus Sans L (from Helvetica,
prepared by Max Miedinger, with Eduard Hoffmann in 1957). {\TeX}
Gyre Pagella is based on URW Palladio L (from Palation,
designed by Hermann Zapf in the 1940s). {\TeX} Gyre Schola is
based on the URW Century % Schoolbook L family (which was
% designed by Morris Fuller Benton for the American Type
% Founders). {\TeX} Gyre Termes is based on the URW Nimbus Roman No9
% L family of fonts (whose original, Times, was designed by
% Stanley Morison together with Starling Burgess and Victor
% Lardent and first offered by Monotype). The constituent
% standard faces of each family have been greatly extended, and
% contain nearly 1200 glyphs each (though Chorus omits Greek
% support, has no small-caps family and has approximately 900
% glyphs). Each family is available in Adobe Type~1 and Open Type
% formats, and {\LaTeX} support (for use with a variety of
% encodings) is provided. Vietnamese and Cyrillic characters were
% added by Han The Thanh and Valek Filippov, respectively.
\end{package}
\begin{package}{ec-qcsbi}{tex-gyre: TeXGyreSchola-BoldItalic}{{\TeX} Fonts extending freely available URW fonts.}
The {\TeX}-GYRE bundle consists of six font families: {\TeX} Gyre
Adventor is based on the URW Gothic L family of fonts (which is
derived from ITC Avant Garde Gothic, designed by Herb Lubalin
and Tom Carnase). {\TeX} Gyre Bonum is based on the URW Bookman L
family (from Bookman Old Style, designed by Alexander
Phemister). {\TeX} Gyre Chorus is based on URW Chancery L Medium
Italic (from ITC Zapf Chancery, designed by Hermann Zapf in
1979). {\TeX}-Gyre Cursor is based on URW Nimbus Mono L (based on
Courier, designed by Howard G. Kettler in 1955, for IBM). {\TeX}
Gyre Heros is based on URW Nimbus Sans L (from Helvetica,
prepared by Max Miedinger, with Eduard Hoffmann in 1957). {\TeX}
Gyre Pagella is based on URW Palladio L (from Palation,
designed by Hermann Zapf in the 1940s). {\TeX} Gyre Schola is
based on the URW Century Schoolbook L family (which % was
% designed by Morris Fuller Benton for the American Type
% Founders). {\TeX} Gyre Termes is based on the URW Nimbus Roman No9
% L family of fonts (whose original, Times, was designed by
% Stanley Morison together with Starling Burgess and Victor
% Lardent and first offered by Monotype). The constituent
% standard faces of each family have been greatly extended, and
% contain nearly 1200 glyphs each (though Chorus omits Greek
% support, has no small-caps family and has approximately 900
% glyphs). Each family is available in Adobe Type~1 and Open Type
% formats, and {\LaTeX} support (for use with a variety of
% encodings) is provided. Vietnamese and Cyrillic characters were
% added by Han The Thanh and Valek Filippov, respectively.
\end{package}
\begin{package}{ec-qcsr}{tex-gyre: TeXGyreSchola-Regular}{{\TeX} Fonts extending freely available URW fonts.}
The {\TeX}-GYRE bundle consists of six font families: {\TeX} Gyre
Adventor is based on the URW Gothic L family of fonts (which is
derived from ITC Avant Garde Gothic, designed by Herb Lubalin
and Tom Carnase). {\TeX} Gyre Bonum is based on the URW Bookman L
family (from Bookman Old Style, designed by Alexander
Phemister). {\TeX} Gyre Chorus is based on URW Chancery L Medium
Italic (from ITC Zapf Chancery, designed by Hermann Zapf in
1979). {\TeX}-Gyre Cursor is based on URW Nimbus Mono L (based on
Courier, designed by Howard G. Kettler in 1955, for IBM). {\TeX}
Gyre Heros is based on URW Nimbus Sans L (from Helvetica,
prepared by Max Miedinger, with Eduard Hoffmann in 1957). {\TeX}
Gyre Pagella is based on URW Palladio L (from Palation,
designed by Hermann Zapf in the 1940s). {\TeX} Gyre Schola is
based on the URW Century Schoolbook L family (which was
designed by Morris Fuller Benton for the American Type
Founders). {\TeX} Gyre % Termes is based on the URW Nimbus Roman No9
% L family of fonts (whose original, Times, was designed by
% Stanley Morison together with Starling Burgess and Victor
% Lardent and first offered by Monotype). The constituent
% standard faces of each family have been greatly extended, and
% contain nearly 1200 glyphs each (though Chorus omits Greek
% support, has no small-caps family and has approximately 900
% glyphs). Each family is available in Adobe Type~1 and Open Type
% formats, and {\LaTeX} support (for use with a variety of
% encodings) is provided. Vietnamese and Cyrillic characters were
% added by Han The Thanh and Valek Filippov, respectively.
\end{package}
\begin{package}{ec-qcsri}{tex-gyre: TeXGyreSchola-Italic}{{\TeX} Fonts extending freely available URW fonts.}
The {\TeX}-GYRE bundle consists of six font families: {\TeX} Gyre
Adventor is based on the URW Gothic L family of fonts (which is
derived from ITC Avant Garde Gothic, designed by Herb Lubalin
and Tom Carnase). {\TeX} Gyre Bonum is based on the URW Bookman L
family (from Bookman Old Style, designed by Alexander
Phemister). {\TeX} Gyre Chorus is based on URW Chancery L Medium
Italic (from ITC Zapf Chancery, designed by Hermann Zapf in
1979). {\TeX}-Gyre Cursor is based on URW Nimbus Mono L (based on
Courier, designed by Howard G. Kettler in 1955, for IBM). {\TeX}
Gyre Heros is based on URW Nimbus Sans L (from Helvetica,
prepared by Max Miedinger, with Eduard Hoffmann in 1957). {\TeX}
Gyre Pagella is based on URW Palladio L (from Palation,
designed by Hermann Zapf in the 1940s). {\TeX} Gyre Schola is
based on the URW Century Schoolbook L family (which was
designed by Morris Fuller Benton for the American Type
Founders). {\TeX} Gyre Termes is based % on the URW Nimbus Roman No9
% L family of fonts (whose original, Times, was designed by
% Stanley Morison together with Starling Burgess and Victor
% Lardent and first offered by Monotype). The constituent
% standard faces of each family have been greatly extended, and
% contain nearly 1200 glyphs each (though Chorus omits Greek
% support, has no small-caps family and has approximately 900
% glyphs). Each family is available in Adobe Type~1 and Open Type
% formats, and {\LaTeX} support (for use with a variety of
% encodings) is provided. Vietnamese and Cyrillic characters were
% added by Han The Thanh and Valek Filippov, respectively.
\end{package}

\begin{package}{ec-qhvb}{tex-gyre: TeXGyreHeros-Bold}{{\TeX} Fonts extending freely available URW fonts.}
The {\TeX}-GYRE bundle consists of six font families: {\TeX} Gyre
Adventor is based on the URW Gothic L family of fonts (which is
derived from ITC Avant Garde Gothic, designed by Herb Lubalin
and Tom Carnase). {\TeX} Gyre Bonum is based on the URW Bookman L
family (from Bookman Old Style, designed by Alexander
Phemister). {\TeX} Gyre Chorus is based on URW Chancery L Medium
Italic (from ITC Zapf Chancery, designed by Hermann Zapf in
1979). {\TeX}-Gyre Cursor is based on URW Nimbus Mono L (based on
Courier, designed by Howard G. Kettler in 1955, for IBM). {\TeX}
Gyre Heros is based on URW Nimbus Sans L (from Helvetica,
prepared by Max Miedinger, with Eduard Hoffmann in 1957). {\TeX}
Gyre Pagella is based on URW Palladio L (from Palation,
designed by Hermann Zapf in the 1940s). {\TeX} Gyre Schola is
based on the URW Century Schoolbook L family (which was
designed by Morris Fuller Benton for the American Type
Founders). {\TeX} Gyre % Termes is based on the URW Nimbus Roman No9
% L family of fonts (whose original, Times, was designed by
% Stanley Morison together with Starling Burgess and Victor
% Lardent and first offered by Monotype). The constituent
% standard faces of each family have been greatly extended, and
% contain nearly 1200 glyphs each (though Chorus omits Greek
% support, has no small-caps family and has approximately 900
% glyphs). Each family is available in Adobe Type~1 and Open Type
% formats, and {\LaTeX} support (for use with a variety of
% encodings) is provided. Vietnamese and Cyrillic characters were
% added by Han The Thanh and Valek Filippov, respectively.
\end{package}
\begin{package}{ec-qhvbi}{tex-gyre: TeXGyreHeros-BoldItalic}{{\TeX} Fonts extending freely available URW fonts.}
The {\TeX}-GYRE bundle consists of six font families: {\TeX} Gyre
Adventor is based on the URW Gothic L family of fonts (which is
derived from ITC Avant Garde Gothic, designed by Herb Lubalin
and Tom Carnase). {\TeX} Gyre Bonum is based on the URW Bookman L
family (from Bookman Old Style, designed by Alexander
Phemister). {\TeX} Gyre Chorus is based on URW Chancery L Medium
Italic (from ITC Zapf Chancery, designed by Hermann Zapf in
1979). {\TeX}-Gyre Cursor is based on URW Nimbus Mono L (based on
Courier, designed by Howard G. Kettler in 1955, for IBM). {\TeX}
Gyre Heros is based on URW Nimbus Sans L (from Helvetica,
prepared by Max Miedinger, with Eduard Hoffmann in 1957). {\TeX}
Gyre Pagella is based on URW Palladio L (from Palation,
designed by Hermann Zapf in the 1940s). {\TeX} Gyre Schola is
based on the URW Century Schoolbook L family (which was
designed by Morris Fuller Benton for the American Type
Founders). {\TeX} Gyre % Termes is based on the URW Nimbus Roman No9
% L family of fonts (whose original, Times, was designed by
% Stanley Morison together with Starling Burgess and Victor
% Lardent and first offered by Monotype). The constituent
% standard faces of each family have been greatly extended, and
% contain nearly 1200 glyphs each (though Chorus omits Greek
% support, has no small-caps family and has approximately 900
% glyphs). Each family is available in Adobe Type~1 and Open Type
% formats, and {\LaTeX} support (for use with a variety of
% encodings) is provided. Vietnamese and Cyrillic characters were
% added by Han The Thanh and Valek Filippov, respectively.
\end{package}
\begin{package}{ec-qhvcb}{tex-gyre: TeXGyreHerosCondensed-Bold}{{\TeX} Fonts extending freely available URW fonts.}
The {\TeX}-GYRE bundle consists of six font families: {\TeX} Gyre
Adventor is based on the URW Gothic L family of fonts (which is
derived from ITC Avant Garde Gothic, designed by Herb Lubalin
and Tom Carnase). {\TeX} Gyre Bonum is based on the URW Bookman L
family (from Bookman Old Style, designed by Alexander
Phemister). {\TeX} Gyre Chorus is based on URW Chancery L Medium
Italic (from ITC Zapf Chancery, designed by Hermann Zapf in
1979). {\TeX}-Gyre Cursor is based on URW Nimbus Mono L (based on
Courier, designed by Howard G. Kettler in 1955, for IBM). {\TeX}
Gyre Heros is based on URW Nimbus Sans L (from Helvetica,
prepared by Max Miedinger, with Eduard Hoffmann in 1957). {\TeX}
Gyre Pagella is based on URW Palladio L (from Palation,
designed by Hermann Zapf in the 1940s). {\TeX} Gyre Schola is
based on the URW Century Schoolbook L family (which was
designed by Morris Fuller Benton for the American Type
Founders). {\TeX} Gyre Termes is based on the URW Nimbus Roman No9
L family of fonts (whose original, Times, was designed by
Stanley Morison together with Starling Burgess and Victor
Lardent and first offered by Monotype). The % constituent
% standard faces of each family have been greatly extended, and
% contain nearly 1200 glyphs each (though Chorus omits Greek
% support, has no small-caps family and has approximately 900
% glyphs). Each family is available in Adobe Type~1 and Open Type
% formats, and {\LaTeX} support (for use with a variety of
% encodings) is provided. Vietnamese and Cyrillic characters were
% added by Han The Thanh and Valek Filippov, respectively.
\end{package}
\begin{package}{ec-qhvcbi}{tex-gyre: TeXGyreHerosCondensed-BoldItalic}{{\TeX} Fonts extending freely available URW fonts.}
The {\TeX}-GYRE bundle consists of six font families: {\TeX} Gyre
Adventor is based on the URW Gothic L family of fonts (which is
derived from ITC Avant Garde Gothic, designed by Herb Lubalin
and Tom Carnase). {\TeX} Gyre Bonum is based on the URW Bookman L
family (from Bookman Old Style, designed by Alexander
Phemister). {\TeX} Gyre Chorus is based on URW Chancery L Medium
Italic (from ITC Zapf Chancery, designed by Hermann Zapf in
1979). {\TeX}-Gyre Cursor is based on URW Nimbus Mono L (based on
Courier, designed by Howard G. Kettler in 1955, for IBM). {\TeX}
Gyre Heros is based on URW Nimbus Sans L (from Helvetica,
prepared by Max Miedinger, with Eduard Hoffmann in 1957). {\TeX}
Gyre Pagella is based on URW Palladio L (from Palation,
designed by Hermann Zapf in the 1940s). {\TeX} Gyre Schola is
based on the URW Century Schoolbook L family (which was
designed by Morris Fuller Benton for the American Type
Founders). {\TeX} Gyre Termes is based on the URW Nimbus Roman No9
L family of fonts (whose original, Times, was designed by
Stanley Morison together with Starling Burgess and Victor
Lardent and first offered by Monotype). The constituent
% standard faces of each family have been greatly extended, and
% contain nearly 1200 glyphs each (though Chorus omits Greek
% support, has no small-caps family and has approximately 900
% glyphs). Each family is available in Adobe Type~1 and Open Type
% formats, and {\LaTeX} support (for use with a variety of
% encodings) is provided. Vietnamese and Cyrillic characters were
% added by Han The Thanh and Valek Filippov, respectively.
\end{package}
\begin{package}{ec-qhvcr}{tex-gyre: TeXGyreHerosCondensed-Regular}{{\TeX} Fonts extending freely available URW fonts.}
The {\TeX}-GYRE bundle consists of six font families: {\TeX} Gyre
Adventor is based on the URW Gothic L family of fonts (which is
derived from ITC Avant Garde Gothic, designed by Herb Lubalin
and Tom Carnase). {\TeX} Gyre Bonum is based on the URW Bookman L
family (from Bookman Old Style, designed by Alexander
Phemister). {\TeX} Gyre Chorus is based on URW Chancery L Medium
Italic (from ITC Zapf Chancery, designed by Hermann Zapf in
1979). {\TeX}-Gyre Cursor is based on URW Nimbus Mono L (based on
Courier, designed by Howard G. Kettler in 1955, for IBM). {\TeX}
Gyre Heros is based on URW Nimbus Sans L (from Helvetica,
prepared by Max Miedinger, with Eduard Hoffmann in 1957). {\TeX}
Gyre Pagella is based on URW Palladio L (from Palation,
designed by Hermann Zapf in the 1940s). {\TeX} Gyre Schola is
based on the URW Century Schoolbook L family (which was
designed by Morris Fuller Benton for the American Type
Founders). {\TeX} Gyre Termes is based on the URW Nimbus Roman No9
L family of fonts (whose original, Times, was designed by
Stanley Morison together with Starling Burgess and Victor
Lardent and first offered by Monotype). The constituent
standard faces of each family have been greatly extended, and
contain nearly 1200 glyphs % each (though Chorus omits Greek
% support, has no small-caps family and has approximately 900
% glyphs). Each family is available in Adobe Type~1 and Open Type
% formats, and {\LaTeX} support (for use with a variety of
% encodings) is provided. Vietnamese and Cyrillic characters were
% added by Han The Thanh and Valek Filippov, respectively.
\end{package}
\begin{package}{ec-qhvcri}{tex-gyre: TeXGyreHerosCondensed-Italic}{{\TeX} Fonts extending freely available URW fonts.}
The {\TeX}-GYRE bundle consists of six font families: {\TeX} Gyre
Adventor is based on the URW Gothic L family of fonts (which is
derived from ITC Avant Garde Gothic, designed by Herb Lubalin
and Tom Carnase). {\TeX} Gyre Bonum is based on the URW Bookman L
family (from Bookman Old Style, designed by Alexander
Phemister). {\TeX} Gyre Chorus is based on URW Chancery L Medium
Italic (from ITC Zapf Chancery, designed by Hermann Zapf in
1979). {\TeX}-Gyre Cursor is based on URW Nimbus Mono L (based on
Courier, designed by Howard G. Kettler in 1955, for IBM). {\TeX}
Gyre Heros is based on URW Nimbus Sans L (from Helvetica,
prepared by Max Miedinger, with Eduard Hoffmann in 1957). {\TeX}
Gyre Pagella is based on URW Palladio L (from Palation,
designed by Hermann Zapf in the 1940s). {\TeX} Gyre Schola is
based on the URW Century Schoolbook L family (which was
designed by Morris Fuller Benton for the American Type
Founders). {\TeX} Gyre Termes is based on the URW Nimbus Roman No9
L family of fonts (whose original, Times, was designed by
Stanley Morison together with Starling Burgess and Victor
Lardent and first offered by Monotype). The constituent
standard faces of each family have been greatly extended, and
contain nearly 1200 glyphs % each (though Chorus omits Greek
% support, has no small-caps family and has approximately 900
% glyphs). Each family is available in Adobe Type~1 and Open Type
% formats, and {\LaTeX} support (for use with a variety of
% encodings) is provided. Vietnamese and Cyrillic characters were
% added by Han The Thanh and Valek Filippov, respectively.
\end{package}
\begin{package}{ec-qhvr}{tex-gyre: TeXGyreHeros-Regular}{{\TeX} Fonts extending freely available URW fonts.}
The {\TeX}-GYRE bundle consists of six font families: {\TeX} Gyre
Adventor is based on the URW Gothic L family of fonts (which is
derived from ITC Avant Garde Gothic, designed by Herb Lubalin
and Tom Carnase). {\TeX} Gyre Bonum is based on the URW Bookman L
family (from Bookman Old Style, designed by Alexander
Phemister). {\TeX} Gyre Chorus is based on URW Chancery L Medium
Italic (from ITC Zapf Chancery, designed by Hermann Zapf in
1979). {\TeX}-Gyre Cursor is based on URW Nimbus Mono L (based on
Courier, designed by Howard G. Kettler in 1955, for IBM). {\TeX}
Gyre Heros is based on URW Nimbus Sans L (from Helvetica,
prepared by Max Miedinger, with Eduard Hoffmann in 1957). {\TeX}
Gyre Pagella is based on URW Palladio L (from Palation,
designed by Hermann Zapf in the 1940s). {\TeX} Gyre Schola is
based on the URW Century Schoolbook L family (which was
designed by Morris Fuller Benton for the American Type
Founders). {\TeX} Gyre Termes is based on the URW Nimbus Roman No9
L family of fonts % (whose original, Times, was designed by
% Stanley Morison together with Starling Burgess and Victor
% Lardent and first offered by Monotype). The constituent
% standard faces of each family have been greatly extended, and
% contain nearly 1200 glyphs each (though Chorus omits Greek
% support, has no small-caps family and has approximately 900
% glyphs). Each family is available in Adobe Type~1 and Open Type
% formats, and {\LaTeX} support (for use with a variety of
% encodings) is provided. Vietnamese and Cyrillic characters were
% added by Han The Thanh and Valek Filippov, respectively.
\end{package}
\begin{package}{ec-qhvri}{tex-gyre: TeXGyreHeros-Italic}{{\TeX} Fonts extending freely available URW fonts.}
The {\TeX}-GYRE bundle consists of six font families: {\TeX} Gyre
Adventor is based on the URW Gothic L family of fonts (which is
derived from ITC Avant Garde Gothic, designed by Herb Lubalin
and Tom Carnase). {\TeX} Gyre Bonum is based on the URW Bookman L
family (from Bookman Old Style, designed by Alexander
Phemister). {\TeX} Gyre Chorus is based on URW Chancery L Medium
Italic (from ITC Zapf Chancery, designed by Hermann Zapf in
1979). {\TeX}-Gyre Cursor is based on URW Nimbus Mono L (based on
Courier, designed by Howard G. Kettler in 1955, for IBM). {\TeX}
Gyre Heros is based on URW Nimbus Sans L (from Helvetica,
prepared by Max Miedinger, with Eduard Hoffmann in 1957). {\TeX}
Gyre Pagella is based on URW Palladio L (from Palation,
designed by Hermann Zapf in the 1940s). {\TeX} Gyre Schola is
based on the URW Century Schoolbook L family (which was
designed by Morris Fuller Benton for the American Type
Founders). {\TeX} Gyre Termes is based on the URW Nimbus Roman No9
L family of fonts % (whose original, Times, was designed by
% Stanley Morison together with Starling Burgess and Victor
% Lardent and first offered by Monotype). The constituent
% standard faces of each family have been greatly extended, and
% contain nearly 1200 glyphs each (though Chorus omits Greek
% support, has no small-caps family and has approximately 900
% glyphs). Each family is available in Adobe Type~1 and Open Type
% formats, and {\LaTeX} support (for use with a variety of
% encodings) is provided. Vietnamese and Cyrillic characters were
% added by Han The Thanh and Valek Filippov, respectively.
\end{package}

\begin{package}{ec-qplb}{tex-gyre: TeXGyrePagella-Bold}{{\TeX} Fonts extending freely available URW fonts.}
The {\TeX}-GYRE bundle consists of six font families: {\TeX} Gyre
Adventor is based on the URW Gothic L family of fonts (which is
derived from ITC Avant Garde Gothic, designed by Herb Lubalin
and Tom Carnase). {\TeX} Gyre Bonum is based on the URW Bookman L
family (from Bookman Old Style, designed by Alexander
Phemister). {\TeX} Gyre Chorus is based on URW Chancery L Medium
Italic (from ITC Zapf Chancery, designed by Hermann Zapf in
1979). {\TeX}-Gyre Cursor is based on URW Nimbus Mono L (based on
Courier, designed by Howard G. Kettler in 1955, for IBM). {\TeX}
Gyre Heros is based on URW Nimbus Sans L (from Helvetica,
prepared by Max Miedinger, with Eduard Hoffmann in 1957). {\TeX}
Gyre Pagella is based on URW Palladio L (from Palation,
designed by Hermann Zapf in the 1940s). {\TeX} Gyre Schola is
based on the URW Century Schoolbook L family (which was
designed by Morris Fuller Benton for the American Type
Founders). {\TeX} Gyre % Termes is based on the URW Nimbus Roman No9
% L family of fonts (whose original, Times, was designed by
% Stanley Morison together with Starling Burgess and Victor
% Lardent and first offered by Monotype). The constituent
% standard faces of each family have been greatly extended, and
% contain nearly 1200 glyphs each (though Chorus omits Greek
% support, has no small-caps family and has approximately 900
% glyphs). Each family is available in Adobe Type~1 and Open Type
% formats, and {\LaTeX} support (for use with a variety of
% encodings) is provided. Vietnamese and Cyrillic characters were
% added by Han The Thanh and Valek Filippov, respectively.
\end{package}
\begin{package}{ec-qplbi}{tex-gyre: TeXGyrePagella-BoldItalic}{{\TeX} Fonts extending freely available URW fonts.}
The {\TeX}-GYRE bundle consists of six font families: {\TeX} Gyre
Adventor is based on the URW Gothic L family of fonts (which is
derived from ITC Avant Garde Gothic, designed by Herb Lubalin
and Tom Carnase). {\TeX} Gyre Bonum is based on the URW Bookman L
family (from Bookman Old Style, designed by Alexander
Phemister). {\TeX} Gyre Chorus is based on URW Chancery L Medium
Italic (from ITC Zapf Chancery, designed by Hermann Zapf in
1979). {\TeX}-Gyre Cursor is based on URW Nimbus Mono L (based on
Courier, designed by Howard G. Kettler in 1955, for IBM). {\TeX}
Gyre Heros is based on URW Nimbus Sans L (from Helvetica,
prepared by Max Miedinger, with Eduard Hoffmann in 1957). {\TeX}
Gyre Pagella is based on URW Palladio L (from Palation,
designed by Hermann Zapf in the 1940s). {\TeX} Gyre Schola is
based on the URW Century Schoolbook L family (which was
designed by Morris Fuller Benton for the American Type
Founders). {\TeX} Gyre Termes is based on the URW Nimbus Roman No9
L family of % fonts (whose original, Times, was designed by
% Stanley Morison together with Starling Burgess and Victor
% Lardent and first offered by Monotype). The constituent
% standard faces of each family have been greatly extended, and
% contain nearly 1200 glyphs each (though Chorus omits Greek
% support, has no small-caps family and has approximately 900
% glyphs). Each family is available in Adobe Type~1 and Open Type
% formats, and {\LaTeX} support (for use with a variety of
% encodings) is provided. Vietnamese and Cyrillic characters were
% added by Han The Thanh and Valek Filippov, respectively.
\end{package}
\begin{package}{ec-qplr}{tex-gyre: TeXGyrePagella-Regular}{{\TeX} Fonts extending freely available URW fonts.}
The {\TeX}-GYRE bundle consists of six font families: {\TeX} Gyre
Adventor is based on the URW Gothic L family of fonts (which is
derived from ITC Avant Garde Gothic, designed by Herb Lubalin
and Tom Carnase). {\TeX} Gyre Bonum is based on the URW Bookman L
family (from Bookman Old Style, designed by Alexander
Phemister). {\TeX} Gyre Chorus is based on URW Chancery L Medium
Italic (from ITC Zapf Chancery, designed by Hermann Zapf in
1979). {\TeX}-Gyre Cursor is based on URW Nimbus Mono L (based on
Courier, designed by Howard G. Kettler in 1955, for IBM). {\TeX}
Gyre Heros is based on URW Nimbus Sans L (from Helvetica,
prepared by Max Miedinger, with Eduard Hoffmann in 1957). {\TeX}
Gyre Pagella is based on URW Palladio L (from Palation,
designed by Hermann Zapf in the 1940s). {\TeX} Gyre Schola is
based on the URW Century Schoolbook L family (which was
designed by Morris Fuller Benton for the American Type
Founders). {\TeX} Gyre Termes is based on the URW Nimbus Roman No9
L family of % fonts (whose original, Times, was designed by
% Stanley Morison together with Starling Burgess and Victor
% Lardent and first offered by Monotype). The constituent
% standard faces of each family have been greatly extended, and
% contain nearly 1200 glyphs each (though Chorus omits Greek
% support, has no small-caps family and has approximately 900
% glyphs). Each family is available in Adobe Type~1 and Open Type
% formats, and {\LaTeX} support (for use with a variety of
% encodings) is provided. Vietnamese and Cyrillic characters were
% added by Han The Thanh and Valek Filippov, respectively.
\end{package}
\begin{package}{ec-qplri}{tex-gyre: TeXGyrePagella-Italic}{{\TeX} Fonts extending freely available URW fonts.}
The {\TeX}-GYRE bundle consists of six font families: {\TeX} Gyre
Adventor is based on the URW Gothic L family of fonts (which is
derived from ITC Avant Garde Gothic, designed by Herb Lubalin
and Tom Carnase). {\TeX} Gyre Bonum is based on the URW Bookman L
family (from Bookman Old Style, designed by Alexander
Phemister). {\TeX} Gyre Chorus is based on URW Chancery L Medium
Italic (from ITC Zapf Chancery, designed by Hermann Zapf in
1979). {\TeX}-Gyre Cursor is based on URW Nimbus Mono L (based on
Courier, designed by Howard G. Kettler in 1955, for IBM). {\TeX}
Gyre Heros is based on URW Nimbus Sans L (from Helvetica,
prepared by Max Miedinger, with Eduard Hoffmann in 1957). {\TeX}
Gyre Pagella is based on URW Palladio L (from Palation,
designed by Hermann Zapf in the 1940s). {\TeX} Gyre Schola is
based on the URW Century Schoolbook L family (which was
designed by Morris Fuller Benton for the American Type
Founders). {\TeX} Gyre Termes is based on the URW Nimbus Roman No9
L family of fonts (whose original, Times, was designed by
Stanley Morison together with Starling Burgess and Victor
Lardent and first offered by Monotype). The % constituent
% standard faces of each family have been greatly extended, and
% contain nearly 1200 glyphs each (though Chorus omits Greek
% support, has no small-caps family and has approximately 900
% glyphs). Each family is available in Adobe Type~1 and Open Type
% formats, and {\LaTeX} support (for use with a variety of
% encodings) is provided. Vietnamese and Cyrillic characters were
% added by Han The Thanh and Valek Filippov, respectively.
\end{package}


\begin{package}{ec-qtmb}{tex-gyre: TeXGyreTermes-Bold}{{\TeX} Fonts extending freely available URW fonts.}
The {\TeX}-GYRE bundle consists of six font families: {\TeX} Gyre
Adventor is based on the URW Gothic L family of fonts (which is
derived from ITC Avant Garde Gothic, designed by Herb Lubalin
and Tom Carnase). {\TeX} Gyre Bonum is based on the URW Bookman L
family (from Bookman Old Style, designed by Alexander
Phemister). {\TeX} Gyre Chorus is based on URW Chancery L Medium
Italic (from ITC Zapf Chancery, designed by Hermann Zapf in
1979). {\TeX}-Gyre Cursor is based on URW Nimbus Mono L (based on
Courier, designed by Howard G. Kettler in 1955, for IBM). {\TeX}
Gyre Heros is based on URW Nimbus Sans L (from Helvetica,
prepared by Max Miedinger, with Eduard Hoffmann in 1957). {\TeX}
Gyre Pagella is based on URW Palladio L (from Palation,
designed by Hermann Zapf in the 1940s). {\TeX} Gyre Schola is
based on the URW Century Schoolbook L family (which was
designed by Morris Fuller Benton for the American Type
Founders). {\TeX} Gyre Termes is based on the URW Nimbus Roman No9
L family of fonts (whose original, % Times, was designed by
% Stanley Morison together with Starling Burgess and Victor
% Lardent and first offered by Monotype). The constituent
% standard faces of each family have been greatly extended, and
% contain nearly 1200 glyphs each (though Chorus omits Greek
% support, has no small-caps family and has approximately 900
% glyphs). Each family is available in Adobe Type~1 and Open Type
% formats, and {\LaTeX} support (for use with a variety of
% encodings) is provided. Vietnamese and Cyrillic characters were
% added by Han The Thanh and Valek Filippov, respectively.
\end{package}
\begin{package}{ec-qtmbi}{tex-gyre: TeXGyreTermes-BoldItalic}{{\TeX} Fonts extending freely available URW fonts.}
The {\TeX}-GYRE bundle consists of six font families: {\TeX} Gyre
Adventor is based on the URW Gothic L family of fonts (which is
derived from ITC Avant Garde Gothic, designed by Herb Lubalin
and Tom Carnase). {\TeX} Gyre Bonum is based on the URW Bookman L
family (from Bookman Old Style, designed by Alexander
Phemister). {\TeX} Gyre Chorus is based on URW Chancery L Medium
Italic (from ITC Zapf Chancery, designed by Hermann Zapf in
1979). {\TeX}-Gyre Cursor is based on URW Nimbus Mono L (based on
Courier, designed by Howard G. Kettler in 1955, for IBM). {\TeX}
Gyre Heros is based on URW Nimbus Sans L (from Helvetica,
prepared by Max Miedinger, with Eduard Hoffmann in 1957). {\TeX}
Gyre Pagella is based on URW Palladio L (from Palation,
designed by Hermann Zapf in the 1940s). {\TeX} Gyre Schola is
based on the URW Century Schoolbook L family (which was
designed by Morris Fuller Benton for the American Type
Founders). {\TeX} Gyre Termes is based on the URW Nimbus Roman No9
L family of fonts (whose original, Times, was designed by
Stanley Morison together % with Starling Burgess and Victor
% Lardent and first offered by Monotype). The constituent
% standard faces of each family have been greatly extended, and
% contain nearly 1200 glyphs each (though Chorus omits Greek
% support, has no small-caps family and has approximately 900
% glyphs). Each family is available in Adobe Type~1 and Open Type
% formats, and {\LaTeX} support (for use with a variety of
% encodings) is provided. Vietnamese and Cyrillic characters were
% added by Han The Thanh and Valek Filippov, respectively.
\end{package}
\begin{package}{ec-qtmr}{tex-gyre: TeXGyreTermes-Regular}{{\TeX} Fonts extending freely available URW fonts.}
The {\TeX}-GYRE bundle consists of six font families: {\TeX} Gyre
Adventor is based on the URW Gothic L family of fonts (which is
derived from ITC Avant Garde Gothic, designed by Herb Lubalin
and Tom Carnase). {\TeX} Gyre Bonum is based on the URW Bookman L
family (from Bookman Old Style, designed by Alexander
Phemister). {\TeX} Gyre Chorus is based on URW Chancery L Medium
Italic (from ITC Zapf Chancery, designed by Hermann Zapf in
1979). {\TeX}-Gyre Cursor is based on URW Nimbus Mono L (based on
Courier, designed by Howard G. Kettler in 1955, for IBM). {\TeX}
Gyre Heros is based on URW Nimbus Sans L (from Helvetica,
prepared by Max Miedinger, with Eduard Hoffmann in 1957). {\TeX}
Gyre Pagella is based on URW Palladio L (from Palation,
designed by Hermann Zapf in the 1940s). {\TeX} Gyre Schola is
based on the URW Century Schoolbook L family (which was
designed by Morris Fuller Benton for the American Type
Founders). {\TeX} Gyre Termes is based on the URW Nimbus Roman No9
L family of fonts (whose original, Times, was designed by
Stanley Morison together with Starling Burgess % and Victor
% Lardent and first offered by Monotype). The constituent
% standard faces of each family have been greatly extended, and
% contain nearly 1200 glyphs each (though Chorus omits Greek
% support, has no small-caps family and has approximately 900
% glyphs). Each family is available in Adobe Type~1 and Open Type
% formats, and {\LaTeX} support (for use with a variety of
% encodings) is provided. Vietnamese and Cyrillic characters were
% added by Han The Thanh and Valek Filippov, respectively.
\end{package}
\begin{package}{ec-qtmri}{tex-gyre: TeXGyreTermes-Italic}{{\TeX} Fonts extending freely available URW fonts.}
The {\TeX}-GYRE bundle consists of six font families: {\TeX} Gyre
Adventor is based on the URW Gothic L family of fonts (which is
derived from ITC Avant Garde Gothic, designed by Herb Lubalin
and Tom Carnase). {\TeX} Gyre Bonum is based on the URW Bookman L
family (from Bookman Old Style, designed by Alexander
Phemister). {\TeX} Gyre Chorus is based on URW Chancery L Medium
Italic (from ITC Zapf Chancery, designed by Hermann Zapf in
1979). {\TeX}-Gyre Cursor is based on URW Nimbus Mono L (based on
Courier, designed by Howard G. Kettler in 1955, for IBM). {\TeX}
Gyre Heros is based on URW Nimbus Sans L (from Helvetica,
prepared by Max Miedinger, with Eduard Hoffmann in 1957). {\TeX}
Gyre Pagella is based on URW Palladio L (from Palation,
designed by Hermann Zapf in the 1940s). {\TeX} Gyre Schola is
based on the URW Century Schoolbook L family (which was
designed by Morris Fuller Benton for the American Type
Founders). {\TeX} Gyre Termes is based on the URW Nimbus Roman No9
L family of fonts (whose original, Times, was designed by
Stanley Morison together with Starling Burgess and Victor
Lardent and first offered by Monotype). The constituent
standard faces of each family have been greatly extended, and
contain % nearly 1200 glyphs each (though Chorus omits Greek
% support, has no small-caps family and has approximately 900
% glyphs). Each family is available in Adobe Type~1 and Open Type
% formats, and {\LaTeX} support (for use with a variety of
% encodings) is provided. Vietnamese and Cyrillic characters were
% added by Han The Thanh and Valek Filippov, respectively.
\end{package}

\begin{package}{ec-qzcmi}{tex-gyre: TeXGyreChorus-MediumItalic}{{\TeX} Fonts extending freely available URW fonts.}
The {\TeX}-GYRE bundle consists of six font families: {\TeX} Gyre
Adventor is based on the URW Gothic L family of fonts (which is
derived from ITC Avant Garde Gothic, designed by Herb Lubalin
and Tom Carnase). {\TeX} Gyre Bonum is based on the URW Bookman L
family (from Bookman Old Style, designed by Alexander
Phemister). {\TeX} Gyre Chorus is based on URW Chancery L Medium
Italic (from ITC Zapf Chancery, designed by Hermann Zapf in
1979). {\TeX}-Gyre Cursor is based on URW Nimbus Mono L (based on
Courier, designed by Howard G. Kettler in 1955, for IBM). {\TeX}
Gyre Heros is based on URW Nimbus Sans L (from Helvetica,
prepared by Max Miedinger, with Eduard Hoffmann in 1957). {\TeX}
Gyre Pagella is based on URW Palladio L (from Palation,
designed by Hermann Zapf in the 1940s). {\TeX} Gyre Schola is
based on the URW Century Schoolbook L family (which was
designed by Morris Fuller Benton for the American Type
Founders). {\TeX} Gyre Termes is based on the URW Nimbus Roman No9
L family of fonts (whose original, Times, was designed by
Stanley Morison together with Starling Burgess and Victor
Lardent and first offered by Monotype). The constituent
standard faces of each family have been greatly extended, and
contain nearly 1200 glyphs % each (though Chorus omits Greek
% support, has no small-caps family and has approximately 900
% glyphs). Each family is available in Adobe Type~1 and Open Type
% formats, and {\LaTeX} support (for use with a variety of
% encodings) is provided. Vietnamese and Cyrillic characters were
% added by Han The Thanh and Valek Filippov, respectively.
\end{package}

%\begin{package}{}{tex-gyre-math}{Maths fonts to match tex-gyre text fonts.}
%TeX-Gyre-Math is to be a collection of maths fonts to match the
%text fonts of the {\TeX}-Gyre collection. The collection will be
%made available in OpenType format, only; fonts will conform to
%the developing standards for OpenType maths fonts. {\TeX}-Gyre-
%Math-Pagella (to match Tex-Gyre-Pagella) and {\TeX}-Gyre-Math-
%Termes (to match Tex-Gyre-Termes) fonts are provided.
%\end{package}

% \begin{package}{ptmb8r}{times: NimbusRomNo9L-Medi}{URW ``Base 35'' font pack for {\LaTeX}.}
% A set of fonts for use as ``drop-in'' replacements for Adobe's
% basic set, comprising: - Century Schoolbook (substituting for
% Adobe's New Century Schoolbook); - Dingbats (substituting for
% Adobe's Zapf Dingbats); - Nimbus Mono L (substituting for
% Abobe's Courier); - Nimbus Roman No9 L (substituting for
% Adobe's Times); - Nimbus Sans L (substituting for Adobe's
% Helvetica); - Standard Symbols L (substituting for Adobe's
% Symbol); - URW Bookman; - URW Chancery L Medium Italic
% (substituting for Adobe's Zapf Chancery); - URW Gothic L Book
% (substituting for Adobe's Avant Garde); and - URW Palladio L
% (substituting for Adobe's Palatino).
% \end{package}
% \begin{package}{ptmbi8r}{times: NimbusRomNo9L-MediItal}{URW ``Base 35'' font pack for {\LaTeX}.}
% A set of fonts for use as ``drop-in'' replacements for Adobe's
% basic set, comprising: - Century Schoolbook (substituting for
% Adobe's New Century Schoolbook); - Dingbats (substituting for
% Adobe's Zapf Dingbats); - Nimbus Mono L (substituting for
% Abobe's Courier); - Nimbus Roman No9 L (substituting for
% Adobe's Times); - Nimbus Sans L (substituting for Adobe's
% Helvetica); - Standard Symbols L (substituting for Adobe's
% Symbol); - URW Bookman; - URW Chancery L Medium Italic
% (substituting for Adobe's Zapf Chancery); - URW Gothic L Book
% (substituting for Adobe's Avant Garde); and - URW Palladio L
% (substituting for Adobe's Palatino).
% \end{package}
% \begin{package}{ptmbo8r}{times: NimbusRomNo9L-MediObli}{URW ``Base 35'' font pack for {\LaTeX}.}
% A set of fonts for use as ``drop-in'' replacements for Adobe's
% basic set, comprising: - Century Schoolbook (substituting for
% Adobe's New Century Schoolbook); - Dingbats (substituting for
% Adobe's Zapf Dingbats); - Nimbus Mono L (substituting for
% Abobe's Courier); - Nimbus Roman No9 L (substituting for
% Adobe's Times); - Nimbus Sans L (substituting for Adobe's
% Helvetica); - Standard Symbols L (substituting for Adobe's
% Symbol); - URW Bookman; - URW Chancery L Medium Italic
% (substituting for Adobe's Zapf Chancery); - URW Gothic L Book
% (substituting for Adobe's Avant Garde); and - URW Palladio L
% (substituting for Adobe's Palatino).
% \end{package}

\begin{package}{ptmr8r}{times: NimbusRomNo9L-Regu}{URW ``Base 35'' font pack for {\LaTeX}.}
A set of fonts for use as ``drop-in'' replacements for Adobe's
basic set, comprising: - Century Schoolbook (substituting for
Adobe's New Century Schoolbook); - Dingbats (substituting for
Adobe's Zapf Dingbats); - Nimbus Mono L (substituting for
Abobe's Courier); - Nimbus Roman No9 L (substituting for
Adobe's Times); - Nimbus Sans L (substituting for Adobe's
Helvetica); - Standard Symbols L (substituting for Adobe's
Symbol); - URW Bookman; - URW Chancery L Medium Italic
(substituting for Adobe's Zapf Chancery); - URW Gothic L Book
(substituting for Adobe's Avant Garde); and - URW Palladio L
(substituting for Adobe's Palatino).
\end{package}
% \begin{package}{ptmri8r}{times: NimbusRomNo9L-ReguItal}{URW ``Base 35'' font pack for {\LaTeX}.}
% A set of fonts for use as ``drop-in'' replacements for Adobe's
% basic set, comprising: - Century Schoolbook (substituting for
% Adobe's New Century Schoolbook); - Dingbats (substituting for
% Adobe's Zapf Dingbats); - Nimbus Mono L (substituting for
% Abobe's Courier); - Nimbus Roman No9 L (substituting for
% Adobe's Times); - Nimbus Sans L (substituting for Adobe's
% Helvetica); - Standard Symbols L (substituting for Adobe's
% Symbol); - URW Bookman; - URW Chancery L Medium Italic
% (substituting for Adobe's Zapf Chancery); - URW Gothic L Book
% (substituting for Adobe's Avant Garde); and - URW Palladio L
% (substituting for Adobe's Palatino).
% \end{package}
% \begin{package}{ptmro8r}{times: NimbusRomNo9L-ReguObli}{URW ``Base 35'' font pack for {\LaTeX}.}
% A set of fonts for use as ``drop-in'' replacements for Adobe's
% basic set, comprising: - Century Schoolbook (substituting for
% Adobe's New Century Schoolbook); - Dingbats (substituting for
% Adobe's Zapf Dingbats); - Nimbus Mono L (substituting for
% Abobe's Courier); - Nimbus Roman No9 L (substituting for
% Adobe's Times); - Nimbus Sans L (substituting for Adobe's
% Helvetica); - Standard Symbols L (substituting for Adobe's
% Symbol); - URW Bookman; - URW Chancery L Medium Italic
% (substituting for Adobe's Zapf Chancery); - URW Gothic L Book
% (substituting for Adobe's Avant Garde); and - URW Palladio L
% (substituting for Adobe's Palatino).
% \end{package}

\begin{package}{tipass10}{tipa}{Fonts and macros for IPA phonetics characters.}
These fonts are considered the 'ultimate answer' to IPA
typesetting. The encoding of these 8-bit fonts has been
registered as {\LaTeX} standard encoding T3, and the set of
addendum symbols as encoding TS3. 'Times-like' Adobe Type~1
versions are provided for both the T3 and the TS3 fonts.
\end{package}
% \begin{package}{txtt}{txfonts}{Times-like fonts in support of mathematics.}
% Txfonts supplies virtual text roman fonts using Adobe Times (or
% URW NimbusRomNo9L) with some modified and additional text
% symbols in the OT1, T1, and TS1 encodings; maths alphabets
% using Times/URW Nimbus; maths fonts providing all the symbols
% of the Computer Modern and AMS fonts, including all the Greek
% capital letters from CMR; and additional maths fonts of various
% other symbols. The set is complemented by a sans-serif set of
% text fonts, based on Helvetica/NimbusSanL, and a monospace set.
% All the fonts are in Type~1 format (AFM and PFB files), and are
% supported by {\TeX} metrics (VF and TFM files) and macros for use
% with {\LaTeX}.
% \end{package}

\begin{package}{putb8r}{utopia}{Adobe Utopia fonts.}
The Adobe Standard Encoding set (upright and italic shapes,
medium and bold weights) of the Utopia font family, which Adobe
donated to the X Consortium. Macro support, and maths fonts
that match the Utopia family, are provided by the Fourier and
the Mathdesign Utopia font packages.
\end{package}
% \begin{package}{putbi8r}{utopia}{Adobe Utopia fonts.}
% The Adobe Standard Encoding set (upright and italic shapes,
% medium and bold weights) of the Utopia font family, which Adobe
% donated to the X Consortium. Macro support, and maths fonts
% that match the Utopia family, are provided by the Fourier and
% the Mathdesign Utopia font packages.
% \end{package}
% \begin{package}{putbo8r}{utopia}{Adobe Utopia fonts.}
% The Adobe Standard Encoding set (upright and italic shapes,
% medium and bold weights) of the Utopia font family, which Adobe
% donated to the X Consortium. Macro support, and maths fonts
% that match the Utopia family, are provided by the Fourier and
% the Mathdesign Utopia font packages.
% \end{package}
\begin{package}{putr8r}{utopia}{Adobe Utopia fonts.}
The Adobe Standard Encoding set (upright and italic shapes,
medium and bold weights) of the Utopia font family, which Adobe
donated to the X Consortium. Macro support, and maths fonts
that match the Utopia family, are provided by the Fourier and
the Mathdesign Utopia font packages.
\end{package}
\begin{package}{putri8r}{utopia}{Adobe Utopia fonts.}
The Adobe Standard Encoding set (upright and italic shapes,
medium and bold weights) of the Utopia font family, which Adobe
donated to the X Consortium. Macro support, and maths fonts
that match the Utopia family, are provided by the Fourier and
the Mathdesign Utopia font packages.
\end{package}
% \begin{package}{putro8r}{utopia}{Adobe Utopia fonts.}
% The Adobe Standard Encoding set (upright and italic shapes,
% medium and bold weights) of the Utopia font family, which Adobe
% donated to the X Consortium. Macro support, and maths fonts
% that match the Utopia family, are provided by the Fourier and
% the Mathdesign Utopia font packages.
% \end{package}

% \begin{package}{wasy10}{wasy}{The wasy fonts (Waldi symbol fonts).}
% These are the wasy (Waldi symbol) fonts, second release. This
% bundle presents the fonts in MetaFont format, but they are also
% available in Adobe Type~1 format. Support under {\LaTeX} is
% provided by the wasysym package.
% \end{package}
% \begin{package}{}{wasysym}{{\LaTeX} support file to use the WASY2 fonts}
% The WASY2 (Waldi Symbol) font by Roland Waldi provides many
% glyphs like male and female symbols and astronomical symbols,
% as well as the complete lasy font set and other odds and ends.
% The wasysym package implements an easy to use interface for
% these symbols.
% \end{package}

% \begin{package}{pzcmi8r}{zapfchan: URWChanceryL-MediItal}{URW ``Base 35'' font pack for {\LaTeX}.}
% A set of fonts for use as ``drop-in'' replacements for Adobe's
% basic set, comprising: - Century Schoolbook (substituting for
% Adobe's New Century Schoolbook); - Dingbats (substituting for
% Adobe's Zapf Dingbats); - Nimbus Mono L (substituting for
% Abobe's Courier); - Nimbus Roman No9 L (substituting for
% Adobe's Times); - Nimbus Sans L (substituting for Adobe's
% Helvetica); - Standard Symbols L (substituting for Adobe's
% Symbol); - URW Bookman; - URW Chancery L Medium Italic
% (substituting for Adobe's Zapf Chancery); - URW Gothic L Book
% (substituting for Adobe's Avant Garde); and - URW Palladio L
% (substituting for Adobe's Palatino).
% \end{package}

\begin{package}{uzdr}{zapfding: Dingbats}{URW ``Base 35'' font pack for {\LaTeX}.}
A set of fonts for use as ``drop-in'' replacements for Adobe's
basic set, comprising: - Century Schoolbook (substituting for
Adobe's New Century Schoolbook); - Dingbats (substituting for
Adobe's Zapf Dingbats); - Nimbus Mono L (substituting for
Abobe's Courier); - Nimbus Roman No9 L (substituting for
Adobe's Times); - Nimbus Sans L (substituting for Adobe's
Helvetica); - Standard Symbols L (substituting for Adobe's
Symbol); - URW Bookman; - URW Chancery L Medium Italic
(substituting for Adobe's Zapf Chancery); - URW Gothic L Book
(substituting for Adobe's Avant Garde); and - URW Palladio L
(substituting for Adobe's Palatino).
\end{package}

% \begin{package}{}{Asana-Math}{A font to typeset maths in Xe(La){\TeX} and Lua(La){\TeX}.}
% The Asana-Math font is an OpenType font that includes almost
% all mathematical Unicode symbols and it can be used to typeset
% mathematical text with any software that can understand the
% MATH OpenType table (e.g., {\XeTeX} 0.997 and Microsoft Word
% 2007). The font is beta software. Typesetting support for use
% with {\LaTeX} is provided by the fontspec and unicode-math
% packages.
% \end{package}

\begin{package}{AccanthisADFStdNo3-Bold-lf-t1--base}{accanthis}{Accanthis fonts, with {\LaTeX} support.}
Accanthis No.3 is designed by Hirwin Harendal and is suitable
as an alternative to fonts such as Garamond, Galliard, Horley
old style, Sabon, and Bembo. The support files are suitable for
use with all {\LaTeX} engines.
\end{package}
% \begin{package}{AccanthisADFStdNo3-BoldItalic-lf-t1--base}{accanthis}{Accanthis fonts, with {\LaTeX} support.}
% Accanthis No.3 is designed by Hirwin Harendal and is suitable
% as an alternative to fonts such as Garamond, Galliard, Horley
% old style, Sabon, and Bembo. The support files are suitable for
% use with all {\LaTeX} engines.
% \end{package}
\begin{package}{AccanthisADFStdNo3-Italic-lf-t1--base}{accanthis}{Accanthis fonts, with {\LaTeX} support.}
Accanthis No.3 is designed by Hirwin Harendal and is suitable
as an alternative to fonts such as Garamond, Galliard, Horley
old style, Sabon, and Bembo. The support files are suitable for
use with all {\LaTeX} engines.
\end{package}
\begin{package}{AccanthisADFStdNo3-Regular-lf-t1--base}{accanthis}{Accanthis fonts, with {\LaTeX} support.}
Accanthis No.3 is designed by Hirwin Harendal and is suitable
as an alternative to fonts such as Garamond, Galliard, Horley
old style, Sabon, and Bembo. The support files are suitable for
use with all {\LaTeX} engines.
\end{package}

% \begin{package}{OrnementsADF}{adforn}{OrnementsADF font with {\TeX}/{\LaTeX} support}
% The bundle provides the Ornements ADF font in PostScript type 1
% format with {\TeX}/{\LaTeX} support files. The font is licensed under
% GPL v2 or later with font exception. (See NOTICE, COPYING,
% README.) The {\TeX}/{\LaTeX} support is licensed under LPPL. (See
% README, manifest.txt.)
% \end{package}
% \begin{package}{ArrowsADF}{adfsymbols}{SymbolsADF with {\TeX}/{\LaTeX} support.}
% The package provides Arkandis foundry's ArrowsADF and
% BulletsADF fonts in Adobe Type~1 format, together with
% TeX/{\LaTeX} support files. The fonts are licensed under GPL v2 or
% later with font exception. (See NOTICE, COPYING, README.) The
% TeX/{\LaTeX} support is licensed under LPPL. (See README,
% manifest.txt.)
% \end{package}
% \begin{package}{BulletsADF}{adfsymbols}{SymbolsADF with {\TeX}/{\LaTeX} support.}
% The package provides Arkandis foundry's ArrowsADF and
% BulletsADF fonts in Adobe Type~1 format, together with
% TeX/{\LaTeX} support files. The fonts are licensed under GPL v2 or
% later with font exception. (See NOTICE, COPYING, README.) The
% TeX/{\LaTeX} support is licensed under LPPL. (See README,
% manifest.txt.)
% \end{package}

\begin{package}{Alegreya-Black-lf-t1--base}{alegreya}{Alegreya fonts with {\LaTeX} support.}
The Alegreya fonts are designed by Juan Pablo del Peral for
Huerta Tipografica. Alegreya is a typeface originally intended
for literature. It conveys a dynamic and varied rhythm which
facilitates the reading of long texts. The italic has just as
much care and attention to detail in the design as the roman.
Bold, black, small caps and five number styles are available.
\end{package}
% \begin{package}{Alegreya-BlackItalic-lf-t1--base}{alegreya}{Alegreya fonts with {\LaTeX} support.}
% The Alegreya fonts are designed by Juan Pablo del Peral for
% Huerta Tipografica. Alegreya is a typeface originally intended
% for literature. It conveys a dynamic and varied rhythm which
% facilitates the reading of long texts. The italic has just as
% much care and attention to detail in the design as the roman.
% Bold, black, small caps and five number styles are available.
% \end{package}
\begin{package}{Alegreya-Bold-lf-t1--base}{alegreya}{Alegreya fonts with {\LaTeX} support.}
The Alegreya fonts are designed by Juan Pablo del Peral for
Huerta Tipografica. Alegreya is a typeface originally intended
for literature. It conveys a dynamic and varied rhythm which
facilitates the reading of long texts. The italic has just as
much care and attention to detail in the design as the roman.
Bold, black, small caps and five number styles are available.
\end{package}
% \begin{package}{Alegreya-BoldItalic-lf-t1--base}{alegreya}{Alegreya fonts with {\LaTeX} support.}
% The Alegreya fonts are designed by Juan Pablo del Peral for
% Huerta Tipografica. Alegreya is a typeface originally intended
% for literature. It conveys a dynamic and varied rhythm which
% facilitates the reading of long texts. The italic has just as
% much care and attention to detail in the design as the roman.
% Bold, black, small caps and five number styles are available.
% \end{package}
\begin{package}{Alegreya-Italic-lf-t1--base}{alegreya}{Alegreya fonts with {\LaTeX} support.}
The Alegreya fonts are designed by Juan Pablo del Peral for
Huerta Tipografica. Alegreya is a typeface originally intended
for literature. It conveys a dynamic and varied rhythm which
facilitates the reading of long texts. The italic has just as
much care and attention to detail in the design as the roman.
Bold, black, small caps and five number styles are available.
\end{package}
\begin{package}{Alegreya-Regular-lf-t1--base}{alegreya}{Alegreya fonts with {\LaTeX} support.}
The Alegreya fonts are designed by Juan Pablo del Peral for
Huerta Tipografica. Alegreya is a typeface originally intended
for literature. It conveys a dynamic and varied rhythm which
facilitates the reading of long texts. The italic has just as
much care and attention to detail in the design as the roman.
Bold, black, small caps and five number styles are available.
\end{package}

\begin{package}{fruakr}{allrunes}{Fonts and {\LaTeX} package for almost all runes.}
This large collection of fonts (in Adobe Type~1 format), with
the {\LaTeX} package gives access to almost all runes ever used in
Europe. The bundle covers not only the main forms but also a
lot of varieties.
\end{package}

\begin{package}{AnonymousPro-Bold-t1}{anonymouspro}{Use AnonymousPro fonts with {\LaTeX}.}
The fonts are a monowidth set, designed for use by coders. They
appear as a set of four TrueType, or Adobe Type~1 font files,
and {\LaTeX} support is also provided.
\end{package}
% \begin{package}{AnonymousPro-BoldItalic-t1}{anonymouspro}{Use AnonymousPro fonts with {\LaTeX}.}
% The fonts are a monowidth set, designed for use by coders. They
% appear as a set of four TrueType, or Adobe Type~1 font files,
% and {\LaTeX} support is also provided.
% \end{package}
\begin{package}{AnonymousPro-Italic-t1}{anonymouspro}{Use AnonymousPro fonts with {\LaTeX}.}
The fonts are a monowidth set, designed for use by coders. They
appear as a set of four TrueType, or Adobe Type~1 font files,
and {\LaTeX} support is also provided.
\end{package}
\begin{package}{AnonymousPro-Regular-t1}{anonymouspro}{Use AnonymousPro fonts with {\LaTeX}.}
The fonts are a monowidth set, designed for use by coders. They
appear as a set of four TrueType, or Adobe Type~1 font files,
and {\LaTeX} support is also provided.
\end{package}

\begin{package}{uaqr8rc}{antiqua}{URW Antiqua condensed font, for use with {\TeX}.}
The directory contains a copy of the Type~1 font ``URW Antiqua
2051 Regular Condensed'' released under the GPL by URW, with
supporting files for use with (La){\TeX}.
\end{package}

\begin{package}{ec-anttr}{antt}{Antykwa Torunska: a Type~1 family of a Polish traditional type.}
Antykwa Torunska is a serif font designed by the late Polish
typographer Zygfryd Gardzielewski, reconstructed and digitized
as as Type~1.
\end{package}

\begin{package}{proto10}{archaic}{A collection of archaic fonts.}
The collection contains fonts to represent Aramaic, Cypriot,
Etruscan, Greek of the 6th and 4th centuries BCE, Egyptian
hieroglyphics, Linear A, Linear B, Nabatean old Persian, the
Phaistos disc, Phoenician, proto-Semitic, runic, South Arabian
Ugaritic and Viking scripts. The bundle also includes a small
font for use in phonetic transcription of the archaic writings.
The bundle's own directory includes a font installation map
file for the whole collection.
\end{package}

% \begin{package}{favr8r}{arev}{Fonts and {\LaTeX} support files for Arev Sans.}
% The package arev provides type 1 and virtual fonts, together
% with {\LaTeX} packages for using Arev Sans in both text and
% mathematics. Arev Sans is a derivative of Bitstream Vera Sans
% created by Tavmjong Bah, adding support for Greek and Cyrillic
% characters. Bah also added a few variant letters that are more
% appropriate for mathematics. The primary purpose for using Arev
% Sans in {\LaTeX} is presentations, particularly when using a
% computer projector. In such a context, Arev Sans is quite
% readable, with large x-height, ``open letters'', wide spacing,
% and thick stems. The style is very similar to the SliTeX font
% lcmss, but heavier. Arev is one of a very small number of sans-
% font mathematics support packages. Others are cmbright, hvmath
% and kerkis.
% \end{package}

\begin{package}{ASCII}{ascii-font}{Use the ASCII ``font'' in {\LaTeX}.}
The package provides glyph and font access commands so that
{\LaTeX} users can use the ASCII glyphs in their documents. The
ASCII font is encoded according to the IBM PC Code Page 437 C0
Graphics. This package replaces any early {\LaTeX}~2.09 package
and ``font'' by R. Ramasubramanian and R.W.D. Nickalls.
\end{package}

%\begin{package}{}{aspectratio}{Capital A and capital R ligature for Aspect Ratio.}
%The package provides fonts (both as Adobe Type~1 format, and as
%Metafont source) for the 'AR' symbol (for Aspect Ratio) used by
%aeronautical scientists and engineers.
%\end{package}

%\begin{package}{}{astro}{Astronomical (planetary) symbols.}
%Astrosym is a font containing astronomical symbols, including
%those used for the planets, four planetoids, the phases of the
%moon, the signs of the zodiac, and some additional symbols. The
%font is distributed in MetaFont format.
%\end{package}

\begin{package}{augie8r}{augie}{Calligraphic font for typesetting handwriting.}
A calligraphic font for simulating American-style informal
handwriting. The font is distributed in Adobe Type~1 format.
\end{package}

\begin{package}{auncl10}{auncial-new}{Artificial Uncial font and {\LaTeX} support macros.}
The auncial-new bundle provides packages and fonts for a script
based on the Artificial Uncial manuscript book-hand used
between the 6th \& 10th century AD. The script consists of
minuscules and digits, with some appropriate period punctuation
marks. Both normal and bold versions are provided, and the font
is distributed in Adobe Type~1 format. This is an experimental
new version of the auncial bundle, which is one of a series of
bookhand fonts. The font follows the B1 encoding developed for
bookhands. Access to the\ldots% encoding is essential. The encoding
%mainly follows the standard T1 encoding.
\end{package}
% \begin{package}{aunclb10}{auncial-new}{Artificial Uncial font and {\LaTeX} support macros.}
% The auncial-new bundle provides packages and fonts for a script
% based on the Artificial Uncial manuscript book-hand used
% between the 6th \& 10th century AD. The script consists of
% minuscules and digits, with some appropriate period punctuation
% marks. Both normal and bold versions are provided, and the font
% is distributed in Adobe Type~1 format. This is an experimental
% new version of the % auncial bundle, which is one of a series of
% % bookhand fonts. The font follows the B1 encoding developed for
% % bookhands. Access to the encoding is essential. The encoding
% % mainly follows the standard T1 encoding.
% \end{package}

\begin{package}{AmiciLogo}{aurical}{Calligraphic fonts for use with {\LaTeX} in T1 encoding.}
The package that implements a set (AuriocusKalligraphicus) of
three calligraphic fonts derived from the author's handwriting
in Adobe Type~1 Format, T1 (Cork) encoding for use with {\LaTeX}:
- Auriocus Kalligraphicus; - Lukas Svatba; and - Jana Skrivana.
Each font features oldstyle digits and (machine-generated)
boldface and slanted versions. A variant of Lukas Svatba offers
a 'long s'.
\end{package}
\begin{package}{AmiciLogoRslant}{aurical}{Calligraphic fonts for use with {\LaTeX} in T1 encoding.}
The package that implements a set (AuriocusKalligraphicus) of
three calligraphic fonts derived from the author's handwriting
in Adobe Type~1 Format, T1 (Cork) encoding for use with {\LaTeX}:
- Auriocus Kalligraphicus; - Lukas Svatba; and - Jana Skrivana.
Each font features oldstyle digits and (machine-generated)
boldface and slanted versions. A variant of Lukas Svatba offers
a 'long s'.
\end{package}
\begin{package}{AmiciLogoSlant}{aurical}{Calligraphic fonts for use with {\LaTeX} in T1 encoding.}
The package that implements a set (AuriocusKalligraphicus) of
three calligraphic fonts derived from the author's handwriting
in Adobe Type~1 Format, T1 (Cork) encoding for use with {\LaTeX}:
- Auriocus Kalligraphicus; - Lukas Svatba; and - Jana Skrivana.
Each font features oldstyle digits and (machine-generated)
boldface and slanted versions. A variant of Lukas Svatba offers
a 'long s'.
\end{package}
\begin{package}{AmiciLogoBold}{aurical}{Calligraphic fonts for use with {\LaTeX} in T1 encoding.}
The package that implements a set (AuriocusKalligraphicus) of
three calligraphic fonts derived from the author's handwriting
in Adobe Type~1 Format, T1 (Cork) encoding for use with {\LaTeX}:
- Auriocus Kalligraphicus; - Lukas Svatba; and - Jana Skrivana.
Each font features oldstyle digits and (machine-generated)
boldface and slanted versions. A variant of Lukas Svatba offers
a 'long s'.
\end{package}
\begin{package}{AmiciLogoBoldRslant}{aurical}{Calligraphic fonts for use with {\LaTeX} in T1 encoding.}
The package that implements a set (AuriocusKalligraphicus) of
three calligraphic fonts derived from the author's handwriting
in Adobe Type~1 Format, T1 (Cork) encoding for use with {\LaTeX}:
- Auriocus Kalligraphicus; - Lukas Svatba; and - Jana Skrivana.
Each font features oldstyle digits and (machine-generated)
boldface and slanted versions. A variant of Lukas Svatba offers
a 'long s'.
\end{package}
\begin{package}{AmiciLogoBoldSlant}{aurical}{Calligraphic fonts for use with {\LaTeX} in T1 encoding.}
The package that implements a set (AuriocusKalligraphicus) of
three calligraphic fonts derived from the author's handwriting
in Adobe Type~1 Format, T1 (Cork) encoding for use with {\LaTeX}:
- Auriocus Kalligraphicus; - Lukas Svatba; and - Jana Skrivana.
Each font features oldstyle digits and (machine-generated)
boldface and slanted versions. A variant of Lukas Svatba offers
a 'long s'.
\end{package}
\begin{package}{AuriocusKalligraphicus}{aurical}{Calligraphic fonts for use with {\LaTeX} in T1 encoding.}
The package that implements a set (AuriocusKalligraphicus) of
three calligraphic fonts derived from the author's handwriting
in Adobe Type~1 Format, T1 (Cork) encoding for use with {\LaTeX}:
- Auriocus Kalligraphicus; - Lukas Svatba; and - Jana Skrivana.
Each font features oldstyle digits and (machine-generated)
boldface and slanted versions. A variant of Lukas Svatba offers
a 'long s'.
\end{package}
\begin{package}{AuriocusKalligraphicusRslant}{aurical}{Calligraphic fonts for use with {\LaTeX} in T1 encoding.}
The package that implements a set (AuriocusKalligraphicus) of
three calligraphic fonts derived from the author's handwriting
in Adobe Type~1 Format, T1 (Cork) encoding for use with {\LaTeX}:
- Auriocus Kalligraphicus; - Lukas Svatba; and - Jana Skrivana.
Each font features oldstyle digits and (machine-generated)
boldface and slanted versions. A variant of Lukas Svatba offers
a 'long s'.
\end{package}
\begin{package}{AuriocusKalligraphicusSlant}{aurical}{Calligraphic fonts for use with {\LaTeX} in T1 encoding.}
The package that implements a set (AuriocusKalligraphicus) of
three calligraphic fonts derived from the author's handwriting
in Adobe Type~1 Format, T1 (Cork) encoding for use with {\LaTeX}:
- Auriocus Kalligraphicus; - Lukas Svatba; and - Jana Skrivana.
Each font features oldstyle digits and (machine-generated)
boldface and slanted versions. A variant of Lukas Svatba offers
a 'long s'.
\end{package}
\begin{package}{AuriocusKalligraphicusBold}{aurical}{Calligraphic fonts for use with {\LaTeX} in T1 encoding.}
The package that implements a set (AuriocusKalligraphicus) of
three calligraphic fonts derived from the author's handwriting
in Adobe Type~1 Format, T1 (Cork) encoding for use with {\LaTeX}:
- Auriocus Kalligraphicus; - Lukas Svatba; and - Jana Skrivana.
Each font features oldstyle digits and (machine-generated)
boldface and slanted versions. A variant of Lukas Svatba offers
a 'long s'.
\end{package}
\begin{package}{AuriocusKalligraphicusBoldRslant}{aurical}{Calligraphic fonts for use with {\LaTeX} in T1 encoding.}
The package that implements a set (AuriocusKalligraphicus) of
three calligraphic fonts derived from the author's handwriting
in Adobe Type~1 Format, T1 (Cork) encoding for use with {\LaTeX}:
- Auriocus Kalligraphicus; - Lukas Svatba; and - Jana Skrivana.
Each font features oldstyle digits and (machine-generated)
boldface and slanted versions. A variant of Lukas Svatba offers
a 'long s'.
\end{package}
\begin{package}{AuriocusKalligraphicusBoldSlant}{aurical}{Calligraphic fonts for use with {\LaTeX} in T1 encoding.}
The package that implements a set (AuriocusKalligraphicus) of
three calligraphic fonts derived from the author's handwriting
in Adobe Type~1 Format, T1 (Cork) encoding for use with {\LaTeX}:
- Auriocus Kalligraphicus; - Lukas Svatba; and - Jana Skrivana.
Each font features oldstyle digits and (machine-generated)
boldface and slanted versions. A variant of Lukas Svatba offers
a 'long s'.
\end{package}
\begin{package}{JanaSkrivana}{aurical}{Calligraphic fonts for use with {\LaTeX} in T1 encoding.}
The package that implements a set (AuriocusKalligraphicus) of
three calligraphic fonts derived from the author's handwriting
in Adobe Type~1 Format, T1 (Cork) encoding for use with {\LaTeX}:
- Auriocus Kalligraphicus; - Lukas Svatba; and - Jana Skrivana.
Each font features oldstyle digits and (machine-generated)
boldface and slanted versions. A variant of Lukas Svatba offers
a 'long s'.
\end{package}
\begin{package}{JanaSkrivanaRslant}{aurical}{Calligraphic fonts for use with {\LaTeX} in T1 encoding.}
The package that implements a set (AuriocusKalligraphicus) of
three calligraphic fonts derived from the author's handwriting
in Adobe Type~1 Format, T1 (Cork) encoding for use with {\LaTeX}:
- Auriocus Kalligraphicus; - Lukas Svatba; and - Jana Skrivana.
Each font features oldstyle digits and (machine-generated)
boldface and slanted versions. A variant of Lukas Svatba offers
a 'long s'.
\end{package}
\begin{package}{JanaSkrivanaSlant}{aurical}{Calligraphic fonts for use with {\LaTeX} in T1 encoding.}
The package that implements a set (AuriocusKalligraphicus) of
three calligraphic fonts derived from the author's handwriting
in Adobe Type~1 Format, T1 (Cork) encoding for use with {\LaTeX}:
- Auriocus Kalligraphicus; - Lukas Svatba; and - Jana Skrivana.
Each font features oldstyle digits and (machine-generated)
boldface and slanted versions. A variant of Lukas Svatba offers
a 'long s'.
\end{package}
\begin{package}{JanaSkrivanaBold}{aurical}{Calligraphic fonts for use with {\LaTeX} in T1 encoding.}
The package that implements a set (AuriocusKalligraphicus) of
three calligraphic fonts derived from the author's handwriting
in Adobe Type~1 Format, T1 (Cork) encoding for use with {\LaTeX}:
- Auriocus Kalligraphicus; - Lukas Svatba; and - Jana Skrivana.
Each font features oldstyle digits and (machine-generated)
boldface and slanted versions. A variant of Lukas Svatba offers
a 'long s'.
\end{package}
\begin{package}{JanaSkrivanaBoldRslant}{aurical}{Calligraphic fonts for use with {\LaTeX} in T1 encoding.}
The package that implements a set (AuriocusKalligraphicus) of
three calligraphic fonts derived from the author's handwriting
in Adobe Type~1 Format, T1 (Cork) encoding for use with {\LaTeX}:
- Auriocus Kalligraphicus; - Lukas Svatba; and - Jana Skrivana.
Each font features oldstyle digits and (machine-generated)
boldface and slanted versions. A variant of Lukas Svatba offers
a 'long s'.
\end{package}
\begin{package}{JanaSkrivanaSlant}{aurical}{Calligraphic fonts for use with {\LaTeX} in T1 encoding.}
The package that implements a set (AuriocusKalligraphicus) of
three calligraphic fonts derived from the author's handwriting
in Adobe Type~1 Format, T1 (Cork) encoding for use with {\LaTeX}:
- Auriocus Kalligraphicus; - Lukas Svatba; and - Jana Skrivana.
Each font features oldstyle digits and (machine-generated)
boldface and slanted versions. A variant of Lukas Svatba offers
a 'long s'.
\end{package}
\begin{package}{LukasSvatba}{aurical}{Calligraphic fonts for use with {\LaTeX} in T1 encoding.}
The package that implements a set (AuriocusKalligraphicus) of
three calligraphic fonts derived from the author's handwriting
in Adobe Type~1 Format, T1 (Cork) encoding for use with {\LaTeX}:
- Auriocus Kalligraphicus; - Lukas Svatba; and - Jana Skrivana.
Each font features oldstyle digits and (machine-generated)
boldface and slanted versions. A variant of Lukas Svatba offers
a 'long s'.
\end{package}
\begin{package}{LukasSvatbaRslant}{aurical}{Calligraphic fonts for use with {\LaTeX} in T1 encoding.}
The package that implements a set (AuriocusKalligraphicus) of
three calligraphic fonts derived from the author's handwriting
in Adobe Type~1 Format, T1 (Cork) encoding for use with {\LaTeX}:
- Auriocus Kalligraphicus; - Lukas Svatba; and - Jana Skrivana.
Each font features oldstyle digits and (machine-generated)
boldface and slanted versions. A variant of Lukas Svatba offers
a 'long s'.
\end{package}
\begin{package}{LukasSvatbaSlant}{aurical}{Calligraphic fonts for use with {\LaTeX} in T1 encoding.}
The package that implements a set (AuriocusKalligraphicus) of
three calligraphic fonts derived from the author's handwriting
in Adobe Type~1 Format, T1 (Cork) encoding for use with {\LaTeX}:
- Auriocus Kalligraphicus; - Lukas Svatba; and - Jana Skrivana.
Each font features oldstyle digits and (machine-generated)
boldface and slanted versions. A variant of Lukas Svatba offers
a 'long s'.
\end{package}
\begin{package}{LukasSvatbaBold}{aurical}{Calligraphic fonts for use with {\LaTeX} in T1 encoding.}
The package that implements a set (AuriocusKalligraphicus) of
three calligraphic fonts derived from the author's handwriting
in Adobe Type~1 Format, T1 (Cork) encoding for use with {\LaTeX}:
- Auriocus Kalligraphicus; - Lukas Svatba; and - Jana Skrivana.
Each font features oldstyle digits and (machine-generated)
boldface and slanted versions. A variant of Lukas Svatba offers
a 'long s'.
\end{package}
\begin{package}{LukasSvatbaBoldRslant}{aurical}{Calligraphic fonts for use with {\LaTeX} in T1 encoding.}
The package that implements a set (AuriocusKalligraphicus) of
three calligraphic fonts derived from the author's handwriting
in Adobe Type~1 Format, T1 (Cork) encoding for use with {\LaTeX}:
- Auriocus Kalligraphicus; - Lukas Svatba; and - Jana Skrivana.
Each font features oldstyle digits and (machine-generated)
boldface and slanted versions. A variant of Lukas Svatba offers
a 'long s'.
\end{package}
\begin{package}{LukasSvatbaBoldSlant}{aurical}{Calligraphic fonts for use with {\LaTeX} in T1 encoding.}
The package that implements a set (AuriocusKalligraphicus) of
three calligraphic fonts derived from the author's handwriting
in Adobe Type~1 Format, T1 (Cork) encoding for use with {\LaTeX}:
- Auriocus Kalligraphicus; - Lukas Svatba; and - Jana Skrivana.
Each font features oldstyle digits and (machine-generated)
boldface and slanted versions. A variant of Lukas Svatba offers
a 'long s'.
\end{package}

% \begin{package}{}{b1encoding}{{\LaTeX} encoding tools for Bookhands fonts.}
% This package characterises and defines the author's B1 encoding
% for use with {\LaTeX} when typesetting things using his Bookhands
% fonts.
% \end{package}
% \begin{package}{wlc128}{barcodes}{Fonts for making barcodes.}
% The package deals with EAN barcodes; MetaFont fonts are
% provided, and a set of examples; for some codes, a small Perl
% script is needed.
% \end{package}

% \begin{package}{yesr8r}{baskervald}{Baskervald ADF fonts collection with {\TeX}/{\LaTeX} support.}
% Baskervald ADF is a serif family with lining figures designed
% as a substitute for Baskerville. The family currently includes
% upright and italic or oblique shapes in each of regular, bold
% and heavy weights. All fonts include the slashed zero and
% additional non-standard ligatures. The support package renames
% them according to the Karl Berry fontname scheme and defines
% two families. One of these primarily provides access to the
% ``standard'' or default characters while the other supports
% additional ligatures. The included package files provide access
% to these features in {\LaTeX}.
% \end{package}

% \begin{package}{bbding10}{bbding}{A symbol (dingbat) font and {\LaTeX} macros for its use.}
% A symbol font (distributed as MetaFont source) that contains
% many of the symbols of the Zapf dingbats set, together with an
% NFSS interface for using the font. An Adobe Type~1 version of
% the fonts is available in the niceframe fonts bundle.
% \end{package}
% \begin{package}{bbm10}{bbm}{``Blackboard-style'' cm fonts.}
% Blackboard variants of Computer Modern fonts. The fonts are
% distributed as MetaFont source (only); {\LaTeX} support is
% available with the bbm-macros package. The Sauter font package
% has MetaFont parameter source files for building the fonts at
% more sizes than you could reasonably imagine. A sample of these
% fonts appears in the blackboard bold sampler.
% \end{package}
% \begin{package}{}{bbm-macros}{{\LaTeX} support for ``blackboard-style'' cm fonts.}
% Provides {\LaTeX} support for Blackboard variants of Computer
% Modern fonts. Declares a font family bbm so you can in
% principle write running text in blackboard bold, and lots of
% math alphabets for using the fonts within maths.
% \end{package}
% \begin{package}{}{bbold}{Sans serif blackboard bold.}
% A geometric sans serif blackboard bold font, for use in
% mathematics; MetaFont sources are provided, as well as macros
% for use with {\LaTeX}. The Sauter font package has MetaFont
% parameter source files for building the fonts at more sizes
% than you could reasonably imagine. See the blackboard sampler
% for a feel for the font's appearance.
% \end{package}
\begin{package}{bbold10}{bbold-type1}{An Adobe Type~1 format version of the bbold font.}
The files offer an Adobe Type~1 format version of the 5pt, 7pt
and 10pt versions of the bbold fonts. The distribution also
includes a map file, for use when incorporating the fonts into
TeX documents, but no macro sets are provided (the fonts will
not provide the correct results using macros designed for use
with the MetaFont versions of the fonts. The fonts were
produced to be part of the {\TeX} distribution from Y\&Y; they were
generously donated to the {\TeX} Users' Group when Y\&Y closed its
doors as a business.
\end{package}
% \begin{package}{}{belleek}{Free replacement for basic MathTime fonts.}
% This package replaces the original MathTime fonts, not
% MathTime-Plus or MathTime Professional (the last being the only
% currently available commercial bundle).
% \end{package}

\begin{package}{fveb8r}{bera: BeraSerif-Bold}{Bera fonts.}
The package contains the Bera Type~1 fonts, and a zip archive
containing files to use the fonts with {\LaTeX}. Bera is a set of
three font families: Bera Serif (a slab-serif Roman), Bera Sans
(a Frutiger descendant), and Bera Mono (monospaced/typewriter).
Support for use in {\LaTeX} is also provided. The Bera family is a
repackaging, for use with {\TeX}, of the Bitstream Vera family.
\end{package}
\begin{package}{fver8r}{bera: BeraSerif-Roman}{Bera fonts.}
The package contains the Bera Type~1 fonts, and a zip archive
containing files to use the fonts with {\LaTeX}. Bera is a set of
three font families: Bera Serif (a slab-serif Roman), Bera Sans
(a Frutiger descendant), and Bera Mono (monospaced/typewriter).
Support for use in {\LaTeX} is also provided. The Bera family is a
repackaging, for use with {\TeX}, of the Bitstream Vera family.
\end{package}
% \begin{package}{fvmb8r}{bera: BeraSansMono-Bold}{Bera fonts.}
% The package contains the Bera Type~1 fonts, and a zip archive
% containing files to use the fonts with {\LaTeX}. Bera is a set of
% three font families: Bera Serif (a slab-serif Roman), Bera Sans
% (a Frutiger descendant), and Bera Mono (monospaced/typewriter).
% Support for use in {\LaTeX} is also provided. The Bera family is a
% repackaging, for use with {\TeX}, of the Bitstream Vera family.
% \end{package}
\begin{package}{fvmr8r}{bera: BeraSansMono-Roman}{Bera fonts.}
The package contains the Bera Type~1 fonts, and a zip archive
containing files to use the fonts with {\LaTeX}. Bera is a set of
three font families: Bera Serif (a slab-serif Roman), Bera Sans
(a Frutiger descendant), and Bera Mono (monospaced/typewriter).
Support for use in {\LaTeX} is also provided. The Bera family is a
repackaging, for use with {\TeX}, of the Bitstream Vera family.
\end{package}
\begin{package}{fvsb8r}{bera: BeraSans-Bold}{Bera fonts.}
The package contains the Bera Type~1 fonts, and a zip archive
containing files to use the fonts with {\LaTeX}. Bera is a set of
three font families: Bera Serif (a slab-serif Roman), Bera Sans
(a Frutiger descendant), and Bera Mono (monospaced/typewriter).
Support for use in {\LaTeX} is also provided. The Bera family is a
repackaging, for use with {\TeX}, of the Bitstream Vera family.
\end{package}
\begin{package}{fvsr8r}{bera: BeraSans-Roman}{Bera fonts.}
The package contains the Bera Type~1 fonts, and a zip archive
containing files to use the fonts with {\LaTeX}. Bera is a set of
three font families: Bera Serif (a slab-serif Roman), Bera Sans
(a Frutiger descendant), and Bera Mono (monospaced/typewriter).
Support for use in {\LaTeX} is also provided. The Bera family is a
repackaging, for use with {\TeX}, of the Bitstream Vera family.
\end{package}

\begin{package}{ybdb28t}{berenisadf}{Berenis ADF fonts and {\TeX}/{\LaTeX} support.}
The bundle provides the BerenisADF Pro font collection, in
OpenType and PostScript Type~1 formats, together with support
files to use the fonts in {\TeX}nANSI (LY1) and {\LaTeX} standard T1
and TS1 encodings.
\end{package}
\begin{package}{ybdb2i8t}{berenisadf}{Berenis ADF fonts and {\TeX}/{\LaTeX} support.}
The bundle provides the BerenisADF Pro font collection, in
OpenType and PostScript Type~1 formats, together with support
files to use the fonts in {\TeX}nANSI (LY1) and {\LaTeX} standard T1
and TS1 encodings.
\end{package}
\begin{package}{ybdr28t}{berenisadf}{Berenis ADF fonts and {\TeX}/{\LaTeX} support.}
The bundle provides the BerenisADF Pro font collection, in
OpenType and PostScript Type~1 formats, together with support
files to use the fonts in {\TeX}nANSI (LY1) and {\LaTeX} standard T1
and TS1 encodings.
\end{package}
\begin{package}{ybdr2i8t}{berenisadf}{Berenis ADF fonts and {\TeX}/{\LaTeX} support.}
The bundle provides the BerenisADF Pro font collection, in
OpenType and PostScript Type~1 formats, together with support
files to use the fonts in {\TeX}nANSI (LY1) and {\LaTeX} standard T1
and TS1 encodings.
\end{package}
\begin{package}{ybdr2j8t}{berenisadf}{Berenis ADF fonts and {\TeX}/{\LaTeX} support.}
The bundle provides the BerenisADF Pro font collection, in
OpenType and PostScript Type~1 formats, together with support
files to use the fonts in {\TeX}nANSI (LY1) and {\LaTeX} standard T1
and TS1 encodings.
\end{package}

% \begin{package}{bguq10t10}{bguq}{Improved quantifier stroke for Begriffsschrift packages.}
% The font contains a single character: the Begriffsschrift
% quantifier (in several sizes), as used to set the
% Begriffsschrift (concept notation) of Frege. The font is not
% intended for end users; instead it is expected that it will be
% used by other packages which implement the Begriffsschrift. An
% (unofficial) modified version of Josh Parsons' begriff is
% included as an example of implementation.
% \end{package}
% \begin{package}{}{blacklettert1}{T1-encoded versions of Haralambous old German fonts.}
% This package provides virtual fonts for T1-like variants of
% Yannis Haralambous's old German fonts Gothic, Schwabacher and
% Fraktur (which are also available in Adobe type 1 format). The
% package includes {\LaTeX} macros to embed the fonts into the {\LaTeX}
% font selection scheme.
% \end{package}

% \begin{package}{bskr10}{boisik}{A font inspired by Baskerville design.}
% Boisik is a serif font (inspired by the Baskerville typeface),
% written in MetaFont. It comprises roman and italic text fonts
% and maths fonts. {\LaTeX} support is offered for use with OT1, IL2
% and OM* encodings.
% \end{package}

\begin{package}{sqrc10}{bookhands}{A collection of book-hand fonts.}
This is a set of book-hand (MetaFont) fonts and packages
covering manuscript scripts from the 1st century until
Gutenberg and Caxton. The included hands are: Square Capitals
(1st century onwards); Roman Rustic (1st-6th centuries);
Insular Minuscule (6th cenury onwards); Carolingian Minuscule
(8th-12th centuries); Early Gothic (11th-12th centuries);
Gothic Textura Quadrata (13th-15th centuries); Gothic Textura
Prescisus vel sine pedibus (13th century onwards); Rotunda (13-
15th centuries); Humanist Minuscule (14th century onwards);
Uncial (3rd-6th centuries); Half Uncial (3rd-9th centuries);
Artificial Uncial (6th-10th centuries); and Insular Majuscule\ldots
%(6th-9th centuries).
\end{package}
% \begin{package}{sqrcb10}{bookhands}{A collection of book-hand fonts.}
% This is a set of book-hand (MetaFont) fonts and packages
% covering manuscript scripts from the 1st century until
% Gutenberg and Caxton. The included hands are: Square Capitals
% (1st century onwards); Roman Rustic (1st-6th centuries);
% Insular Minuscule (6th cenury onwards); Carolingian Minuscule
% (8th-12th centuries); Early Gothic (11th-12th centuries);
% Gothic Textura Quadrata (13th-15th centuries); Gothic Textura
% Prescisus vel sine pedibus (13th century onwards); Rotunda (13-
% 15th centuries); Humanist Minuscule (14th century onwards);
% Uncial (3rd-6th centuries); Half Uncial (3rd-9th centuries);
% Artificial Uncial (6th-10th centuries); and Insular Majuscule\ldots
% %(6th-9th centuries).
% \end{package}

\begin{package}{zxxrl7z}{boondox}{Mathematical alphabets derived from the STIX fonts.}
The package contains a number of PostScript fonts derived from
the STIX OpenType fonts, that may be used in maths mode in
regular and bold weights for calligraphic, fraktur and double-
struck alphabets. Virtual fonts with metrics suitable for maths
mode are provided, as are {\LaTeX} support files.
\end{package}

% \begin{package}{}{braille}{Support for braille.}
% This package allows the user to produce Braille documents on
% paper for the blind without knowing Braille (which can take
% years to learn). Python scripts grade1.py and grade2.py convert
% ordinary text to grade 1 and 2 braille tags; then, the {\LaTeX}
% package braille.sty takes the tags and prints out corresponding
% braille symbols.
% \end{package}

\begin{package}{pbsi8r}{brushscr}{A handwriting script font.}
The BrushScript font simulates hand-written characters; it is
distributed in Adobe Type~1 format (but is available in italic
shape only). The package includes the files needed by {\LaTeX} in
order to use that font. The file AAA\_readme.tex fully describes
the package and sample.tex illustrates its use.
\end{package}

\begin{package}{Cabin-Bold-tlf-t1--base}{cabin}{A humanist Sans Serif font, with {\LaTeX} support.}
Cabin is a humanist sans with four weights and true italics and
small capitals. According to the designer, Pablo Impallari,
Cabin was inspired by Edward Johnston's and Eric Gill's
typefaces, with a touch of modernism. Cabin incorporates modern
proportions, optical adjustments, and some elements of the
geometric sans. cabin.sty supports use of the font under {\LaTeX},
{\pdfLaTeX}, {\XeLaTeX} and {\LuaLaTeX}; it uses the mweights, to manage
the user's view of all those font weights. An sfdefault option
is provided to enable Cabin as the default text font. The
fontaxes package is required for use with [pdf]{\LaTeX}.
\end{package}
% \begin{package}{Cabin-BoldItalic-tlf-t1--base}{cabin}{A humanist Sans Serif font, with {\LaTeX} support.}
% Cabin is a humanist sans with four weights and true italics and
% small capitals. According to the designer, Pablo Impallari,
% Cabin was inspired by Edward Johnston's and Eric Gill's
% typefaces, with a touch of modernism. Cabin incorporates modern
% proportions, optical adjustments, and some elements of the
% geometric sans. cabin.sty supports use of the font under {\LaTeX},
% {\pdfLaTeX}, {\XeLaTeX} and {\LuaLaTeX}; it uses the mweights, to manage
% the user's view of all those font weights. An sfdefault option
% is provided to enable Cabin as the default text font. The
% fontaxes package is required for use with [pdf]{\LaTeX}.
% \end{package}
\begin{package}{Cabin-Italic-tlf-t1--base}{cabin}{A humanist Sans Serif font, with {\LaTeX} support.}
Cabin is a humanist sans with four weights and true italics and
small capitals. According to the designer, Pablo Impallari,
Cabin was inspired by Edward Johnston's and Eric Gill's
typefaces, with a touch of modernism. Cabin incorporates modern
proportions, optical adjustments, and some elements of the
geometric sans. cabin.sty supports use of the font under {\LaTeX},
{\pdfLaTeX}, {\XeLaTeX} and {\LuaLaTeX}; it uses the mweights, to manage
the user's view of all those font weights. An sfdefault option
is provided to enable Cabin as the default text font. The
fontaxes package is required for use with [pdf]{\LaTeX}.
\end{package}
% \begin{package}{Cabin-Medium-tlf-t1--base}{cabin}{A humanist Sans Serif font, with {\LaTeX} support.}
% Cabin is a humanist sans with four weights and true italics and
% small capitals. According to the designer, Pablo Impallari,
% Cabin was inspired by Edward Johnston's and Eric Gill's
% typefaces, with a touch of modernism. Cabin incorporates modern
% proportions, optical adjustments, and some elements of the
% geometric sans. cabin.sty supports use of the font under {\LaTeX},
% {\pdfLaTeX}, {\XeLaTeX} and {\LuaLaTeX}; it uses the mweights, to manage
% the user's view of all those font weights. An sfdefault option
% is provided to enable Cabin as the default text font. The
% fontaxes package is required for use with [pdf]{\LaTeX}.
% \end{package}
% \begin{package}{Cabin-MediumItalic-tlf-t1--base}{cabin}{A humanist Sans Serif font, with {\LaTeX} support.}
% Cabin is a humanist sans with four weights and true italics and
% small capitals. According to the designer, Pablo Impallari,
% Cabin was inspired by Edward Johnston's and Eric Gill's
% typefaces, with a touch of modernism. Cabin incorporates modern
% proportions, optical adjustments, and some elements of the
% geometric sans. cabin.sty supports use of the font under {\LaTeX},
% {\pdfLaTeX}, {\XeLaTeX} and {\LuaLaTeX}; it uses the mweights, to manage
% the user's view of all those font weights. An sfdefault option
% is provided to enable Cabin as the default text font. The
% fontaxes package is required for use with [pdf]{\LaTeX}.
% \end{package}
\begin{package}{Cabin-Regular-tlf-t1--base}{cabin}{A humanist Sans Serif font, with {\LaTeX} support.}
Cabin is a humanist sans with four weights and true italics and
small capitals. According to the designer, Pablo Impallari,
Cabin was inspired by Edward Johnston's and Eric Gill's
typefaces, with a touch of modernism. Cabin incorporates modern
proportions, optical adjustments, and some elements of the
geometric sans. cabin.sty supports use of the font under {\LaTeX},
{\pdfLaTeX}, {\XeLaTeX} and {\LuaLaTeX}; it uses the mweights, to manage
the user's view of all those font weights. An sfdefault option
is provided to enable Cabin as the default text font. The
fontaxes package is required for use with [pdf]{\LaTeX}.
\end{package}
% \begin{package}{Cabin-SemiBold-tlf-t1--base}{cabin}{A humanist Sans Serif font, with {\LaTeX} support.}
% Cabin is a humanist sans with four weights and true italics and
% small capitals. According to the designer, Pablo Impallari,
% Cabin was inspired by Edward Johnston's and Eric Gill's
% typefaces, with a touch of modernism. Cabin incorporates modern
% proportions, optical adjustments, and some elements of the
% geometric sans. cabin.sty supports use of the font under {\LaTeX},
% {\pdfLaTeX}, {\XeLaTeX} and {\LuaLaTeX}; it uses the mweights, to manage
% the user's view of all those font weights. An sfdefault option
% is provided to enable Cabin as the default text font. The
% fontaxes package is required for use with [pdf]{\LaTeX}.
% \end{package}
% \begin{package}{Cabin-SemiBoldItalic-tlf-t1--base}{cabin}{A humanist Sans Serif font, with {\LaTeX} support.}
% Cabin is a humanist sans with four weights and true italics and
% small capitals. According to the designer, Pablo Impallari,
% Cabin was inspired by Edward Johnston's and Eric Gill's
% typefaces, with a touch of modernism. Cabin incorporates modern
% proportions, optical adjustments, and some elements of the
% geometric sans. cabin.sty supports use of the font under {\LaTeX},
% {\pdfLaTeX}, {\XeLaTeX} and {\LuaLaTeX}; it uses the mweights, to manage
% the user's view of all those font weights. An sfdefault option
% is provided to enable Cabin as the default text font. The
% fontaxes package is required for use with [pdf]{\LaTeX}.
% \end{package}

% \begin{package}{}{calligra}{Calligraphic font.}
% A calligraphic font in the handwriting style of the author,
% Peter Vanroose. The font is supplied as MetaFont source {\LaTeX}
% support of the font is provided in the calligra package in the
% fundus bundle.
% \end{package}

\begin{package}{callig15}{calligra-type1}{Type~1 version of Caliigra.}
This is a converstion (using mf2pt1 of Peter Vanroose's
handwriting font.

A calligraphic font in the handwriting style of the author,
Peter Vanroose. The font is supplied as MetaFont source {\LaTeX}
support of the font is provided in the calligra package in the
fundus bundle.
\end{package}

% \begin{package}{}{cantarell}{{\LaTeX} support for the Cantarell font family.}
% Cantarell is a contemporary Humanist sans serif designed by
% Dave Crossland and Jakub Steiner. This font, delivered under
% the OFL version~1.1, is available on the GNOME download server.
% The present package provides support for this font in {\LaTeX}. It
% includes Type~1 versions of the fonts, converted for this
% package using FontForge from its sources, for full support with
% Dvips.
% \end{package}

\begin{package}{cmin17}{carolmin-ps: CarolinganMinusculesRegular17}{Adobe Type~1 format of Carolingian Minuscule fonts.}
The bundle offers Adobe Type~1 format versions of Peter
Wilson's Carolingian Minuscule font set (part of the bookhands
collection). The fonts in the bundle are ready-to-use
replacements for the MetaFont originals.
\end{package}
\begin{package}{cminb17}{carolmin-ps: CarolinganMinusculesBold17}{Adobe Type~1 format of Carolingian Minuscule fonts.}
The bundle offers Adobe Type~1 format versions of Peter
Wilson's Carolingian Minuscule font set (part of the bookhands
collection). The fonts in the bundle are ready-to-use
replacements for the MetaFont originals.
\end{package}

% \begin{package}{ccicons}{ccicons}{{\LaTeX} support for Creative Commons icons.}
% The package provides the means to typeset Creative Commons
% icons, in documents licensed under CC licences. A font (in
% Adobe Type~1 format) and {\LaTeX} support macros are provided.
% \end{package}

% \begin{package}{}{cfr-lm}{Enhanced support for the Latin Modern fonts.}
% The package supports a number of features of the Latin Modern
% fonts which are not easily accessible via the default (La){\TeX}
% support provided in the official distribution. In particular,
% the package supports the use of the various styles of digits
% available, small-caps and upright italic shapes, and
% alternative weights and widths. It also supports variable width
% typewriter and the ``quotation'' font. Version 2.004 of the Latin
% Modern fonts is supported. By default, the package uses
% proportional oldstyle digits and variable width typewriter but
% this can be changed by passing appropriate options to the
% package. The package also supports using (for example)
% different styles of digits within a document so it is possible
% to use proportional oldstyle digits by default, say, but
% tabular lining digits within a particular table. The package
% requires the official Latin Modern distribution, including its
% (La){\TeX} support. The package relies on the availability of both
% the fonts themselves and the official font support files. The
% package also makes use of the nfssext-cfr package. Only the T1
% and TS1 encodings are supported for text fonts. The set up of
% fonts for mathematics is identical to that provided by Latin
% Modern.
% \end{package}

% \begin{package}{}{cherokee}{A font for the Cherokee script.}
% The Cherokee script was designed in 1821 by Segwoya. The
% alphabet is essentially syllabic, only 6 characters (a e i o s
% u) correspond to Roman letters: the font encodes these to the
% corresponding roman letter. The remaining 79 characters have
% been arbitrarily encoded in the range 38-122; the cherokee
% package provides commands that map each such syllable to the
% appropriate character; for example, Segwoya himself would be
% represented CseCgwoCya.
% \end{package}

% \begin{package}{fcmr8r}{cm-lgc}{Type~1 CM-based fonts for Latin, Greek and Cyrillic.}
% The fonts are converted from Metafont sources of the Computer
% Modern font families, using textrace. Supported encodings are:
% T1 (Latin), T2A (Cyrillic), LGR (Greek) and TS1. The package
% also includes Unicode virtual fonts for use with Omega. The
% font set is not a replacement for any of the other Computer
% Modern-based font sets (for example, cm-super for Latin and
% Cyrillic, or cbgreek for Greek), since it is available at a
% single size only; it offers a compact set for 'general'
% working. The fonts themselves are encoded to external
% standards, and virtual fonts are provided for use with {\TeX}.
% \end{package}

% \begin{package}{}{cm-unicode}{Computer Modern Unicode font family.}
% Computer Modern Unicode fonts were converted from metafont
% sources using mftrace with autotrace backend and fontforge.
% Some characters in several fonts are copied from Blue Sky type
% 1 fonts released by AMS. Currently the fonts contain glyphs
% from Latin (Metafont ec, tc, vnr), Cyrillic (lh), Greek
% (cbgreek when available) code sets and IPA extensions (from
% tipa). This font set contains 33 fonts. This archive contains
% AFM, PFB and OTF versions; the OTF version of the Computer
% Modern Unicode fonts works with {\TeX} engines that directly
% support OpenType features, such as {\XeTeX} and LuaTeX.
% \end{package}

% \begin{package}{}{cmbright}{Computer Modern Bright fonts.}
% A family of sans serif fonts for {\TeX} and {\LaTeX}, based on Donald
% Knuth's CM fonts. It comprises OT1, T1 and TS1 encoded text
% fonts of various shapes as well as all the fonts necessary for
% mathematical typesetting, including AMS symbols. This
% collection provides all the necessary files for using the fonts
% with {\LaTeX}. A commercial-quality Adobe Type~1 version of these
% fonts is available from Micropress. Free versions are
% available, in the cm-super font bundle (the T1 and TS1 encoded
% part of the set), and in the hfbright (the OT1 encoded part,
% and the maths fonts).
% \end{package}

% \begin{package}{cmllr10}{cmll}{Symbols for linear logic.}
% This is a very small font set that contain some symbols useful
% in linear logic, which are apparently not available elsewhere.
% Variants are included for use with Computer Modern serif and
% sans-serif and with the AMS Euler series. The font is provided
% both as MetaFont source, and in Adobe Type~1 format. {\LaTeX}
% support is provided. format.
% \end{package}

% \begin{package}{}{cmpica}{A Computer Modern Pica variant.}
% An approximate equivalent of the Xerox Pica typeface; the font
% is optimised for submitting fiction manuscripts to mainline
% publishers. The font is a fixed-width one, rather less heavy
% than Computer Modern typewriter. Emphasis for bold-face comes
% from a wavy underline of each letter. The two fonts are
% supplied as MetaFont source.
% \end{package}
% \begin{package}{}{cmtiup}{Upright punctuation with CM slanted.}
% The cmtiup fonts address a problem with the appearance of
% punctuation in italic text in mathematical documents. To
% achieve this, all punctuation characters are upright, and
% kerning between letters and punctuation is adjusted to allow
% for the italic correction. The fonts are implemented as a set
% of vf files; a package for support in {\LaTeXe} is provided.
% \end{package}
% \begin{package}{}{comfortaa}{Sans serif font, with {\LaTeX} support.}
% Comfortaa is a sans-serif font, comfortable in every aspect,
% designed by Johan Aakerlund. The font, which includes three
% weights (thin, regular and bold), is available on Johan's
% deviantArt web page as TrueType files under the Open Font
% License version~1.1. This package provides support for this
% font in {\LaTeX}, and includes both the TrueType fonts, and
% conversions to Adobe Type~1 format.
% \end{package}
% \begin{package}{}{concmath-fonts}{Concrete mathematics fonts.}
% The fonts are derived from the computer modern mathematics
% fonts and from Knuth's Concrete Roman fonts. {\LaTeX} support is
% offered by the concmath package.
% \end{package}
% \begin{package}{CountriesOfEurope}{countriesofeurope}{A font with the images of the countries of Europe.}
% The bundle provides a font ``CountriesOfEurope'' (in Adobe Type~1
% format) and the necessary metrics, together with {\LaTeX} macros
% for its use. The font provides glyphs with a filled outline of
% the shape of each country; each glyph is at the same
% cartographic scale.
% \end{package}
% \begin{package}{}{courier-scaled}{Provides a scaled Courier font.}
% This package sets the default typewriter font to Courier with a
% possible scale factor (in the same way as the helvet package
% for Helvetica works for sans serif).
% \end{package}
% \begin{package}{}{cryst}{Font for graphical symbols used in crystallography.}
% The font is provided as an Adobe Type~1 font, and as MetaFont
% source. Instructions for use are available both in the README
% file and (with a font diagram) in the documentation.
% \end{package}

\begin{package}{ec-cyklopi}{cyklop: Cyklop-Italic}{The Cyclop typeface.}
The Cyclop typeface was designed in the 1920s at the workshop
of Warsaw type foundry ``Odlewnia Czcionek J. Idzkowski i S-ka''.
This sans serif typeface has a highly modulated stroke so it
has high typographic contrast. The vertical stems are much
heavier then horizontal ones. Most characters have thin
rectangles as additional counters giving the unique shape of
the characters. The lead types of Cyclop typeface were produced
in slanted variant at sizes 8-48 pt. It was heavily used for
heads in newspapers and accidents prints. Typesetters used
Cyclop in the inter-war period, % during the occupation in the
% underground press. The typeface was used until the beginnings
% of the offset print and computer typesetting era. Nowadays it
% is hard to find the metal types of this typeface. The font was
% generated using the Metatype1 package. Then the original set of
% characters was completed by adding the full set of accented
% letters and characters of the modern Latin alphabets (including
% Vietnamese). The upright variant was generated and it was more
% complicated task than it appeared at the beginning. 11 upright
% letters of the Cyclop typeface were presented in the book by
% Filip Trzaska, ``Podstawy techniki wydawniczej'' (``Foundation of
% the publishing techonology''), Warsaw 1967. But even the author
% of the book does not know what was the source of the presented
% examples. The fonts are distributed in the Type1 and OpenType
% formats along with the files necessary for use these fonts in
% TeX and {\LaTeX} including encoding definition files: T1 (ec), T5
% (Vietnamese), OT4, QX, texnansi and nonstandard ones (IL2 for
% Czech fonts).
\end{package}
\begin{package}{ec-cyklopr}{cyklop: Cyklop-Regular}{The Cyclop typeface.}
The Cyclop typeface was designed in the 1920s at the workshop
of Warsaw type foundry ``Odlewnia Czcionek J. Idzkowski i S-ka''.
This sans serif typeface has a highly modulated stroke so it
has high typographic contrast. The vertical stems are much
heavier then horizontal ones. Most characters have thin
rectangles as additional counters giving the unique shape of
the characters. The lead types of Cyclop typeface were produced
in slanted variant at sizes 8-48 pt. It was heavily used for
heads in newspapers and accidents prints. Typesetters used
Cyclop in the inter-war period, % during the occupation in the
% underground press. The typeface was used until the beginnings
% of the offset print and computer typesetting era. Nowadays it
% is hard to find the metal types of this typeface. The font was
% generated using the Metatype1 package. Then the original set of
% characters was completed by adding the full set of accented
% letters and characters of the modern Latin alphabets (including
% Vietnamese). The upright variant was generated and it was more
% complicated task than it appeared at the beginning. 11 upright
% letters of the Cyclop typeface were presented in the book by
% Filip Trzaska, ``Podstawy techniki wydawniczej'' (``Foundation of
% the publishing techonology''), Warsaw 1967. But even the author
% of the book does not know what was the source of the presented
% examples. The fonts are distributed in the Type1 and OpenType
% formats along with the files necessary for use these fonts in
% TeX and {\LaTeX} including encoding definition files: T1 (ec), T5
% (Vietnamese), OT4, QX, texnansi and nonstandard ones (IL2 for
% Czech fonts).
\end{package}
% \begin{package}{ec-cyklopr-sc}{cyklop: Cyklop-Regular}{The Cyclop typeface.}
% The Cyclop typeface was designed in the 1920s at the workshop
% of Warsaw type foundry ``Odlewnia Czcionek J. Idzkowski i S-ka''.
% This sans serif typeface has a highly modulated stroke so it
% has high typographic contrast. The vertical stems are much
% heavier then horizontal ones. Most characters have thin
% rectangles as additional counters giving the unique shape of
% the characters. The lead types of Cyclop typeface were produced
% in slanted variant at sizes 8-48 pt. It was heavily used for
% heads in newspapers and accidents prints. Typesetters used
% Cyclop in the inter-war period, % during the occupation in the
% % underground press. The typeface was used until the beginnings
% % of the offset print and computer typesetting era. Nowadays it
% % is hard to find the metal types of this typeface. The font was
% % generated using the Metatype1 package. Then the original set of
% % characters was completed by adding the full set of accented
% % letters and characters of the modern Latin alphabets (including
% % Vietnamese). The upright variant was generated and it was more
% % complicated task than it appeared at the beginning. 11 upright
% % letters of the Cyclop typeface were presented in the book by
% % Filip Trzaska, ``Podstawy techniki wydawniczej'' (``Foundation of
% % the publishing techonology''), Warsaw 1967. But even the author
% % of the book does not know what was the source of the presented
% % examples. The fonts are distributed in the Type1 and OpenType
% % formats along with the files necessary for use these fonts in
% % TeX and {\LaTeX} including encoding definition files: T1 (ec), T5
% % (Vietnamese), OT4, QX, texnansi and nonstandard ones (IL2 for
% % Czech fonts).
% \end{package}

% \begin{package}{}{dancers}{Font for Conan Doyle's ``The Dancing Men''.}
% The (Sherlock Holmes) book contains a code which uses dancing
% men as glyphs. The alphabet as given is not complete, lacking
% f, j, k, q, u, w, x and z, so those letters in the font are not
% due to Conan Doyle. The code required word endings to be marked
% by the dancing man representing the last letter to be holding a
% flag: these are coded as A-Z.
% thaTiStOsaYsentenceSiNthEcodElooKlikEthiS. In some cases, the
% man has no arms, making it impossible for him to hold a flag.
% In these cases, he is wearing a flag on his hat in the
% 'character'. The font is distributed as MetaFont source.
% \end{package}

\begin{package}{DejaVuSans-Bold-tlf-t1--base}{dejavu}{{\LaTeX} support for the DejaVu fonts.}
The package contains {\LaTeX} support for the DejaVu fonts. They
are derived from the Vera fonts, but contain more characters
and styles. The fonts are included in the original TrueType
format and converted Type~1 format. The (currently) supported
encodings are: OT1, T1, IL2, TS1, T2*, X2, QX, and LGR. The
package doesn't (currently) support mathematics. More encodings
and/or features will come later.
\end{package}

% \begin{package}{DejaVuSans-BoldOblique-tlf-t1--base}{dejavu}{{\LaTeX} support for the DejaVu fonts.}
% The package contains {\LaTeX} support for the DejaVu fonts. They
% are derived from the Vera fonts, but contain more characters
% and styles. The fonts are included in the original TrueType
% format and converted Type~1 format. The (currently) supported
% encodings are: OT1, T1, IL2, TS1, T2*, X2, QX, and LGR. The
% package doesn't (currently) support mathematics. More encodings
% and/or features will come later.
% \end{package}

\begin{package}{DejaVuSans-ExtraLight-tlf-t1--base}{dejavu}{{\LaTeX} support for the DejaVu fonts.}
The package contains {\LaTeX} support for the DejaVu fonts. They
are derived from the Vera fonts, but contain more characters
and styles. The fonts are included in the original TrueType
format and converted Type~1 format. The (currently) supported
encodings are: OT1, T1, IL2, TS1, T2*, X2, QX, and LGR. The
package doesn't (currently) support mathematics. More encodings
and/or features will come later.
\end{package}

% \begin{package}{DejaVuSans-Oblique-tlf-t1--base}{dejavu}{{\LaTeX} support for the DejaVu fonts.}
% The package contains {\LaTeX} support for the DejaVu fonts. They
% are derived from the Vera fonts, but contain more characters
% and styles. The fonts are included in the original TrueType
% format and converted Type~1 format. The (currently) supported
% encodings are: OT1, T1, IL2, TS1, T2*, X2, QX, and LGR. The
% package doesn't (currently) support mathematics. More encodings
% and/or features will come later.
% \end{package}

\begin{package}{DejaVuSans-tlf-t1--base}{dejavu}{{\LaTeX} support for the DejaVu fonts.}
The package contains {\LaTeX} support for the DejaVu fonts. They
are derived from the Vera fonts, but contain more characters
and styles. The fonts are included in the original TrueType
format and converted Type~1 format. The (currently) supported
encodings are: OT1, T1, IL2, TS1, T2*, X2, QX, and LGR. The
package doesn't (currently) support mathematics. More encodings
and/or features will come later.
\end{package}

\begin{package}{DejaVuSansCondensed-Bold-tlf-t1--base}{dejavu}{{\LaTeX} support for the DejaVu fonts.}
The package contains {\LaTeX} support for the DejaVu fonts. They
are derived from the Vera fonts, but contain more characters
and styles. The fonts are included in the original TrueType
format and converted Type~1 format. The (currently) supported
encodings are: OT1, T1, IL2, TS1, T2*, X2, QX, and LGR. The
package doesn't (currently) support mathematics. More encodings
and/or features will come later.
\end{package}

% \begin{package}{DejaVuSansCondensed-BoldOblique-tlf-t1--base}{dejavu}{{\LaTeX} support for the DejaVu fonts.}
% The package contains {\LaTeX} support for the DejaVu fonts. They
% are derived from the Vera fonts, but contain more characters
% and styles. The fonts are included in the original TrueType
% format and converted Type~1 format. The (currently) supported
% encodings are: OT1, T1, IL2, TS1, T2*, X2, QX, and LGR. The
% package doesn't (currently) support mathematics. More encodings
% and/or features will come later.
% \end{package}

% \begin{package}{DejaVuSansCondensed-Oblique-tlf-t1--base}{dejavu}{{\LaTeX} support for the DejaVu fonts.}
% The package contains {\LaTeX} support for the DejaVu fonts. They
% are derived from the Vera fonts, but contain more characters
% and styles. The fonts are included in the original TrueType
% format and converted Type~1 format. The (currently) supported
% encodings are: OT1, T1, IL2, TS1, T2*, X2, QX, and LGR. The
% package doesn't (currently) support mathematics. More encodings
% and/or features will come later.
% \end{package}

\begin{package}{DejaVuSansCondensed-tlf-t1--base}{dejavu}{{\LaTeX} support for the DejaVu fonts.}
The package contains {\LaTeX} support for the DejaVu fonts. They
are derived from the Vera fonts, but contain more characters
and styles. The fonts are included in the original TrueType
format and converted Type~1 format. The (currently) supported
encodings are: OT1, T1, IL2, TS1, T2*, X2, QX, and LGR. The
package doesn't (currently) support mathematics. More encodings
and/or features will come later.
\end{package}

% \begin{package}{DejaVuSansMono-Bold-tlf-t1--base}{dejavu}{{\LaTeX} support for the DejaVu fonts.}
% The package contains {\LaTeX} support for the DejaVu fonts. They
% are derived from the Vera fonts, but contain more characters
% and styles. The fonts are included in the original TrueType
% format and converted Type~1 format. The (currently) supported
% encodings are: OT1, T1, IL2, TS1, T2*, X2, QX, and LGR. The
% package doesn't (currently) support mathematics. More encodings
% and/or features will come later.
% \end{package}

% \begin{package}{DejaVuSansMono-BoldOblique-tlf-t1--base}{dejavu}{{\LaTeX} support for the DejaVu fonts.}
% The package contains {\LaTeX} support for the DejaVu fonts. They
% are derived from the Vera fonts, but contain more characters
% and styles. The fonts are included in the original TrueType
% format and converted Type~1 format. The (currently) supported
% encodings are: OT1, T1, IL2, TS1, T2*, X2, QX, and LGR. The
% package doesn't (currently) support mathematics. More encodings
% and/or features will come later.
% \end{package}

% \begin{package}{DejaVuSansMono-Oblique-tlf-t1--base}{dejavu}{{\LaTeX} support for the DejaVu fonts.}
% The package contains {\LaTeX} support for the DejaVu fonts. They
% are derived from the Vera fonts, but contain more characters
% and styles. The fonts are included in the original TrueType
% format and converted Type~1 format. The (currently) supported
% encodings are: OT1, T1, IL2, TS1, T2*, X2, QX, and LGR. The
% package doesn't (currently) support mathematics. More encodings
% and/or features will come later.
% \end{package}

\begin{package}{DejaVuSansMono-tlf-t1--base}{dejavu}{{\LaTeX} support for the DejaVu fonts.}
The package contains {\LaTeX} support for the DejaVu fonts. They
are derived from the Vera fonts, but contain more characters
and styles. The fonts are included in the original TrueType
format and converted Type~1 format. The (currently) supported
encodings are: OT1, T1, IL2, TS1, T2*, X2, QX, and LGR. The
package doesn't (currently) support mathematics. More encodings
and/or features will come later.
\end{package}

\begin{package}{DejaVuSerif-Bold-tlf-t1--base}{dejavu}{{\LaTeX} support for the DejaVu fonts.}
The package contains {\LaTeX} support for the DejaVu fonts. They
are derived from the Vera fonts, but contain more characters
and styles. The fonts are included in the original TrueType
format and converted Type~1 format. The (currently) supported
encodings are: OT1, T1, IL2, TS1, T2*, X2, QX, and LGR. The
package doesn't (currently) support mathematics. More encodings
and/or features will come later.
\end{package}

% \begin{package}{DejaVuSerif-BoldItalic-tlf-t1--base}{dejavu}{{\LaTeX} support for the DejaVu fonts.}
% The package contains {\LaTeX} support for the DejaVu fonts. They
% are derived from the Vera fonts, but contain more characters
% and styles. The fonts are included in the original TrueType
% format and converted Type~1 format. The (currently) supported
% encodings are: OT1, T1, IL2, TS1, T2*, X2, QX, and LGR. The
% package doesn't (currently) support mathematics. More encodings
% and/or features will come later.
% \end{package}

\begin{package}{DejaVuSerif-Italic-tlf-t1--base}{dejavu}{{\LaTeX} support for the DejaVu fonts.}
The package contains {\LaTeX} support for the DejaVu fonts. They
are derived from the Vera fonts, but contain more characters
and styles. The fonts are included in the original TrueType
format and converted Type~1 format. The (currently) supported
encodings are: OT1, T1, IL2, TS1, T2*, X2, QX, and LGR. The
package doesn't (currently) support mathematics. More encodings
and/or features will come later.
\end{package}

\begin{package}{DejaVuSerif-tlf-t1--base}{dejavu}{{\LaTeX} support for the DejaVu fonts.}
The package contains {\LaTeX} support for the DejaVu fonts. They
are derived from the Vera fonts, but contain more characters
and styles. The fonts are included in the original TrueType
format and converted Type~1 format. The (currently) supported
encodings are: OT1, T1, IL2, TS1, T2*, X2, QX, and LGR. The
package doesn't (currently) support mathematics. More encodings
and/or features will come later.
\end{package}

\begin{package}{DejaVuSerifCondensed-Bold-tlf-t1--base}{dejavu}{{\LaTeX} support for the DejaVu fonts.}
The package contains {\LaTeX} support for the DejaVu fonts. They
are derived from the Vera fonts, but contain more characters
and styles. The fonts are included in the original TrueType
format and converted Type~1 format. The (currently) supported
encodings are: OT1, T1, IL2, TS1, T2*, X2, QX, and LGR. The
package doesn't (currently) support mathematics. More encodings
and/or features will come later.
\end{package}

% \begin{package}{DejaVuSerifCondensed-BoldItalic-tlf-t1--base}{dejavu}{{\LaTeX} support for the DejaVu fonts.}
% The package contains {\LaTeX} support for the DejaVu fonts. They
% are derived from the Vera fonts, but contain more characters
% and styles. The fonts are included in the original TrueType
% format and converted Type~1 format. The (currently) supported
% encodings are: OT1, T1, IL2, TS1, T2*, X2, QX, and LGR. The
% package doesn't (currently) support mathematics. More encodings
% and/or features will come later.
% \end{package}

\begin{package}{DejaVuSerifCondensed-Italic-tlf-t1--base}{dejavu}{{\LaTeX} support for the DejaVu fonts.}
The package contains {\LaTeX} support for the DejaVu fonts. They
are derived from the Vera fonts, but contain more characters
and styles. The fonts are included in the original TrueType
format and converted Type~1 format. The (currently) supported
encodings are: OT1, T1, IL2, TS1, T2*, X2, QX, and LGR. The
package doesn't (currently) support mathematics. More encodings
and/or features will come later.
\end{package}

\begin{package}{DejaVuSerifCondensed-tlf-t1--base}{dejavu}{{\LaTeX} support for the DejaVu fonts.}
The package contains {\LaTeX} support for the DejaVu fonts. They
are derived from the Vera fonts, but contain more characters
and styles. The fonts are included in the original TrueType
format and converted Type~1 format. The (currently) supported
encodings are: OT1, T1, IL2, TS1, T2*, X2, QX, and LGR. The
package doesn't (currently) support mathematics. More encodings
and/or features will come later.
\end{package}

% \begin{package}{}{dice}{A font for die faces.}
% A Metafont font that can produce die faces in 2D or with
% various 3D effects.
% \end{package}
% \begin{package}{dictsym}{dictsym}{DictSym font and macro package}
% This directory contains the DictSym Type1 font designed by
% Georg Verweyen and all files required to use it with {\LaTeX} on
% the Unix or PC platforms. The font provides a number of symbols
% commonly used in dictionaries. The accompanying macro package
% makes the symbols accessible as {\LaTeX} commands.
% \end{package}
% \begin{package}{}{dingbat}{Two dingbat symbol fonts.}
% The fonts (ark10 and dingbat) are specified in Metafont;
% support macros are provided for use in {\LaTeX}. An Adobe Type~1
% version of the fonts is available in the niceframe fonts
% bundle.
% \end{package}
% \begin{package}{}{doublestroke}{Typeset mathematical double stroke symbols.}
% A font based on Computer Modern Roman useful for typesetting
% the mathematical symbols for the natural numbers (N), whole
% numbers (Z), rational numbers (Q), real numbers (R) and complex
% numbers (C); coverage includes all Roman capital letters, '1',
% 'h' and 'k'. The font is available both as MetaFont source and
% in Adobe Type~1 format, and {\LaTeX} macros for its use are
% provided. The fonts appear in the blackboard bold sampler.
% \end{package}
% \begin{package}{}{dozenal}{Typeset documents using base twelve numbering (also called ``dozenal'')}
% The package supports typesetting documents whose counters are
% represented in base twelve, also called ``dozenal''. It includes
% a macro by David Kastrup for converting positive whole numbers
% to dozenal from decimal (base ten) representation. The package
% also includes a few other macros and redefines all the standard
% counters to produce dozenal output. Fonts, in Roman, italic,
% slanted, and boldface versions, provide ten and eleven (the
% Pitman characters preferred by the Dozenal Society of Great
% Britain). The fonts were designed to blend well with the
% Computer Modern fonts, and are available both as Metafont
% source and in Adobe Type~1 format.
% \end{package}

% \begin{package}{}{droid}{{\LaTeX} support for the Droid font families.}
% The Droid typeface family was designed in the fall of 2006 by
% Steve Matteson, as a commission from Google to create a set of
% system fonts for its Android platform. The goal was to provide
% optimal quality and comfort on a mobile handset when rendered
% in application menus, web browsers and for other screen text.
% The Droid family consists of Droid Serif, Droid Sans and Droid
% Sans Mono fonts, licensed under the Apache License Version~2.0.
% The bundle includes the fonts in both TrueType and Adobe Type~1
% formats. The package does not support the Droid Pro family of
% fonts, available for purchase from the Ascender foundry.
% \end{package}

% \begin{package}{}{duerer}{Computer Duerer fonts.}
% These fonts are designed for titling use, and consist of
% capital roman letters only. Together with the normal set of
% base shapes, the family also offers an informal shape. {\LaTeX}
% support is available in the duerer-latex bundle.
% \end{package}
% \begin{package}{}{duerer-latex}{{\LaTeX} support for the Duerer fonts.}
% {\LaTeX} support for Hoenig's Computer Duerer fonts, using their
% standard fontname names.
% \end{package}
% \begin{package}{rdutchcalb}{dutchcal}{A reworking of ESSTIX13, adding a bold version.}
% This package reworks the mathematical calligraphic font
% ESSTIX13, adding a bold version. {\LaTeX} support files are
% included. The new fonts may also be accessed from the most
% recent version of mathalfa. The fonts themselves are subject to
% the SIL OPEN FONT LICENSE, version~1.1.
% \end{package}
% \begin{package}{}{ean}{Macros for making EAN barcodes.}
% Provides EAN-8 and EAN-13 forms. The package needs the ocr-b
% fonts; note that the fonts are not available under a free
% licence, as the macros are.
% \end{package}

\begin{package}{EBGaramond12-Italic-tlf-t1--base}{ebgaramond: EBGaramond12-Italic}{{\LaTeX} support for EBGaramond fonts.}
EB Garamond is a revival by Georg Duffner of the 16th century
fonts designed by Claude Garamond. The {\LaTeX} support package
works for (pdf){\LaTeX}, {\XeLaTeX} and {\LuaLaTeX} users; configuration
files for use with microtype are provided.
\end{package}
\begin{package}{EBGaramond12-Regular-tlf-t1--base}{ebgaramond: EBGaramond12-Regular}{{\LaTeX} support for EBGaramond fonts.}
EB Garamond is a revival by Georg Duffner of the 16th century
fonts designed by Claude Garamond. The {\LaTeX} support package
works for (pdf){\LaTeX}, {\XeLaTeX} and {\LuaLaTeX} users; configuration
files for use with microtype are provided.
\end{package}
% \begin{package}{EBGaramondInitials-tlf-t1--base}{ebgaramond: EBGaramondInitials}{{\LaTeX} support for EBGaramond fonts.}
% EB Garamond is a revival by Georg Duffner of the 16th century
% fonts designed by Claude Garamond. The {\LaTeX} support package
% works for (pdf){\LaTeX}, {\XeLaTeX} and {\LuaLaTeX} users; configuration
% files for use with microtype are provided.
% \end{package}
\begin{package}{EBGaramond12-Regular-osf-t1--base}{ebgaramond: EBGaramond12-Regular}{{\LaTeX} support for EBGaramond fonts.}
EB Garamond is a revival by Georg Duffner of the 16th century
fonts designed by Claude Garamond. The {\LaTeX} support package
works for (pdf){\LaTeX}, {\XeLaTeX} and {\LuaLaTeX} users; configuration
files for use with microtype are provided.
\end{package}

% \begin{package}{}{ecc}{Sources for the European Concrete fonts.}
% The MetaFont sources and TFM files of the European Concrete
% Fonts. This is the T1-encoded extension of Knuth's Concrete
% fonts, including also the corresponding text companion fonts.
% Adobe Type~1 versions of the fonts are available as part of the
% cm-super font bundle.
% \end{package}
% \begin{package}{}{eco}{Oldstyle numerals using EC fonts.}
% A set of font metric files and virtual fonts for using the EC
% fonts with oldstyle numerals. These files can only be used
% together with the standard ec fonts. The style file eco.sty is
% sufficient to use the eco fonts but if you intend to use other
% font families as well, e.g., PostScript fonts, try altfont.
% \end{package}
% \begin{package}{}{eiad}{Traditional style Irish fonts.}
% In both lower and upper case 32 letters are defined (18 'plain'
% ones, 5 long vowels and 9 aspirated consonants). The ligature
% 'agus' is also made available. The remaining characters
% (digits, punctuation and accents) are inherited from the
% Computer Modern family of fonts. The font definitions use code
% from the sauter fonts, so those fonts have to be installed
% before using eiad. OT1*.fd files are provided for use with
% {\LaTeX}.
% \end{package}
% \begin{package}{}{eiad-ltx}{{\LaTeX} support for the eiad font.}
% Provides macros for use of the eiad fonts in OT1 encoding. Also
% offers a couple of MetaFont files described in the font
% package, but not provided there.
% \end{package}
% \begin{package}{yesr8r}{electrum}{Electrum ADF fonts collection.}
% Electrum ADF is a slab-serif font featuring optical and italic
% small-caps; additional ligatures and an alternate Q; lining,
% hanging, inferior and superior digits; and four weights. The
% fonts are provided in Adobe Type~1 format and the support
% material enables use with {\LaTeX}. Licence is mixed: LPPL for
% {\LaTeX} support; GPL with font exception for the fonts.
% \end{package}
% \begin{package}{}{elvish}{Fonts for typesetting Tolkien Elvish scripts.}
% The bundle provides fonts for Cirth (cirth.mf, etc.) and for
% Tengwar (teng10.mf). The Tengwar fonts are supported by macros
% in teng.tex, or by the (better documented) tengtex package.
% \end{package}
\begin{package}{epigraficab8a}{epigrafica}{A Greek and Latin font.}
Epigrafica is forked from the development of the MgOpen font
Cosmetica, which is a similar design to Optima and includes
Greek. Development has been supported by the Laboratory of
Digital Typography and Mathematical Software, of the Department
of Mathematics of the University of the Aegean, Greece.
\end{package}
% \begin{package}{}{epsdice}{A scalable dice ``font''.}
% The epsdice package defines a single command epsdice that
% takes a numeric argument (in the range 1-6), and selects a face
% image from a file that contains each of the 6 possible die
% faces. The graphic file is provided in both Encapsulated
% PostScript and PDF formats.
% \end{package}

\begin{package}{rESSTIX13}{esstix}{PostScript versions of the ESSTIX, with macro support.}
These fonts represent translation to PostScript Type~1 of the
ESSTIX fonts. ESSTIX seem to have been a precursor to the STIX
project, and were donated by Elsevier to that project. The
accompanying virtual fonts with customized metrics and {\LaTeX}
support files allow their use as calligraphic, fraktur and
double-struck (blackboard bold) in maths mode.
\end{package}

% \begin{package}{}{esvect}{Vector arrows.}
% Write vectors using an arrow which is different to the Computer
% Modern one. You have the choice between several kinds of
% arrows. The package consists of the relevant Metafont code and
% a package to use it.
% \end{package}
\begin{package}{zeurm10}{eulervm}{Euler virtual math fonts.}
The well-known Euler fonts are suitable for typsetting
mathematics in conjunction with a variety of text fonts which
do not provide mathematical character sets of their own. Euler-%
VM is a set of virtual mathematics fonts based on Euler and CM.
This approach has several advantages over immediately using the
real Euler fonts: Most noticeably, less {\TeX} resources are
consumed, the quality of various math symbols is improved and a
usable hslash symbol can be provided. The virtual fonts are
accompanied by a {\LaTeX} package which makes them easy to use,
particularly in conjunction with Type1 PostScript text fonts.
They are compatible with amsmath. A package option allows the
fonts to be loaded at 95{\%} of their nominal size, thus blending
better with certain text fonts, e.g., Minion.
\end{package}
% \begin{package}{}{euxm}{}
% \end{package}

\begin{package}{fbb-Bold-lf-t1--base}{fbb}{A free Bembo-like font.}
The package provides a Bembo-like font package based on Cardo
but with many modifications, adding Bold Italic, small caps in
all styles, six figure choices in all styles, updated kerning
tables, added figure tables and corrected f-ligatures. Both
OpenType and Adobe Type~1 versions are provided; all necessary
support files are provided. The font works well with
newtxmath's libertine option.
\end{package}
% \begin{package}{fbb-BoldItalic-lf-t1--base}{fbb}{A free Bembo-like font.}
% The package provides a Bembo-like font package based on Cardo
% but with many modifications, adding Bold Italic, small caps in
% all styles, six figure choices in all styles, updated kerning
% tables, added figure tables and corrected f-ligatures. Both
% OpenType and Adobe Type~1 versions are provided; all necessary
% support files are provided. The font works well with
% newtxmath's libertine option.
% \end{package}
\begin{package}{fbb-Italic-lf-t1--base}{fbb}{A free Bembo-like font.}
The package provides a Bembo-like font package based on Cardo
but with many modifications, adding Bold Italic, small caps in
all styles, six figure choices in all styles, updated kerning
tables, added figure tables and corrected f-ligatures. Both
OpenType and Adobe Type~1 versions are provided; all necessary
support files are provided. The font works well with
newtxmath's libertine option.
\end{package}
\begin{package}{fbb-Regular-lf-t1--base}{fbb}{A free Bembo-like font.}
The package provides a Bembo-like font package based on Cardo
but with many modifications, adding Bold Italic, small caps in
all styles, six figure choices in all styles, updated kerning
tables, added figure tables and corrected f-ligatures. Both
OpenType and Adobe Type~1 versions are provided; all necessary
support files are provided. The font works well with
newtxmath's libertine option.
\end{package}

\begin{package}{FdSymbolC-Medium}{fdsymbol}{A maths symbol font.}
FdSymbol is a maths symbol font, designed as a companion to the
Fedra family by Typotheque, but it might also fit other
contemporary typefaces.
\end{package}
% \begin{package}{}{feyn}{A font for in-text Feynman diagrams.}
% Feyn may be used to produce relatively simple Feynman diagrams
% within equations in a {\LaTeX} document. While the feynmf package
% is good at drawing large diagrams for figures, the present
% package and its fonts allow diagrams within equations or text,
% at a matching size. The fonts are distributed as MetaFont
% source, and macros for their use are also provided.
% \end{package}
% \begin{package}{fgerm10}{fge}{A font for Frege's Grundgesetze der Arithmetik.}
% The fonts are provided as Metafont source and Adobe Type~1
% (pfb) files. A small {\LaTeX} package (fge) is included.
% \end{package}
\begin{package}{foekfont}{FOEKFONT}{\uppercase{The title font of the mads fok magazine.}}
\uppercase{The bundle provides an adobe type~1 font, and {\LaTeX} support for
its use. The magazine web site shows the font in use in a few
places.}
\end{package}
% \begin{package}{fonetika}{fonetika}{Support for the danish ``Dania'' phonetic system.}
% Fonetika Dania is a font bundle with a serif font and a sans
% serif font for the danish phonetic system Dania. Both fonts
% exist in regular and bold weights. {\LaTeX} support is provided.
% The fonts are based on URW Palladio and Iwona Condensed, and
% were created using FontForge.
% \end{package}
% \begin{package}{}{fontawesome}{Font containing web-related icons.}
% The package offers access to the large number of web-related
% icons provided by the included font. The package requires the
% package, fontspec, running under {\XeTeX} or LuaTeX.
% \end{package}

% \begin{package}{futr8r}{fourier}{Using Utopia fonts in {\LaTeX} documents.}
% Fourier-GUTenberg is a {\LaTeX} typesetting system which uses
% Adobe Utopia as its standard base font. Fourier-GUTenberg
% provides all complementary typefaces needed to allow Utopia
% based {\TeX} typesetting, including an extensive mathematics set
% and several other symbols. The system is absolutely stand-%
% alone: apart from Utopia and Fourier, no other typefaces are
% required. The fourier fonts will also work with Adobe Utopia
% Expert fonts, which are only available for purchase. Utopia is
% a registered trademark of Adobe Systems Incorporated
% \end{package}

% \begin{package}{}{fouriernc}{Use New Century Schoolbook text with Fourier maths fonts.}
% This package provides a {\LaTeX} mathematics font setup for use
% with New Century Schoolbook text. In order to use it you need
% to have the Fourier-GUTenberg fonts installed.
% \end{package}

\begin{package}{frca10}{frcursive}{French cursive hand fonts.}
A hand-writing font in the style of the French academic
running-hand. The font was written in Metafont and and has been
converted to Adobe Type~1 format. {\LaTeX} support (NFFS fd files,
and a package) and font maps are provided.
\end{package}
\begin{package}{frcbx14}{frcursive}{French cursive hand fonts.}
A hand-writing font in the style of the French academic
running-hand. The font was written in Metafont and and has been
converted to Adobe Type~1 format. {\LaTeX} support (NFFS fd files,
and a package) and font maps are provided.
\end{package}
\begin{package}{frcc14}{frcursive}{French cursive hand fonts.}
A hand-writing font in the style of the French academic
running-hand. The font was written in Metafont and and has been
converted to Adobe Type~1 format. {\LaTeX} support (NFFS fd files,
and a package) and font maps are provided.
\end{package}
\begin{package}{frcf14}{frcursive}{French cursive hand fonts.}
A hand-writing font in the style of the French academic
running-hand. The font was written in Metafont and and has been
converted to Adobe Type~1 format. {\LaTeX} support (NFFS fd files,
and a package) and font maps are provided.
\end{package}
\begin{package}{frcr14}{frcursive}{French cursive hand fonts.}
A hand-writing font in the style of the French academic
running-hand. The font was written in Metafont and and has been
converted to Adobe Type~1 format. {\LaTeX} support (NFFS fd files,
and a package) and font maps are provided.
\end{package}
\begin{package}{frcsl14}{frcursive}{French cursive hand fonts.}
A hand-writing font in the style of the French academic
running-hand. The font was written in Metafont and and has been
converted to Adobe Type~1 format. {\LaTeX} support (NFFS fd files,
and a package) and font maps are provided.
\end{package}
\begin{package}{frcslbx14}{frcursive}{French cursive hand fonts.}
A hand-writing font in the style of the French academic
running-hand. The font was written in Metafont and and has been
converted to Adobe Type~1 format. {\LaTeX} support (NFFS fd files,
and a package) and font maps are provided.
\end{package}
\begin{package}{frcslc14}{frcursive}{French cursive hand fonts.}
A hand-writing font in the style of the French academic
running-hand. The font was written in Metafont and and has been
converted to Adobe Type~1 format. {\LaTeX} support (NFFS fd files,
and a package) and font maps are provided.
\end{package}
\begin{package}{frcw10}{frcursive}{French cursive hand fonts.}
A hand-writing font in the style of the French academic
running-hand. The font was written in Metafont and and has been
converted to Adobe Type~1 format. {\LaTeX} support (NFFS fd files,
and a package) and font maps are provided.
\end{package}

% \begin{package}{}{genealogy}{A compilation genealogy font.}
% A simple compilation of the genealogical symbols found in the
% wasy and gen fonts, essentially adding the male and female
% symbols to Knuth's 'gen' font, and so avoiding loading two
% fonts when you need only genealogical symbols. The font is
% distributed as Metafont source.
% \end{package}

\begin{package}{ec-gentiumbasic-bold}{gentium-tug}{Gentium fonts (in two formats) and support files.}
Gentium is a typeface family designed to enable the diverse
ethnic groups around the world who use the Latin, Cyrillic and
Greek scripts to produce readable, high-quality publications.
It supports a wide range of Latin- and Cyrillic-based
alphabets. The package consists of: - The original (unaltered)
GentiumPlus, GentiumBook, and other Gentium-family fonts in
TrueType format, as developed by SIL and released under the OFL
(see OFL.txt and OFL-FAQ.txt); - Converted fonts in PostScript
Type~1 format, released under the same terms. These incorporate
the name ``Gentium'' by permission of SIL given to the {\TeX} Users
Group; - ConTeXt, {\LaTeX} and other supporting files; - {\TeX}-
related documentation, and the SIL documentation and other
files.
\end{package}
% \begin{package}{ec-gentiumbasic-bolditalic}{gentium-tug}{Gentium fonts (in two formats) and support files.}
% Gentium is a typeface family designed to enable the diverse
% ethnic groups around the world who use the Latin, Cyrillic and
% Greek scripts to produce readable, high-quality publications.
% It supports a wide range of Latin- and Cyrillic-based
% alphabets. The package consists of: - The original (unaltered)
% GentiumPlus, GentiumBook, and other Gentium-family fonts in
% TrueType format, as developed by SIL and released under the OFL
% (see OFL.txt and OFL-FAQ.txt); - Converted fonts in PostScript
% Type~1 format, released under the same terms. These incorporate
% the name ``Gentium'' by permission of SIL given to the {\TeX} Users
% Group; - ConTeXt, {\LaTeX} and other supporting files; - {\TeX}-
% related documentation, and the SIL documentation and other
% files.
% \end{package}
\begin{package}{ec-gentiumplus-italic}{gentium-tug}{Gentium fonts (in two formats) and support files.}
Gentium is a typeface family designed to enable the diverse
ethnic groups around the world who use the Latin, Cyrillic and
Greek scripts to produce readable, high-quality publications.
It supports a wide range of Latin- and Cyrillic-based
alphabets. The package consists of: - The original (unaltered)
GentiumPlus, GentiumBook, and other Gentium-family fonts in
TrueType format, as developed by SIL and released under the OFL
(see OFL.txt and OFL-FAQ.txt); - Converted fonts in PostScript
Type~1 format, released under the same terms. These incorporate
the name ``Gentium'' by permission of SIL given to the {\TeX} Users
Group; - ConTeXt, {\LaTeX} and other supporting files; - {\TeX}-
related documentation, and the SIL documentation and other
files.
\end{package}
\begin{package}{ec-gentiumplus-regular}{gentium-tug}{Gentium fonts (in two formats) and support files.}
Gentium is a typeface family designed to enable the diverse
ethnic groups around the world who use the Latin, Cyrillic and
Greek scripts to produce readable, high-quality publications.
It supports a wide range of Latin- and Cyrillic-based
alphabets. The package consists of: - The original (unaltered)
GentiumPlus, GentiumBook, and other Gentium-family fonts in
TrueType format, as developed by SIL and released under the OFL
(see OFL.txt and OFL-FAQ.txt); - Converted fonts in PostScript
Type~1 format, released under the same terms. These incorporate
the name ``Gentium'' by permission of SIL given to the {\TeX} Users
Group; - ConTeXt, {\LaTeX} and other supporting files; - {\TeX}-
related documentation, and the SIL documentation and other
files.
\end{package}
\begin{package}{ec-gentiumplus-regular-sc}{gentium-tug}{Gentium fonts (in two formats) and support files.}
Gentium is a typeface family designed to enable the diverse
ethnic groups around the world who use the Latin, Cyrillic and
Greek scripts to produce readable, high-quality publications.
It supports a wide range of Latin- and Cyrillic-based
alphabets. The package consists of: - The original (unaltered)
GentiumPlus, GentiumBook, and other Gentium-family fonts in
TrueType format, as developed by SIL and released under the OFL
(see OFL.txt and OFL-FAQ.txt); - Converted fonts in PostScript
Type~1 format, released under the same terms. These incorporate
the name ``Gentium'' by permission of SIL given to the {\TeX} Users
Group; - ConTeXt, {\LaTeX} and other supporting files; - {\TeX}-
related documentation, and the SIL documentation and other
files.
\end{package}

\begin{package}{artemisiab8r}{gfsartemisia: GFSArtemisia-Bold}{A modern Greek font design.}
GFS Artemisia is a relatively modern font, designed as a
'general purpose' font in the same sense as Times is nowadays
treated. The present version has been provided by the Greek
Font Society. The font supports the Greek and Latin alphabets.
{\LaTeX} support is provided, using the OT1, T1 and LGR encodings.
\end{package}
% \begin{package}{artemisiabi8r}{gfsartemisia: GFSArtemisia-BoldItalic}{A modern Greek font design.}
% GFS Artemisia is a relatively modern font, designed as a
% 'general purpose' font in the same sense as Times is nowadays
% treated. The present version has been provided by the Greek
% Font Society. The font supports the Greek and Latin alphabets.
% {\LaTeX} support is provided, using the OT1, T1 and LGR encodings.
% \end{package}
\begin{package}{artemisiai8r}{gfsartemisia}{A modern Greek font design.}
GFS Artemisia is a relatively modern font, designed as a
'general purpose' font in the same sense as Times is nowadays
treated. The present version has been provided by the Greek
Font Society. The font supports the Greek and Latin alphabets.
{\LaTeX} support is provided, using the OT1, T1 and LGR encodings.
\end{package}
\begin{package}{artemisiarg8r}{gfsartemisia}{A modern Greek font design.}
GFS Artemisia is a relatively modern font, designed as a
'general purpose' font in the same sense as Times is nowadays
treated. The present version has been provided by the Greek
Font Society. The font supports the Greek and Latin alphabets.
{\LaTeX} support is provided, using the OT1, T1 and LGR encodings.
\end{package}

\begin{package}{bodonib8r}{gfsbodoni}{A Greek and Latin font based on Bodoni.}
Bodoni's Greek fonts in the 18th century broke, for the first
time, with the Byzantine cursive tradition of Greek fonts. GFS
Bodoni resurrects his work for general use. The font family
supports both Greek and Latin letters. {\LaTeX} support of the
fonts is provided, offering OT1, T1 and LGR encodings. The
fonts themselves are provided in Adobe Type~1 and OpenType
formats.
\end{package}

% \begin{package}{bodonibi8r}{gfsbodoni}{A Greek and Latin font based on Bodoni.}
% Bodoni's Greek fonts in the 18th century broke, for the first
% time, with the Byzantine cursive tradition of Greek fonts. GFS
% Bodoni resurrects his work for general use. The font family
% supports both Greek and Latin letters. {\LaTeX} support of the
% fonts is provided, offering OT1, T1 and LGR encodings. The
% fonts themselves are provided in Adobe Type~1 and OpenType
% formats.
% \end{package}

% \begin{package}{bodonibo8r}{gfsbodoni}{A Greek and Latin font based on Bodoni.}
% Bodoni's Greek fonts in the 18th century broke, for the first
% time, with the Byzantine cursive tradition of Greek fonts. GFS
% Bodoni resurrects his work for general use. The font family
% supports both Greek and Latin letters. {\LaTeX} support of the
% fonts is provided, offering OT1, T1 and LGR encodings. The
% fonts themselves are provided in Adobe Type~1 and OpenType
% formats.
% \end{package}

\begin{package}{bodonii8r}{gfsbodoni}{A Greek and Latin font based on Bodoni.}
Bodoni's Greek fonts in the 18th century broke, for the first
time, with the Byzantine cursive tradition of Greek fonts. GFS
Bodoni resurrects his work for general use. The font family
supports both Greek and Latin letters. {\LaTeX} support of the
fonts is provided, offering OT1, T1 and LGR encodings. The
fonts themselves are provided in Adobe Type~1 and OpenType
formats.
\end{package}

% \begin{package}{bodonio8r}{gfsbodoni}{A Greek and Latin font based on Bodoni.}
% Bodoni's Greek fonts in the 18th century broke, for the first
% time, with the Byzantine cursive tradition of Greek fonts. GFS
% Bodoni resurrects his work for general use. The font family
% supports both Greek and Latin letters. {\LaTeX} support of the
% fonts is provided, offering OT1, T1 and LGR encodings. The
% fonts themselves are provided in Adobe Type~1 and OpenType
% formats.
% \end{package}

\begin{package}{bodonirg8r}{gfsbodoni}{A Greek and Latin font based on Bodoni.}
Bodoni's Greek fonts in the 18th century broke, for the first
time, with the Byzantine cursive tradition of Greek fonts. GFS
Bodoni resurrects his work for general use. The font family
supports both Greek and Latin letters. {\LaTeX} support of the
fonts is provided, offering OT1, T1 and LGR encodings. The
fonts themselves are provided in Adobe Type~1 and OpenType
formats.
\end{package}

\begin{package}{bodonisc8r}{gfsbodoni}{A Greek and Latin font based on Bodoni.}
Bodoni's Greek fonts in the 18th century broke, for the first
time, with the Byzantine cursive tradition of Greek fonts. GFS
Bodoni resurrects his work for general use. The font family
supports both Greek and Latin letters. {\LaTeX} support of the
fonts is provided, offering OT1, T1 and LGR encodings. The
fonts themselves are provided in Adobe Type~1 and OpenType
formats.
\end{package}

% \begin{package}{bodonisco8r}{gfsbodoni}{A Greek and Latin font based on Bodoni.}
% Bodoni's Greek fonts in the 18th century broke, for the first
% time, with the Byzantine cursive tradition of Greek fonts. GFS
% Bodoni resurrects his work for general use. The font family
% supports both Greek and Latin letters. {\LaTeX} support of the
% fonts is provided, offering OT1, T1 and LGR encodings. The
% fonts themselves are provided in Adobe Type~1 and OpenType
% formats.
% \end{package}

\begin{package}{gcomplutum8r}{gfscomplutum}{A Greek font with a long history.}
GFS Complutum derives, via a long development, from a
minuscule-only font cut in the 16th century. An unsatisfactory
set of majuscules were added in the early 20th century, but its
author died before he could complete the revival of the font.
The Greek Font Society has released this version, which has a
new set of majuscules.
\end{package}

\begin{package}{didotrg8r}{gfsdidot}{A Greek font based on Didot's work.}
The design of Didot's 1805 Greek typeface was influenced by the
neoclassical ideals of the late 18th century. The font was
brought to Greece at the time of the 1821 Greek Revolution, by
Didot's son, and was very widely used. The present version is
provided by the Greek Font Society. The font supports the Greek
alphabet, and is accompanied by a matching Latin alphabet based
on Zapf's Palatino. {\LaTeX} support is provided, using the OT1,
T1 and LGR encodings.
\end{package}

\begin{package}{neohellenicrg8r}{gfsneohellenic}{A Greek font in the Neo-Hellenic style.}
The NeoHellenic style evolved in academic circles in the 19th
and 20th century; the present font follows a cut commissioned
from Monotype in 1927. The present version was provided by the
Greek Font Society. The font supports both Greek and Latin
characters, and has been adjusted to work well with the
cmbright fonts for mathematics support. {\LaTeX} support of the
fonts is provided, offering OT1, T1 and LGR encodings.
\end{package}
\begin{package}{gsolomos8r}{gfssolomos}{A Greek-alphabet font.}
Solomos is a font which traces its descent from a
calligraphically-inspired font of the mid-19th century. {\LaTeX}
support, for use with the LGR encoding only, is provided.
\end{package}
% \begin{package}{}{gillcm}{Alternative unslanted italic Computer Modern fonts.}
% This is a demonstration of the use of virtual fonts for unusual
% effects: the package implements an old idea of Eric Gill. The
% package was written for the author's talk at TUG 2010.
% \end{package}

\begin{package}{GilliusADF-Bold-lf-t1--base}{gillius}{Gillius fonts with {\LaTeX} support.}
This package provides {\LaTeX}, {\pdfLaTeX}, {\XeLaTeX} and {\LuaLaTeX}
support for the Gillius and Gillius No.2 families of sans
serif fonts and condensed versions of them, designed by Hirwen
Harendal. According to the designer, the fonts were inspired by
Gill Sans.
\end{package}

% \begin{package}{GilliusADF-BoldItalic-lf-t1--base}{gillius}{Gillius fonts with {\LaTeX} support.}
% This package provides {\LaTeX}, {\pdfLaTeX}, {\XeLaTeX} and {\LuaLaTeX}
% support for the Gillius and Gillius No.2 families of sans
% serif fonts and condensed versions of them, designed by Hirwen
% Harendal. According to the designer, the fonts were inspired by
% Gill Sans.
% \end{package}

\begin{package}{GilliusADF-Italic-lf-t1--base}{gillius}{Gillius fonts with {\LaTeX} support.}
This package provides {\LaTeX}, {\pdfLaTeX}, {\XeLaTeX} and {\LuaLaTeX}
support for the Gillius and Gillius No.2 families of sans
serif fonts and condensed versions of them, designed by Hirwen
Harendal. According to the designer, the fonts were inspired by
Gill Sans.
\end{package}

\begin{package}{GilliusADF-Regular-lf-t1--base}{gillius}{Gillius fonts with {\LaTeX} support.}
This package provides {\LaTeX}, {\pdfLaTeX}, {\XeLaTeX} and {\LuaLaTeX}
support for the Gillius and Gillius No.2 families of sans
serif fonts and condensed versions of them, designed by Hirwen
Harendal. According to the designer, the fonts were inspired by
Gill Sans.
\end{package}

\begin{package}{GilliusADFCond-Bold-lf-t1--base}{gillius}{Gillius fonts with {\LaTeX} support.}
This package provides {\LaTeX}, {\pdfLaTeX}, {\XeLaTeX} and {\LuaLaTeX}
support for the Gillius and Gillius No.2 families of sans
serif fonts and condensed versions of them, designed by Hirwen
Harendal. According to the designer, the fonts were inspired by
Gill Sans.
\end{package}

% \begin{package}{GilliusADFCond-BoldItalic-lf-t1--base}{gillius}{Gillius fonts with {\LaTeX} support.}
% This package provides {\LaTeX}, {\pdfLaTeX}, {\XeLaTeX} and {\LuaLaTeX}
% support for the Gillius and Gillius No.2 families of sans
% serif fonts and condensed versions of them, designed by Hirwen
% Harendal. According to the designer, the fonts were inspired by
% Gill Sans.
% \end{package}

\begin{package}{GilliusADFCond-Italic-lf-t1--base}{gillius}{Gillius fonts with {\LaTeX} support.}
This package provides {\LaTeX}, {\pdfLaTeX}, {\XeLaTeX} and {\LuaLaTeX}
support for the Gillius and Gillius No.2 families of sans
serif fonts and condensed versions of them, designed by Hirwen
Harendal. According to the designer, the fonts were inspired by
Gill Sans.
\end{package}

\begin{package}{GilliusADFCond-Regular-lf-t1--base}{gillius}{Gillius fonts with {\LaTeX} support.}
This package provides {\LaTeX}, {\pdfLaTeX}, {\XeLaTeX} and {\LuaLaTeX}
support for the Gillius and Gillius No.2 families of sans
serif fonts and condensed versions of them, designed by Hirwen
Harendal. According to the designer, the fonts were inspired by
Gill Sans.
\end{package}

\begin{package}{GilliusADFNo2-Bold-lf-t1--base}{gillius}{Gillius fonts with {\LaTeX} support.}
This package provides {\LaTeX}, {\pdfLaTeX}, {\XeLaTeX} and {\LuaLaTeX}
support for the Gillius and Gillius No.2 families of sans
serif fonts and condensed versions of them, designed by Hirwen
Harendal. According to the designer, the fonts were inspired by
Gill Sans.
\end{package}

% \begin{package}{GilliusADFNo2-BoldItalic-lf-t1--base}{gillius}{Gillius fonts with {\LaTeX} support.}
% This package provides {\LaTeX}, {\pdfLaTeX}, {\XeLaTeX} and {\LuaLaTeX}
% support for the Gillius and Gillius No.2 families of sans
% serif fonts and condensed versions of them, designed by Hirwen
% Harendal. According to the designer, the fonts were inspired by
% Gill Sans.
% \end{package}

\begin{package}{GilliusADFNo2-Italic-lf-t1--base}{gillius}{Gillius fonts with {\LaTeX} support.}
This package provides {\LaTeX}, {\pdfLaTeX}, {\XeLaTeX} and {\LuaLaTeX}
support for the Gillius and Gillius No.2 families of sans
serif fonts and condensed versions of them, designed by Hirwen
Harendal. According to the designer, the fonts were inspired by
Gill Sans.
\end{package}

\begin{package}{GilliusADFNo2-Regular-lf-t1--base}{gillius}{Gillius fonts with {\LaTeX} support.}
This package provides {\LaTeX}, {\pdfLaTeX}, {\XeLaTeX} and {\LuaLaTeX}
support for the Gillius and Gillius No.2 families of sans
serif fonts and condensed versions of them, designed by Hirwen
Harendal. According to the designer, the fonts were inspired by
Gill Sans.
\end{package}

\begin{package}{GilliusADFNo2Cond-Bold-lf-t1--base}{gillius}{Gillius fonts with {\LaTeX} support.}
This package provides {\LaTeX}, {\pdfLaTeX}, {\XeLaTeX} and {\LuaLaTeX}
support for the Gillius and Gillius No.2 families of sans
serif fonts and condensed versions of them, designed by Hirwen
Harendal. According to the designer, the fonts were inspired by
Gill Sans.
\end{package}

% \begin{package}{GilliusADFNo2Cond-BoldItalic-lf-t1--base}{gillius}{Gillius fonts with {\LaTeX} support.}
% This package provides {\LaTeX}, {\pdfLaTeX}, {\XeLaTeX} and {\LuaLaTeX}
% support for the Gillius and Gillius No.2 families of sans
% serif fonts and condensed versions of them, designed by Hirwen
% Harendal. According to the designer, the fonts were inspired by
% Gill Sans.
% \end{package}

\begin{package}{GilliusADFNo2Cond-Italic-lf-t1--base}{gillius}{Gillius fonts with {\LaTeX} support.}
This package provides {\LaTeX}, {\pdfLaTeX}, {\XeLaTeX} and {\LuaLaTeX}
support for the Gillius and Gillius No.2 families of sans
serif fonts and condensed versions of them, designed by Hirwen
Harendal. According to the designer, the fonts were inspired by
Gill Sans.
\end{package}

\begin{package}{GilliusADFNo2Cond-Regular-lf-t1--base}{gillius}{Gillius fonts with {\LaTeX} support.}
This package provides {\LaTeX}, {\pdfLaTeX}, {\XeLaTeX} and {\LuaLaTeX}
support for the Gillius and Gillius No.2 families of sans
serif fonts and condensed versions of them, designed by Hirwen
Harendal. According to the designer, the fonts were inspired by
Gill Sans.
\end{package}

% \begin{package}{}{gnu-freefont}{A Unicode font, with rather wide coverage.}
% The package provides a set of outline (i.e. OpenType) fonts
% covering as much as possible of the Unicode character set. The
% set consists of three typefaces: one monospaced and two
% proportional (one with uniform and one with modulated stroke).
% \end{package}

\begin{package}{yfrak}{gothic}{A collection of old German-style fonts.}
A collection of fonts that reproduce those used in ``old German''
printing. The set comprises Gothic, Schwabacher and Fraktur
fonts, a pair of handwriting fonts, Suetterlin and Schwell, and
a font containing decorative initials. In addition, there are
two re-encoding packages for Haralambous's fonts, providing T1,
using virtual fonts, and OT1 and T1, using Metafont.
\end{package}
\begin{package}{ygoth}{gothic}{A collection of old German-style fonts.}
A collection of fonts that reproduce those used in ``old German''
printing. The set comprises Gothic, Schwabacher and Fraktur
fonts, a pair of handwriting fonts, Suetterlin and Schwell, and
a font containing decorative initials. In addition, there are
two re-encoding packages for Haralambous's fonts, providing T1,
using virtual fonts, and OT1 and T1, using Metafont.
\end{package}
\begin{package}{yswab}{gothic}{A collection of old German-style fonts.}
A collection of fonts that reproduce those used in ``old German''
printing. The set comprises Gothic, Schwabacher and Fraktur
fonts, a pair of handwriting fonts, Suetterlin and Schwell, and
a font containing decorative initials. In addition, there are
two re-encoding packages for Haralambous's fonts, providing T1,
using virtual fonts, and OT1 and T1, using Metafont.
\end{package}

% \begin{package}{}{greenpoint}{The Green Point logo.}
% A MetaFont-implementation of the logo commonly known as 'Der
% Grune Punkt' ('The Green Point'). In Austria, it can be found
% on nearly every bottle. It should not be confused with the
% 'Recycle'-logo, implemented by Ian Green.
% \end{package}
\begin{package}{ugqb8r}{grotesq}{URW Grotesq font pack for {\LaTeX}.}
The directory contains a copy of the Type~1 font ``URW Grotesq
2031 Bold'' released under the GPL by URW, with supporting files
for use with (La){\TeX}.
\end{package}
\begin{package}{rinje}{hacm}{Font support for the Arka language.}
The package supports typesetting hacm, the alphabet of the
constructed language Arka. The bundle provides nine official
fonts, in Adobe Type~1 format.
\end{package}
% \begin{package}{}{hands}{Pointing hand font.}
% Provides right- and left-pointing hands in both black-on-white
% and white-on-black realisation. The font is distributed as
% MetaFont source.
% \end{package}

\begin{package}{cmbr17}{hfbright}{The hfbright fonts.}
These are Adobe Type~1 versions of the OT1-encoded and maths
parts of the Computer Modern Bright fonts.
\end{package}

\begin{package}{hfocc1000}{hfoldsty}{Old style numerals with EC fonts.}
The hfoldsty package provides virtual fonts for using oldstyle
(0123456789) figures with the European Computer Modern fonts.
It does a similar job as the eco package by Sebastian Kirsch
but includes a couple of improvements, i.e., better kerning
with guillemets, and support for character protruding using the
pdfcprot package.
\end{package}

\begin{package}{fibr84}{ibygrk}{Fonts and macros to typeset ancient Greek.}
Ibycus is a Greek typeface, based on Silvio Levy's realisation
of a classic Didot cut of Greek type from around 1800. The
fonts are available both as MetaFont source and in Adobe Type 1
format. This distribution of ibycus is accompanied by a set of
macro packages to use it with Plain TeX or LaTeX, but for use
with Babel, see the ibycus-babel package.
\end{package}

% \begin{package}{}{ifsym}{A collection of symbols.}
% A set of symbol fonts, written in Metafont, offering
% (respectively) clock-face symbols, geometrical symbols, weather
% symbols, mountaineering symbols, electronic circuit symbols and
% a set of miscellaneous symbols. A {\LaTeX} package is provided,
% that allows the user to load only those symbols needed in a
% document.
% \end{package}

% \begin{package}{t1-zi4b-0}{inconsolata}{A monospaced font, with support files for use with {\TeX}.}
% Inconsolata is a monospaced font designed by Raph Levien. This
% package contains the font (in both Adobe Type~1 and OpenType
% formats) in regular and bold weights, with additional glyphs
% and options to control slashed zero, upright quotes and a
% shapelier lower-case L, plus metric files for use with {\TeX}, and
% {\LaTeX} font definition and other relevant files.
% \end{package}
\begin{package}{t1-zi4r-0}{inconsolata}{A monospaced font, with support files for use with {\TeX}.}
Inconsolata is a monospaced font designed by Raph Levien. This
package contains the font (in both Adobe Type~1 and OpenType
formats) in regular and bold weights, with additional glyphs
and options to control slashed zero, upright quotes and a
shapelier lower-case L, plus metric files for use with {\TeX}, and
{\LaTeX} font definition and other relevant files.
\end{package}
\begin{package}{t1-zi4r-1}{inconsolata}{A monospaced font, with support files for use with {\TeX}.}
Inconsolata is a monospaced font designed by Raph Levien. This
package contains the font (in both Adobe Type~1 and OpenType
formats) in regular and bold weights, with additional glyphs
and options to control slashed zero, upright quotes and a
shapelier lower-case L, plus metric files for use with {\TeX}, and
{\LaTeX} font definition and other relevant files.
\end{package}

\begin{package}{Acorn}{initials}{Adobe Type~1 decorative initial fonts.}
For each font, at least an .pfb and .tfm file is provided, with
a .fd file for use with {\LaTeX}.
\end{package}
\begin{package}{AnnSton}{initials}{Adobe Type~1 decorative initial fonts.}
For each font, at least an .pfb and .tfm file is provided, with
a .fd file for use with {\LaTeX}.
\end{package}
% \begin{package}{ArtNouv}{initials}{\uppercase{Adobe Type~1 decorative initial fonts.}}
% \uppercase{For each font, at least an .pfb and .tfm file is provided, with
% a .fd file for use with {\LaTeX}.}
% \end{package}
\begin{package}{ArtNouvc}{initials}{Adobe Type~1 decorative initial fonts.}
For each font, at least an .pfb and .tfm file is provided, with
a .fd file for use with {\LaTeX}.
\end{package}
\begin{package}{Carrickc}{initials}{Adobe Type~1 decorative initial fonts.}
For each font, at least an .pfb and .tfm file is provided, with
a .fd file for use with {\LaTeX}.
\end{package}
\begin{package}{Eichenla}{initials}{Adobe Type~1 decorative initial fonts.}
For each font, at least an .pfb and .tfm file is provided, with
a .fd file for use with {\LaTeX}.
\end{package}
\begin{package}{Eileen}{initials}{Adobe Type~1 decorative initial fonts.}
For each font, at least an .pfb and .tfm file is provided, with
a .fd file for use with {\LaTeX}.
\end{package}
\begin{package}{EileenBl}{initials}{Adobe Type~1 decorative initial fonts.}
For each font, at least an .pfb and .tfm file is provided, with
a .fd file for use with {\LaTeX}.
\end{package}
\begin{package}{Elzevier}{initials}{Adobe Type~1 decorative initial fonts.}
For each font, at least an .pfb and .tfm file is provided, with
a .fd file for use with {\LaTeX}.
\end{package}
% \begin{package}{GotIn}{initials}{Adobe Type~1 decorative initial fonts.}
% For each font, at least an .pfb and .tfm file is provided, with
% a .fd file for use with {\LaTeX}.
% \end{package}
% \begin{package}{GoudyIn}{initials}{Adobe Type~1 decorative initial fonts.}
% For each font, at least an .pfb and .tfm file is provided, with
% a .fd file for use with {\LaTeX}.
% \end{package}
\begin{package}{Kinigcap}{initials}{Adobe Type~1 decorative initial fonts.}
For each font, at least an .pfb and .tfm file is provided, with
a .fd file for use with {\LaTeX}.
\end{package}
\begin{package}{Konanur}{initials}{Adobe Type~1 decorative initial fonts.}
For each font, at least an .pfb and .tfm file is provided, with
a .fd file for use with {\LaTeX}.
\end{package}
\begin{package}{Kramer}{initials}{Adobe Type~1 decorative initial fonts.}
For each font, at least an .pfb and .tfm file is provided, with
a .fd file for use with {\LaTeX}.
\end{package}
\begin{package}{MorrisIn}{initials}{Adobe Type~1 decorative initial fonts.}
For each font, at least an .pfb and .tfm file is provided, with
a .fd file for use with {\LaTeX}.
\end{package}
\begin{package}{Nouveaud}{initials}{Adobe Type~1 decorative initial fonts.}
For each font, at least an .pfb and .tfm file is provided, with
a .fd file for use with {\LaTeX}.
\end{package}
\begin{package}{Romantik}{initials}{Adobe Type~1 decorative initial fonts.}
For each font, at least an .pfb and .tfm file is provided, with
a .fd file for use with {\LaTeX}.
\end{package}
\begin{package}{Rothdn}{initials}{Adobe Type~1 decorative initial fonts.}
For each font, at least an .pfb and .tfm file is provided, with
a .fd file for use with {\LaTeX}.
\end{package}
% \begin{package}{RoyalIn}{initials}{ADOBE TYPE~1 DECORATIVE INITIAL FONTS.}
% FOR EACH FONT, AT LEAST AN .PFB AND .TFM FILE IS PROVIDED, WITH
% A .FD FILE FOR USE WITH {\LaTeX}.
% \end{package}
\begin{package}{Sanremo}{initials}{Adobe Type~1 decorative initial fonts.}
For each font, at least an .pfb and .tfm file is provided, with
a .fd file for use with {\LaTeX}.
\end{package}
\begin{package}{Starburst}{initials}{Adobe Type~1 decorative initial fonts.}
For each font, at least an .pfb and .tfm file is provided, with
a .fd file for use with {\LaTeX}.
\end{package}
\begin{package}{Typocaps}{initials}{Adobe Type~1 decorative initial fonts.}
For each font, at least an .pfb and .tfm file is provided, with
a .fd file for use with {\LaTeX}.
\end{package}
\begin{package}{Zallman}{initials}{Adobe Type~1 decorative initial fonts.}
For each font, at least an .pfb and .tfm file is provided, with
a .fd file for use with {\LaTeX}.
\end{package}

\begin{package}{ipxg-r-t1}{ipaex-type1}{IPAex fonts converted to Type-1 format Unicode subfonts.}
The package contains the IPAex Fonts converted into Unicode
subfonts in Type1 format, which is most suitable for use with
the CJK package. Font conversion was done with ttf2pt1.
\end{package}
\begin{package}{ipxg-ro-t1}{ipaex-type1}{IPAex fonts converted to Type-1 format Unicode subfonts.}
The package contains the IPAex Fonts converted into Unicode
subfonts in Type1 format, which is most suitable for use with
the CJK package. Font conversion was done with ttf2pt1.
\end{package}
\begin{package}{ipxm-r-t1}{ipaex-type1}{IPAex fonts converted to Type-1 format Unicode subfonts.}
The package contains the IPAex Fonts converted into Unicode
subfonts in Type1 format, which is most suitable for use with
the CJK package. Font conversion was done with ttf2pt1.
\end{package}
\begin{package}{ipxm-ro-t1}{ipaex-type1}{IPAex fonts converted to Type-1 format Unicode subfonts.}
The package contains the IPAex Fonts converted into Unicode
subfonts in Type1 format, which is most suitable for use with
the CJK package. Font conversion was done with ttf2pt1.
\end{package}

\begin{package}{ec-iwonar}{iwona}{A two-element sans-serif font.}
Iwona is a two-element sans-serif typeface. It was created as
an alternative version of the Kurier typeface, which was
designed in 1975 for a diploma in typeface design at the Warsaw
Academy of Fine Arts under the supervision of Roman
Tomaszewski. This distribution contains a significantly
extended set of characters covering the following modern
alphabets: latin (including Vietnamese), Cyrillic and Greek as
well as a number of additional symbols (including mathematical
symbols). The fonts are prepared in Type~1 and OpenType
formats. For use with {\TeX} the following encoding files have
been prepared: T1 (ec), T2 (abc), and OT2--Cyrillic, T5
(Vietnamese), OT4, QX, texansi and nonstandard (IL2 for the
Czech fonts), as well as supporting macros and files defining
fonts for {\LaTeX}.
\end{package}

% \begin{package}{}{jablantile}{Metafont version of tiles in the style of Slavik Jablan.}
% This is a small Metafont font to implement the modular tiles
% described by Slavik Jablan. For an outline of the theoretical
% structure of the tiles, see (for example) Jablan's JMM 2006
% Exhibit.
% \end{package}
% \begin{package}{}{jamtimes}{Expanded Times Roman fonts.}
% The package offers {\LaTeX} support for the expanded Times Roman
% font, which has been used for many years in the Journal
% d'Analyse Mathematique. Mathematics support is based on the
% Belleek fonts.
% \end{package}
% \begin{package}{}{junicode}{A TrueType font for mediaevalists.}
% Junicode is a TrueType font with many OpenType features for
% antiquarians (especially medievalists) based on typefaces used
% by the Oxford Press in the late 17th and early 18th centuries.
% It works well with Xe(La){\TeX}.
% \end{package}
% \begin{package}{}{kixfont}{A font for KIX codes.}
% The KIX code is a barcode-like format used by the Dutch PTT to
% encode country codes, zip codes and street numbers in a
% machine-readable format. If printed below the address line on
% bulk mailings, a discount can be obtained. The font is
% distributed in MetaFont format, and covers the numbers and
% upper-case letters.
% \end{package}
% \begin{package}{}{knuthotherfonts}{}
% \end{package}
% \begin{package}{jkpbn8r}{kpfonts}{A complete set of fonts for text and mathematics.}
% The family contains text fonts in roman, sans-serif and
% monospaced shapes, with true small caps and old-style numbers;
% the package offers full support of the textcomp package. The
% mathematics fonts include all the AMS fonts, in both normal and
% bold weights. Each of the font types is available in two main
% versions: default and 'light'. Each version is available in
% four variants: default; oldstyle numbers; oldstyle numbers with
% old ligatures such as ct and st, and long-tailed capital Q; and
% veryoldstyle with long s. Other variants include small caps as
% default or 'large small caps', and for mathematics both upright
% and slanted shapes for Greek letters, as well as default and
% narrow versions of multiple integrals. The fonts were
% originally derived from URW Palladio (with URW's agreement)
% though the fonts are very clearly different in appearance from
% their parent.
% \end{package}

\begin{package}{ec-kurierr}{kurier}{A two-element sans-serif typeface.}
Kurier is a two-element sans-serif typeface. It was designed
for a diploma in typeface design at the Warsaw Academy of Fine
Arts under the supervision of Roman Tomaszewski. This
distribution contains a significantly extended set of
characters covering the following modern alphabets: latin
(including Vietnamese), Cyrillic and Greek as well as a number
of additional symbols (including mathematical symbols). The
fonts are prepared in Type~1 and OpenType formats. For use with
TeX the following encoding files have been prepared: T1 (ec),
T2 (abc), and OT2--Cyrillic, T5 (Vietnamese), OT4, QX, texansi
and--nonstandard (IL2 for the Czech fonts), as well as
supporting macros and files defining fonts for {\LaTeX}.
\end{package}

% \begin{package}{}{lato}{Lato font fanily and {\LaTeX} support.}
% Lato is a sanserif typeface family designed in the Summer 2010
% by Warsaw-based designer Lukasz Dziedzic for the tyPoland
% foundry. This font, which includes five weights (hairline,
% light, regular, bold and black), is available from the Google
% Font Directory as TrueType files under the Open Font License
% version~1.1. The package provides support for this font in
% {\LaTeX}. It includes the original TrueType fonts, as well as Type
% 1 versions, converted for this package using FontForge for full
% support with Dvips.
% \end{package}
% \begin{package}{}{lfb}{A Greek font with normal and bold variants.}
% This is a Greek font written in MetaFont, with inspiration from
% the Bodoni typefaces in old books. It is stylistically a little
% more exotic than the standard textbook Greek fonts,
% particularly in glyphs like the lowercase rho and kappa. It
% aims for a rather calligraphic feel, but seems to blend well
% with Computer Modern. There is a ligature scheme which
% automatically inserts the breathings required for ancient
% texts, making the input text more readable than in some
% schemes.
% \end{package}

\begin{package}{LinBiolinumT-lf-t1--base}{libertine}{Use of Linux Libertine and Biolinum fonts with {\LaTeX}.}
The package provides the Libertine and Biolinum fonts in both
Type~1 and OTF styles, together with support macros for their
use. Monospaced and display fonts, and the ``keyboard'' set are
also included, in OTF style, only. The mweights package is used
to manage the selection of font weights. The package supersedes
both the libertineotf and the libertine-legacy packages.
\end{package}
\begin{package}{LinBiolinumTB-lf-t1--base}{libertine}{Use of Linux Libertine and Biolinum fonts with {\LaTeX}.}
The package provides the Libertine and Biolinum fonts in both
Type~1 and OTF styles, together with support macros for their
use. Monospaced and display fonts, and the ``keyboard'' set are
also included, in OTF style, only. The mweights package is used
to manage the selection of font weights. The package supersedes
both the libertineotf and the libertine-legacy packages.
\end{package}
% \begin{package}{LinBiolinumTBO-lf-t1--base}{libertine}{Use of Linux Libertine and Biolinum fonts with {\LaTeX}.}
% The package provides the Libertine and Biolinum fonts in both
% Type~1 and OTF styles, together with support macros for their
% use. Monospaced and display fonts, and the ``keyboard'' set are
% also included, in OTF style, only. The mweights package is used
% to manage the selection of font weights. The package supersedes
% both the libertineotf and the libertine-legacy packages.
% \end{package}
\begin{package}{LinBiolinumTI-lf-t1--base}{libertine}{Use of Linux Libertine and Biolinum fonts with {\LaTeX}.}
The package provides the Libertine and Biolinum fonts in both
Type~1 and OTF styles, together with support macros for their
use. Monospaced and display fonts, and the ``keyboard'' set are
also included, in OTF style, only. The mweights package is used
to manage the selection of font weights. The package supersedes
both the libertineotf and the libertine-legacy packages.
\end{package}
\begin{package}{LinLibertineDisplayT-lf-t1--base}{libertine}{Use of Linux Libertine and Biolinum fonts with {\LaTeX}.}
The package provides the Libertine and Biolinum fonts in both
Type~1 and OTF styles, together with support macros for their
use. Monospaced and display fonts, and the ``keyboard'' set are
also included, in OTF style, only. The mweights package is used
to manage the selection of font weights. The package supersedes
both the libertineotf and the libertine-legacy packages.
\end{package}
% \begin{package}{LinLibertineIT-lf-t1--base}{libertine}{Use of Linux Libertine and Biolinum fonts with {\LaTeX}.}
% The package provides the Libertine and Biolinum fonts in both
% Type~1 and OTF styles, together with support macros for their
% use. Monospaced and display fonts, and the ``keyboard'' set are
% also included, in OTF style, only. The mweights package is used
% to manage the selection of font weights. The package supersedes
% both the libertineotf and the libertine-legacy packages.
% \end{package}
\begin{package}{LinLibertineT-lf-t1--base}{libertine}{Use of Linux Libertine and Biolinum fonts with {\LaTeX}.}
The package provides the Libertine and Biolinum fonts in both
Type~1 and OTF styles, together with support macros for their
use. Monospaced and display fonts, and the ``keyboard'' set are
also included, in OTF style, only. The mweights package is used
to manage the selection of font weights. The package supersedes
both the libertineotf and the libertine-legacy packages.
\end{package}
\begin{package}{LinLibertineTB-lf-t1--base}{libertine}{Use of Linux Libertine and Biolinum fonts with {\LaTeX}.}
The package provides the Libertine and Biolinum fonts in both
Type~1 and OTF styles, together with support macros for their
use. Monospaced and display fonts, and the ``keyboard'' set are
also included, in OTF style, only. The mweights package is used
to manage the selection of font weights. The package supersedes
both the libertineotf and the libertine-legacy packages.
\end{package}
% \begin{package}{LinLibertineTBI-lf-t1--base}{libertine}{Use of Linux Libertine and Biolinum fonts with {\LaTeX}.}
% The package provides the Libertine and Biolinum fonts in both
% Type~1 and OTF styles, together with support macros for their
% use. Monospaced and display fonts, and the ``keyboard'' set are
% also included, in OTF style, only. The mweights package is used
% to manage the selection of font weights. The package supersedes
% both the libertineotf and the libertine-legacy packages.
% \end{package}
\begin{package}{LinLibertineTI-lf-t1--base}{libertine}{Use of Linux Libertine and Biolinum fonts with {\LaTeX}.}
The package provides the Libertine and Biolinum fonts in both
Type~1 and OTF styles, together with support macros for their
use. Monospaced and display fonts, and the ``keyboard'' set are
also included, in OTF style, only. The mweights package is used
to manage the selection of font weights. The package supersedes
both the libertineotf and the libertine-legacy packages.
\end{package}
\begin{package}{LinLibertineTZ-lf-t1--base}{libertine}{Use of Linux Libertine and Biolinum fonts with {\LaTeX}.}
The package provides the Libertine and Biolinum fonts in both
Type~1 and OTF styles, together with support macros for their
use. Monospaced and display fonts, and the ``keyboard'' set are
also included, in OTF style, only. The mweights package is used
to manage the selection of font weights. The package supersedes
both the libertineotf and the libertine-legacy packages.
\end{package}
% \begin{package}{LinLibertineTZI-lf-t1--base}{libertine}{Use of Linux Libertine and Biolinum fonts with {\LaTeX}.}
% The package provides the Libertine and Biolinum fonts in both
% Type~1 and OTF styles, together with support macros for their
% use. Monospaced and display fonts, and the ``keyboard'' set are
% also included, in OTF style, only. The mweights package is used
% to manage the selection of font weights. The package supersedes
% both the libertineotf and the libertine-legacy packages.
% \end{package}

\begin{package}{LibreBaskerville-Bold-tlf-t1--base}{librebaskerville}{{\LaTeX} support for the Libre Baskerville family of fonts.}
Libre Baskerville is designed by Pablo Impallari. It is
primarily intended to be a web font but is also attractive as a
TeX font. As there is currently no bold italic variant, an
artificially slanted version of the bold variant has been
generated.
\end{package}
% \begin{package}{LibreBaskerville-BoldItalic-tlf-t1--base}{librebaskerville}{{\LaTeX} support for the Libre Baskerville family of fonts.}
% Libre Baskerville is designed by Pablo Impallari. It is
% primarily intended to be a web font but is also attractive as a
% TeX font. As there is currently no bold italic variant, an
% artificially slanted version of the bold variant has been
% generated.
% \end{package}
\begin{package}{LibreBaskerville-Italic-tlf-t1--base}{librebaskerville}{{\LaTeX} support for the Libre Baskerville family of fonts.}
Libre Baskerville is designed by Pablo Impallari. It is
primarily intended to be a web font but is also attractive as a
TeX font. As there is currently no bold italic variant, an
artificially slanted version of the bold variant has been
generated.
\end{package}
\begin{package}{LibreBaskerville-Regular-tlf-t1--base}{librebaskerville}{{\LaTeX} support for the Libre Baskerville family of fonts.}
Libre Baskerville is designed by Pablo Impallari. It is
primarily intended to be a web font but is also attractive as a
TeX font. As there is currently no bold italic variant, an
artificially slanted version of the bold variant has been
generated.
\end{package}

\begin{package}{LibreCaslonText-Bold-tlf-t1--base}{librecaslon}{Libre Caslon fonts, with {\LaTeX} support.}
The Libre Caslon fonts are designed by Pablo Impallari.
Although they have been designed for use as web fonts, they
work well as conventional text fonts. A bold italic variant is
not currently available. As a stopgap, an artificially slanted
bold variant has been created and treated as italic.
\end{package}
% \begin{package}{LibreCaslonText-BoldItalic-tlf-t1--base}{librecaslon}{Libre Caslon fonts, with {\LaTeX} support.}
% The Libre Caslon fonts are designed by Pablo Impallari.
% Although they have been designed for use as web fonts, they
% work well as conventional text fonts. A bold italic variant is
% not currently available. As a stopgap, an artificially slanted
% bold variant has been created and treated as italic.
% \end{package}
\begin{package}{LibreCaslonText-Italic-tlf-t1--base}{librecaslon}{Libre Caslon fonts, with {\LaTeX} support.}
The Libre Caslon fonts are designed by Pablo Impallari.
Although they have been designed for use as web fonts, they
work well as conventional text fonts. A bold italic variant is
not currently available. As a stopgap, an artificially slanted
bold variant has been created and treated as italic.
\end{package}
\begin{package}{LibreCaslonText-Regular-tlf-t1--base}{librecaslon}{Libre Caslon fonts, with {\LaTeX} support.}
The Libre Caslon fonts are designed by Pablo Impallari.
Although they have been designed for use as web fonts, they
work well as conventional text fonts. A bold italic variant is
not currently available. As a stopgap, an artificially slanted
bold variant has been created and treated as italic.
\end{package}

% \begin{package}{ylyr-t1}{libris}{Libris ADF fonts, with {\LaTeX} support.}
% LibrisADF is a sans-serif family designed to mimic Lydian. The
% bundle includes: - fonts, in Adobe Type~1, TrueType and
% OpenType formats, and - {\LaTeX} support macros, for use with the
% Type~1 versions of the fonts. The {\LaTeX} macros depend on the
% nfssext-cfr bundle. GPL licensing applies the the fonts
% themselves; the support macros are distributed under LPPL
% licensing.
% \end{package}

\begin{package}{LinearA}{linearA}{Linear A script fonts.}
The linearA package provides a simple interface to two fonts
which include all known symbols, simple and complex, of the
Linear A script. This way one can easily replicate Linear A
``texts'' using modern typographic technology. Note that the
Linear A script has not been deciphered yet and probably never
will be deciphered.
\end{package}
\begin{package}{LinearACmplxSigns}{linearA}{Linear A script fonts.}
The linearA package provides a simple interface to two fonts
which include all known symbols, simple and complex, of the
Linear A script. This way one can easily replicate Linear A
``texts'' using modern typographic technology. Note that the
Linear A script has not been deciphered yet and probably never
will be deciphered.
\end{package}

\begin{package}{leclb8}{lxfonts}{Set of slide fonts based on CM.}
The bundle contains the traditional slides fonts revised to be
completely usable both as text fonts and mathematics fonts;
they are fully integrate with the new operators, letters,
symbols and extensible delimiter fonts, as well as with the AMS
fonts, all redone with the same stylistic parameters.
\end{package}
\begin{package}{llcmss8}{lxfonts}{Set of slide fonts based on CM.}
The bundle contains the traditional slides fonts revised to be
completely usable both as text fonts and mathematics fonts;
they are fully integrate with the new operators, letters,
symbols and extensible delimiter fonts, as well as with the AMS
fonts, all redone with the same stylistic parameters.
\end{package}

% \begin{package}{}{ly1}{Support for LY1 {\LaTeX} encoding.}
% The Y\&Y 'texnansi' (TeX and ANSI, for Microsoft interpretations
% of ANSI standards) encoding lives on, even after the decease of
% the company; it is known in the {\LaTeX} scheme of things as LY1
% encoding. This bundle includes metrics and {\LaTeX} macros to use
% the basic three (Times, Helvetica and Courier) Adobe Type~1
% fonts in {\LaTeX} using LY1 encoding.
% \end{package}
% \begin{package}{}{mathabx}{Three series of mathematical symbols.}
% Mathabx is a set of 3 mathematical symbols font series: matha,
% mathb and mathx. They are defined by MetaFont code and should
% be of reasonable quality (bitmap output). Things change from
% time to time, so there is no claim of stability (encoding,
% metrics, design). The package includes Plain {\TeX} and {\LaTeX}
% support macros.
% \end{package}

\begin{package}{mathc10}{mathabx-type1}{Outline version of the mathabx fonts.}
This is an Adobe Type~1 outline version of the mathabx fonts.

Mathabx is a set of 3 mathematical symbols font series: matha,
mathb and mathx. They are defined by MetaFont code and should
be of reasonable quality (bitmap output). Things change from
time to time, so there is no claim of stability (encoding,
metrics, design). The package includes Plain {\TeX} and {\LaTeX}
support macros.
\end{package}
% \begin{package}{md-gmm7t}{mathdesign}{Mathematical fonts to fit with particular text fonts.}
% The Math Design project offers free mathematical fonts that
% match with existing text fonts. To date, three free font
% families are available: Adobe Utopia, URW Garamond and
% Bitstream Charter. Three commercial fonts are also supported:
% Adobe Garamond Pro, Adobe UtopiaStd and ITC Charter. Mathdesign
% covers the whole {\LaTeX} glyph set, including AMS symbols and
% some extra. Both roman and bold versions of these symbols can
% be used. Moreover you can choose between three greek fonts (two
% of them created by the Greek Font Society).
% \end{package}
% \begin{package}{}{mdputu}{Upright digits in Adobe Utopia Italic.}
% The Annals of Mathematics uses italics for theorems. However,
% slanted digits and parentheses look disturbing when surrounded
% by (upright) mathematics. This package provides virtual fonts
% with italics and upright digits and punctuation, as an
% extension to Mathdesign Utopia.
% \end{package}
% \begin{package}{}{mdsymbol}{Symbol fonts to match Adobe Myriad Pro.}
% The package provides a font of mathematical symbols, MyriadPro
% The font is designed as a companion to Adobe Myriad Pro, but it
% might also fit well with other contemporary typefaces.
% \end{package}

\begin{package}{Merriweather-Bold-osf-t1--base}{merriweather}{Merriweather and MerriweatherSans fonts, with {\LaTeX} support.}
Merriweather features a very large x height, slightly condensed
letterforms, a mild diagonal stress, sturdy serifs and open
forms. The Sans family closely harmonizes with the weights and
styles of the serif family. There are four weights and italics
for each.
\end{package}
% \begin{package}{Merriweather-BoldItalic-osf-t1--base}{merriweather}{Merriweather and MerriweatherSans fonts, with {\LaTeX} support.}
% Merriweather features a very large x height, slightly condensed
% letterforms, a mild diagonal stress, sturdy serifs and open
% forms. The Sans family closely harmonizes with the weights and
% styles of the serif family. There are four weights and italics
% for each.
% \end{package}
\begin{package}{Merriweather-UltraBold-osf-t1--base}{merriweather}{Merriweather and MerriweatherSans fonts, with {\LaTeX} support.}
Merriweather features a very large x height, slightly condensed
letterforms, a mild diagonal stress, sturdy serifs and open
forms. The Sans family closely harmonizes with the weights and
styles of the serif family. There are four weights and italics
for each.
\end{package}
% \begin{package}{Merriweather-UltraBdIt-osf-t1--base}{merriweather}{Merriweather and MerriweatherSans fonts, with {\LaTeX} support.}
% Merriweather features a very large x height, slightly condensed
% letterforms, a mild diagonal stress, sturdy serifs and open
% forms. The Sans family closely harmonizes with the weights and
% styles of the serif family. There are four weights and italics
% for each.
% \end{package}
\begin{package}{Merriweather-Italic-osf-t1--base}{merriweather}{Merriweather and MerriweatherSans fonts, with {\LaTeX} support.}
Merriweather features a very large x height, slightly condensed
letterforms, a mild diagonal stress, sturdy serifs and open
forms. The Sans family closely harmonizes with the weights and
styles of the serif family. There are four weights and italics
for each.
\end{package}
\begin{package}{Merriweather-Light-osf-t1--base}{merriweather}{Merriweather and MerriweatherSans fonts, with {\LaTeX} support.}
Merriweather features a very large x height, slightly condensed
letterforms, a mild diagonal stress, sturdy serifs and open
forms. The Sans family closely harmonizes with the weights and
styles of the serif family. There are four weights and italics
for each.
\end{package}
% \begin{package}{Merriweather-LightItalic-osf-t1--base}{merriweather}{Merriweather and MerriweatherSans fonts, with {\LaTeX} support.}
% Merriweather features a very large x height, slightly condensed
% letterforms, a mild diagonal stress, sturdy serifs and open
% forms. The Sans family closely harmonizes with the weights and
% styles of the serif family. There are four weights and italics
% for each.
% \end{package}
\begin{package}{Merriweather-Regular-osf-t1--base}{merriweather}{Merriweather and MerriweatherSans fonts, with {\LaTeX} support.}
Merriweather features a very large x height, slightly condensed
letterforms, a mild diagonal stress, sturdy serifs and open
forms. The Sans family closely harmonizes with the weights and
styles of the serif family. There are four weights and italics
for each.
\end{package}
\begin{package}{MerriweatherSans-Bold-tlf-t1--base}{merriweather}{Merriweather and MerriweatherSans fonts, with {\LaTeX} support.}
Merriweather features a very large x height, slightly condensed
letterforms, a mild diagonal stress, sturdy serifs and open
forms. The Sans family closely harmonizes with the weights and
styles of the serif family. There are four weights and italics
for each.
\end{package}
% \begin{package}{MerriweatherSans-BoldItalic-tlf-t1--base}{merriweather}{Merriweather and MerriweatherSans fonts, with {\LaTeX} support.}
% Merriweather features a very large x height, slightly condensed
% letterforms, a mild diagonal stress, sturdy serifs and open
% forms. The Sans family closely harmonizes with the weights and
% styles of the serif family. There are four weights and italics
% for each.
% \end{package}
% \begin{package}{MerriweatherSans-ExtraBldItalic-tlf-t1--base}{merriweather}{Merriweather and MerriweatherSans fonts, with {\LaTeX} support.}
% Merriweather features a very large x height, slightly condensed
% letterforms, a mild diagonal stress, sturdy serifs and open
% forms. The Sans family closely harmonizes with the weights and
% styles of the serif family. There are four weights and italics
% for each.
% \end{package}
\begin{package}{MerriweatherSans-ExtraBold-tlf-t1--base}{merriweather}{Merriweather and MerriweatherSans fonts, with {\LaTeX} support.}
Merriweather features a very large x height, slightly condensed
letterforms, a mild diagonal stress, sturdy serifs and open
forms. The Sans family closely harmonizes with the weights and
styles of the serif family. There are four weights and italics
for each.
\end{package}
\begin{package}{MerriweatherSans-Italic-tlf-t1--base}{merriweather}{Merriweather and MerriweatherSans fonts, with {\LaTeX} support.}
Merriweather features a very large x height, slightly condensed
letterforms, a mild diagonal stress, sturdy serifs and open
forms. The Sans family closely harmonizes with the weights and
styles of the serif family. There are four weights and italics
for each.
\end{package}
\begin{package}{MerriweatherSans-Light-tlf-t1--base}{merriweather}{Merriweather and MerriweatherSans fonts, with {\LaTeX} support.}
Merriweather features a very large x height, slightly condensed
letterforms, a mild diagonal stress, sturdy serifs and open
forms. The Sans family closely harmonizes with the weights and
styles of the serif family. There are four weights and italics
for each.
\end{package}
% \begin{package}{MerriweatherSans-LightItalic-tlf-t1--base}{merriweather}{Merriweather and MerriweatherSans fonts, with {\LaTeX} support.}
% Merriweather features a very large x height, slightly condensed
% letterforms, a mild diagonal stress, sturdy serifs and open
% forms. The Sans family closely harmonizes with the weights and
% styles of the serif family. There are four weights and italics
% for each.
% \end{package}
\begin{package}{MerriweatherSans-Regular-tlf-t1--base}{merriweather}{Merriweather and MerriweatherSans fonts, with {\LaTeX} support.}
Merriweather features a very large x height, slightly condensed
letterforms, a mild diagonal stress, sturdy serifs and open
forms. The Sans family closely harmonizes with the weights and
styles of the serif family. There are four weights and italics
for each.
\end{package}

\begin{package}{MintSpirit-Bold-tlf-t1--base}{mintspirit}{{\LaTeX} support for MintSpirit font families.}
The package provides {\LaTeX}, {\pdfLaTeX}, {\XeLaTeX} and {\LuaLaTeX}
support for the MintSpirit and MintSpiritNo2 families of fonts,
designed by Hirwen Harendal. MintSpirit was originally designed
for use as a system font on a Linux Mint system. The No.2
variant provides more conventional shapes for some glyphs.
\end{package}
% \begin{package}{MintSpirit-BoldItalic-tlf-t1--base}{mintspirit}{{\LaTeX} support for MintSpirit font families.}
% The package provides {\LaTeX}, {\pdfLaTeX}, {\XeLaTeX} and {\LuaLaTeX}
% support for the MintSpirit and MintSpiritNo2 families of fonts,
% designed by Hirwen Harendal. MintSpirit was originally designed
% for use as a system font on a Linux Mint system. The No.2
% variant provides more conventional shapes for some glyphs.
% \end{package}
\begin{package}{MintSpirit-Italic-tlf-t1--base}{mintspirit}{{\LaTeX} support for MintSpirit font families.}
The package provides {\LaTeX}, {\pdfLaTeX}, {\XeLaTeX} and {\LuaLaTeX}
support for the MintSpirit and MintSpiritNo2 families of fonts,
designed by Hirwen Harendal. MintSpirit was originally designed
for use as a system font on a Linux Mint system. The No.2
variant provides more conventional shapes for some glyphs.
\end{package}
\begin{package}{MintSpirit-Regular-tlf-t1--base}{mintspirit}{{\LaTeX} support for MintSpirit font families.}
The package provides {\LaTeX}, {\pdfLaTeX}, {\XeLaTeX} and {\LuaLaTeX}
support for the MintSpirit and MintSpiritNo2 families of fonts,
designed by Hirwen Harendal. MintSpirit was originally designed
for use as a system font on a Linux Mint system. The No.2
variant provides more conventional shapes for some glyphs.
\end{package}
\begin{package}{MintSpiritNo2-Bold-tlf-t1--base}{mintspirit}{{\LaTeX} support for MintSpirit font families.}
The package provides {\LaTeX}, {\pdfLaTeX}, {\XeLaTeX} and {\LuaLaTeX}
support for the MintSpirit and MintSpiritNo2 families of fonts,
designed by Hirwen Harendal. MintSpirit was originally designed
for use as a system font on a Linux Mint system. The No.2
variant provides more conventional shapes for some glyphs.
\end{package}
% \begin{package}{MintSpiritNo2-BoldItalic-tlf-t1--base}{mintspirit}{{\LaTeX} support for MintSpirit font families.}
% The package provides {\LaTeX}, {\pdfLaTeX}, {\XeLaTeX} and {\LuaLaTeX}
% support for the MintSpirit and MintSpiritNo2 families of fonts,
% designed by Hirwen Harendal. MintSpirit was originally designed
% for use as a system font on a Linux Mint system. The No.2
% variant provides more conventional shapes for some glyphs.
% \end{package}
\begin{package}{MintSpiritNo2-Italic-tlf-t1--base}{mintspirit}{{\LaTeX} support for MintSpirit font families.}
The package provides {\LaTeX}, {\pdfLaTeX}, {\XeLaTeX} and {\LuaLaTeX}
support for the MintSpirit and MintSpiritNo2 families of fonts,
designed by Hirwen Harendal. MintSpirit was originally designed
for use as a system font on a Linux Mint system. The No.2
variant provides more conventional shapes for some glyphs.
\end{package}
\begin{package}{MintSpiritNo2-Regular-tlf-t1--base}{mintspirit}{{\LaTeX} support for MintSpirit font families.}
The package provides {\LaTeX}, {\pdfLaTeX}, {\XeLaTeX} and {\LuaLaTeX}
support for the MintSpirit and MintSpiritNo2 families of fonts,
designed by Hirwen Harendal. MintSpirit was originally designed
for use as a system font on a Linux Mint system. The No.2
variant provides more conventional shapes for some glyphs.
\end{package}

% \begin{package}{MnSymbolA10}{mnsymbol}{Mathematical symbol font for Adobe MinionPro.}
% MnSymbol is a symbol font family, designed to be used in
% conjunction with Adobe Minion Pro (via the MinionPro package).
% Almost all of {\LaTeX} and AMS mathematical symbols are provided;
% remaining coverage is available from the MinionPro font with
% the MinionPro package. The fonts are available in both MetaFont
% and Adobe Type~1 formats, and a comprehensive support package
% is provided. While the fonts were designed to fit with Minon
% Pro, the design should fit well with other renaissance or
% baroque faces: indeed, it will probably work with most fonts
% that are neither too wide nor too thin, for example Palatino or
% Times; it is known to look good with Sabon. There is no package
% designed to configure its use with any font other than Minion
% Pro, but (for example) simply loading mnsymbol after mathpazo
% will probably do what is needed.
% \end{package}

% \begin{package}{xtie20}{musixtex-fonts}{Fonts used by MusixTeX.}
% These are fonts for use with MusixTeX; they are provided both
% as original Metafont source, and as converted Adobe Type 1. The
% bundle renders the older (Type 1 fonts only) bundle musixtex-
% t1fonts obsolete.
% \end{package}

\begin{package}{T1-TeXGyrePagellaX-Regular-lnum-kern-liga--base}{newpx}{Alternative uses of the PX fonts, with improved metrics.}
The package, based on pxfonts, provides many fixes and
enhancements to that package, and splits it in two parts
(newpxtext and newpxmath) which may be run independently of one
another. It provides scaling, improved metrics, and other
options. For proper operation, the packages require that the
packages newtxmath and txfonts be installed and their map files
enabled.
\end{package}
\begin{package}{T1-TeXGyrePagellaX-Regular-onum-kern-liga--base}{newpx}{Alternative uses of the PX fonts, with improved metrics.}
The package, based on pxfonts, provides many fixes and
enhancements to that package, and splits it in two parts
(newpxtext and newpxmath) which may be run independently of one
another. It provides scaling, improved metrics, and other
options. For proper operation, the packages require that the
packages newtxmath and txfonts be installed and their map files
enabled.
\end{package}
\begin{package}{T1-TeXGyrePagellaX-Regular-pnum-kern-liga--base}{newpx}{Alternative uses of the PX fonts, with improved metrics.}
The package, based on pxfonts, provides many fixes and
enhancements to that package, and splits it in two parts
(newpxtext and newpxmath) which may be run independently of one
another. It provides scaling, improved metrics, and other
options. For proper operation, the packages require that the
packages newtxmath and txfonts be installed and their map files
enabled.
\end{package}


\begin{package}{MinLibBol-t1}{newtx}{Alternative uses of the TX fonts, with improved metrics.}
The bundle splits txfonts.sty (from the TX fonts distribution)
into two independent packages, ntxtext.sty and ntxmath.sty,
each with fixes and enhancements. Ntxmath's metrics have been
re-evaluated to provide a less tight appearance, and to provide
a libertine option that substitutes Libertine italic and Greek
letter for the existing math italic and Greek glyphs, making a
mathematics package that matches Libertine text quite well.
Ntxmath can also use the maths italic font provided with the
garamondx package, thus offering a garamond-alike text-with-
maths combination.
\end{package}
% \begin{package}{MinLibBolIta-t1}{newtx}{Alternative uses of the TX fonts, with improved metrics.}
% The bundle splits txfonts.sty (from the TX fonts distribution)
% into two independent packages, ntxtext.sty and ntxmath.sty,
% each with fixes and enhancements. Ntxmath's metrics have been
% re-evaluated to provide a less tight appearance, and to provide
% a libertine option that substitutes Libertine italic and Greek
% letter for the existing math italic and Greek glyphs, making a
% mathematics package that matches Libertine text quite well.
% Ntxmath can also use the maths italic font provided with the
% garamondx package, thus offering a garamond-alike text-with-
% maths combination.
% \end{package}
\begin{package}{MinLibIta-t1}{newtx}{Alternative uses of the TX fonts, with improved metrics.}
The bundle splits txfonts.sty (from the TX fonts distribution)
into two independent packages, ntxtext.sty and ntxmath.sty,
each with fixes and enhancements. Ntxmath's metrics have been
re-evaluated to provide a less tight appearance, and to provide
a libertine option that substitutes Libertine italic and Greek
letter for the existing math italic and Greek glyphs, making a
mathematics package that matches Libertine text quite well.
Ntxmath can also use the maths italic font provided with the
garamondx package, thus offering a garamond-alike text-with-
maths combination.
\end{package}
\begin{package}{MinLibReg-t1}{newtx}{Alternative uses of the TX fonts, with improved metrics.}
The bundle splits txfonts.sty (from the TX fonts distribution)
into two independent packages, ntxtext.sty and ntxmath.sty,
each with fixes and enhancements. Ntxmath's metrics have been
re-evaluated to provide a less tight appearance, and to provide
a libertine option that substitutes Libertine italic and Greek
letter for the existing math italic and Greek glyphs, making a
mathematics package that matches Libertine text quite well.
Ntxmath can also use the maths italic font provided with the
garamondx package, thus offering a garamond-alike text-with-
maths combination.
\end{package}

% \begin{package}{}{nkarta}{A ``new'' version of the karta cartographic fonts.}
% A development of the karta font, offering more mathematical
% stability in MetaFont. A version that will produce the glyphs
% as Encapsulated PostScript, using MetaPost, is also provided.
% \end{package}
% \begin{package}{Cherokee}{ocherokee}{{\LaTeX} Support for the Cherokee language.}
% Macros and Type~1 fonts for Typesetting the Cherokee language
% with the Omega version of {\LaTeX} (known as Lambda).
% \end{package}
% \begin{package}{}{ocr-b}{Fonts for OCR-B.}
% MetaFont programs for OCR-B at several sizes.
% \end{package}

\begin{package}{ocrb10}{ocr-b-outline}{OCR-B fonts in Type~1 and OpenType.}
The package contains OCR-B fonts in Type1 and OpenType formats.
They were generated from the MetaFont sources of the OCR-B
fonts. The metric files are not included here, so that original
ocr-b package should also be installed.
\end{package}

% \begin{package}{}{ogham}{Fonts for typesetting Ogham script.}
% The font provides the Ogham alphabet, which is found on a
% number of Irish and Pictish carvings dating from the 4th
% century AD. The font is distributed as Metafont source, which
% has been patched (with the author's permission) for stability
% at different output device resolutions. (Thanks are due to
% Peter Flynn and Dan Luecking.)
% \end{package}
% \begin{package}{}{oinuit}{{\LaTeX} Support for the Inuktitut Language.}
% The package provides a set of Lambda (Omega {\LaTeX}) typesetting
% tools for the Inuktitut language. Five different input methods
% are supported and with the necessary fonts are also provided.
% \end{package}
% \begin{package}{}{oldlatin}{Compute Modern like font with long s.}
% Metafont sources modified from Computer Modern in order to
% generate ``long s'' which was used in old text.
% \end{package}
% \begin{package}{}{oldstandard}{Old Standard: A Unicode Font for Classical and Medieval Studies.}
% Old Standard is designed to reproduce the actual printing style
% of the early 20th century, reviving a specific type of Modern
% (classicist) style of serif typefaces, very commonly used in
% various editions of the late 19th and early 20th century, but
% almost completely abandoned later. The font supports
% typesetting of Old and Middle English, Old Icelandic, Cyrillic
% (with historical characters, extensions for Old Slavonic and
% localised forms), Gothic transliterations, critical editions of
% Classical Greek and Latin, and many more. Old Standard works
% with {\TeX} engines that directly support OpenType features, such
% as {\XeTeX} and LuaTeX.
% \end{package}
% \begin{package}{}{opensans}{The Open Sans font family, and {\LaTeX} support.}
% Open Sans is a humanist sans serif typeface designed by Steve
% Matteson; the font is available from the Google Font Directory
% as TrueType files licensed under the Apache License version
% 2.0. The package provides support for this font family in
% {\LaTeX}. It includes the original TrueType fonts, as well as Type
% 1 versions, converted for this package using FontForge for full
% support with dvips
% \end{package}
% \begin{package}{}{orkhun}{A font for orkhun script.}
% The font covers an old Turkic script. It is provided as
% MetaFont source.
% \end{package}
% \begin{package}{}{pacioli}{Fonts designed by Fra Luca de Pacioli in 1497.}
% Pacioli was a c.15 mathematician, and his font was designed
% according to 'the divine proportion'. The font is uppercase
% letters together with punctuation and some analphabetics; no
% lowercase or digits. The Metafont source is distributed in a
% .dtx file, together with {\LaTeX} support.
% \end{package}

% \begin{package}{PTMono-Bold-tlf-t1--base}{paratype}{{\LaTeX} support for free fonts by ParaType.}
% The package offers {\LaTeX} support for the fonts PT Sans, PT
% Serif and PT Mono developed by ParaType for the project ``Public
% Types of Russian Federation'', and released under an open user
% license. The fonts themselves are provided in both the TrueType
% and Type~1 formats, both created by ParaType). The fonts
% provide encodings OT1, T1, IL2, TS1, T2* and X2. The package
% provides a convenient replacement of the two packages ptsans
% and ptserif.
% \end{package}
% \begin{package}{PTMono-BoldSlanted-tlf-t1--base}{paratype}{{\LaTeX} support for free fonts by ParaType.}
% The package offers {\LaTeX} support for the fonts PT Sans, PT
% Serif and PT Mono developed by ParaType for the project ``Public
% Types of Russian Federation'', and released under an open user
% license. The fonts themselves are provided in both the TrueType
% and Type~1 formats, both created by ParaType). The fonts
% provide encodings OT1, T1, IL2, TS1, T2* and X2. The package
% provides a convenient replacement of the two packages ptsans
% and ptserif.
% \end{package}
\begin{package}{PTMono-Regular-tlf-t1--base}{paratype}{{\LaTeX} support for free fonts by ParaType.}
The package offers {\LaTeX} support for the fonts PT Sans, PT
Serif and PT Mono developed by ParaType for the project ``Public
Types of Russian Federation'', and released under an open user
license. The fonts themselves are provided in both the TrueType
and Type~1 formats, both created by ParaType). The fonts
provide encodings OT1, T1, IL2, TS1, T2* and X2. The package
provides a convenient replacement of the two packages ptsans
and ptserif.
\end{package}
% \begin{package}{PTMono-Slanted-tlf-t1--base}{paratype}{{\LaTeX} support for free fonts by ParaType.}
% The package offers {\LaTeX} support for the fonts PT Sans, PT
% Serif and PT Mono developed by ParaType for the project ``Public
% Types of Russian Federation'', and released under an open user
% license. The fonts themselves are provided in both the TrueType
% and Type~1 formats, both created by ParaType). The fonts
% provide encodings OT1, T1, IL2, TS1, T2* and X2. The package
% provides a convenient replacement of the two packages ptsans
% and ptserif.
% \end{package}
\begin{package}{PTSans-Bold-tlf-t1--base}{paratype}{{\LaTeX} support for free fonts by ParaType.}
The package offers {\LaTeX} support for the fonts PT Sans, PT
Serif and PT Mono developed by ParaType for the project ``Public
Types of Russian Federation'', and released under an open user
license. The fonts themselves are provided in both the TrueType
and Type~1 formats, both created by ParaType). The fonts
provide encodings OT1, T1, IL2, TS1, T2* and X2. The package
provides a convenient replacement of the two packages ptsans
and ptserif.
\end{package}
% \begin{package}{PTSans-BoldItalic-tlf-t1--base}{paratype}{{\LaTeX} support for free fonts by ParaType.}
% The package offers {\LaTeX} support for the fonts PT Sans, PT
% Serif and PT Mono developed by ParaType for the project ``Public
% Types of Russian Federation'', and released under an open user
% license. The fonts themselves are provided in both the TrueType
% and Type~1 formats, both created by ParaType). The fonts
% provide encodings OT1, T1, IL2, TS1, T2* and X2. The package
% provides a convenient replacement of the two packages ptsans
% and ptserif.
% \end{package}
\begin{package}{PTSans-Caption-tlf-t1--base}{paratype}{{\LaTeX} support for free fonts by ParaType.}
The package offers {\LaTeX} support for the fonts PT Sans, PT
Serif and PT Mono developed by ParaType for the project ``Public
Types of Russian Federation'', and released under an open user
license. The fonts themselves are provided in both the TrueType
and Type~1 formats, both created by ParaType). The fonts
provide encodings OT1, T1, IL2, TS1, T2* and X2. The package
provides a convenient replacement of the two packages ptsans
and ptserif.
\end{package}
% \begin{package}{PTSans-CaptionBold-tlf-t1--base}{paratype}{{\LaTeX} support for free fonts by ParaType.}
% The package offers {\LaTeX} support for the fonts PT Sans, PT
% Serif and PT Mono developed by ParaType for the project ``Public
% Types of Russian Federation'', and released under an open user
% license. The fonts themselves are provided in both the TrueType
% and Type~1 formats, both created by ParaType). The fonts
% provide encodings OT1, T1, IL2, TS1, T2* and X2. The package
% provides a convenient replacement of the two packages ptsans
% and ptserif.
% \end{package}
% \begin{package}{PTSans-CaptionBoldSlanted-tlf-t1--base}{paratype}{{\LaTeX} support for free fonts by ParaType.}
% The package offers {\LaTeX} support for the fonts PT Sans, PT
% Serif and PT Mono developed by ParaType for the project ``Public
% Types of Russian Federation'', and released under an open user
% license. The fonts themselves are provided in both the TrueType
% and Type~1 formats, both created by ParaType). The fonts
% provide encodings OT1, T1, IL2, TS1, T2* and X2. The package
% provides a convenient replacement of the two packages ptsans
% and ptserif.
% \end{package}
% \begin{package}{PTSans-CaptionSlanted-tlf-t1--base}{paratype}{{\LaTeX} support for free fonts by ParaType.}
% The package offers {\LaTeX} support for the fonts PT Sans, PT
% Serif and PT Mono developed by ParaType for the project ``Public
% Types of Russian Federation'', and released under an open user
% license. The fonts themselves are provided in both the TrueType
% and Type~1 formats, both created by ParaType). The fonts
% provide encodings OT1, T1, IL2, TS1, T2* and X2. The package
% provides a convenient replacement of the two packages ptsans
% and ptserif.
% \end{package}
\begin{package}{PTSans-Italic-tlf-t1--base}{paratype}{{\LaTeX} support for free fonts by ParaType.}
The package offers {\LaTeX} support for the fonts PT Sans, PT
Serif and PT Mono developed by ParaType for the project ``Public
Types of Russian Federation'', and released under an open user
license. The fonts themselves are provided in both the TrueType
and Type~1 formats, both created by ParaType). The fonts
provide encodings OT1, T1, IL2, TS1, T2* and X2. The package
provides a convenient replacement of the two packages ptsans
and ptserif.
\end{package}
\begin{package}{PTSans-Narrow-tlf-t1--base}{paratype}{{\LaTeX} support for free fonts by ParaType.}
The package offers {\LaTeX} support for the fonts PT Sans, PT
Serif and PT Mono developed by ParaType for the project ``Public
Types of Russian Federation'', and released under an open user
license. The fonts themselves are provided in both the TrueType
and Type~1 formats, both created by ParaType). The fonts
provide encodings OT1, T1, IL2, TS1, T2* and X2. The package
provides a convenient replacement of the two packages ptsans
and ptserif.
\end{package}
% \begin{package}{PTSans-NarrowBold-tlf-t1--base}{paratype}{{\LaTeX} support for free fonts by ParaType.}
% The package offers {\LaTeX} support for the fonts PT Sans, PT
% Serif and PT Mono developed by ParaType for the project ``Public
% Types of Russian Federation'', and released under an open user
% license. The fonts themselves are provided in both the TrueType
% and Type~1 formats, both created by ParaType). The fonts
% provide encodings OT1, T1, IL2, TS1, T2* and X2. The package
% provides a convenient replacement of the two packages ptsans
% and ptserif.
% \end{package}
% \begin{package}{PTSans-NarrowBoldSlanted-tlf-t1--base}{paratype}{{\LaTeX} support for free fonts by ParaType.}
% The package offers {\LaTeX} support for the fonts PT Sans, PT
% Serif and PT Mono developed by ParaType for the project ``Public
% Types of Russian Federation'', and released under an open user
% license. The fonts themselves are provided in both the TrueType
% and Type~1 formats, both created by ParaType). The fonts
% provide encodings OT1, T1, IL2, TS1, T2* and X2. The package
% provides a convenient replacement of the two packages ptsans
% and ptserif.
% \end{package}
% \begin{package}{PTSans-NarrowSlanted-tlf-t1--base}{paratype}{{\LaTeX} support for free fonts by ParaType.}
% The package offers {\LaTeX} support for the fonts PT Sans, PT
% Serif and PT Mono developed by ParaType for the project ``Public
% Types of Russian Federation'', and released under an open user
% license. The fonts themselves are provided in both the TrueType
% and Type~1 formats, both created by ParaType). The fonts
% provide encodings OT1, T1, IL2, TS1, T2* and X2. The package
% provides a convenient replacement of the two packages ptsans
% and ptserif.
% \end{package}
\begin{package}{PTSans-Regular-tlf-t1--base}{paratype}{{\LaTeX} support for free fonts by ParaType.}
The package offers {\LaTeX} support for the fonts PT Sans, PT
Serif and PT Mono developed by ParaType for the project ``Public
Types of Russian Federation'', and released under an open user
license. The fonts themselves are provided in both the TrueType
and Type~1 formats, both created by ParaType). The fonts
provide encodings OT1, T1, IL2, TS1, T2* and X2. The package
provides a convenient replacement of the two packages ptsans
and ptserif.
\end{package}
\begin{package}{PTSerif-Bold-tlf-t1--base}{paratype}{{\LaTeX} support for free fonts by ParaType.}
The package offers {\LaTeX} support for the fonts PT Sans, PT
Serif and PT Mono developed by ParaType for the project ``Public
Types of Russian Federation'', and released under an open user
license. The fonts themselves are provided in both the TrueType
and Type~1 formats, both created by ParaType). The fonts
provide encodings OT1, T1, IL2, TS1, T2* and X2. The package
provides a convenient replacement of the two packages ptsans
and ptserif.
\end{package}
% \begin{package}{PTSerif-BoldItalic-tlf-t1--base}{paratype}{{\LaTeX} support for free fonts by ParaType.}
% The package offers {\LaTeX} support for the fonts PT Sans, PT
% Serif and PT Mono developed by ParaType for the project ``Public
% Types of Russian Federation'', and released under an open user
% license. The fonts themselves are provided in both the TrueType
% and Type~1 formats, both created by ParaType). The fonts
% provide encodings OT1, T1, IL2, TS1, T2* and X2. The package
% provides a convenient replacement of the two packages ptsans
% and ptserif.
% \end{package}
% \begin{package}{PTSerif-BoldSlanted-tlf-t1--base}{paratype}{{\LaTeX} support for free fonts by ParaType.}
% The package offers {\LaTeX} support for the fonts PT Sans, PT
% Serif and PT Mono developed by ParaType for the project ``Public
% Types of Russian Federation'', and released under an open user
% license. The fonts themselves are provided in both the TrueType
% and Type~1 formats, both created by ParaType). The fonts
% provide encodings OT1, T1, IL2, TS1, T2* and X2. The package
% provides a convenient replacement of the two packages ptsans
% and ptserif.
% \end{package}
% \begin{package}{PTSerif-BoldUprightItalic-tlf-t1--base}{paratype}{{\LaTeX} support for free fonts by ParaType.}
% The package offers {\LaTeX} support for the fonts PT Sans, PT
% Serif and PT Mono developed by ParaType for the project ``Public
% Types of Russian Federation'', and released under an open user
% license. The fonts themselves are provided in both the TrueType
% and Type~1 formats, both created by ParaType). The fonts
% provide encodings OT1, T1, IL2, TS1, T2* and X2. The package
% provides a convenient replacement of the two packages ptsans
% and ptserif.
% \end{package}
\begin{package}{PTSerif-Caption-tlf-t1--base}{paratype}{{\LaTeX} support for free fonts by ParaType.}
The package offers {\LaTeX} support for the fonts PT Sans, PT
Serif and PT Mono developed by ParaType for the project ``Public
Types of Russian Federation'', and released under an open user
license. The fonts themselves are provided in both the TrueType
and Type~1 formats, both created by ParaType). The fonts
provide encodings OT1, T1, IL2, TS1, T2* and X2. The package
provides a convenient replacement of the two packages ptsans
and ptserif.
\end{package}
% \begin{package}{PTSerif-CaptionItalic-tlf-t1--base}{paratype}{{\LaTeX} support for free fonts by ParaType.}
% The package offers {\LaTeX} support for the fonts PT Sans, PT
% Serif and PT Mono developed by ParaType for the project ``Public
% Types of Russian Federation'', and released under an open user
% license. The fonts themselves are provided in both the TrueType
% and Type~1 formats, both created by ParaType). The fonts
% provide encodings OT1, T1, IL2, TS1, T2* and X2. The package
% provides a convenient replacement of the two packages ptsans
% and ptserif.
% \end{package}
% \begin{package}{PTSerif-CaptionSlanted-tlf-t1--base}{paratype}{{\LaTeX} support for free fonts by ParaType.}
% The package offers {\LaTeX} support for the fonts PT Sans, PT
% Serif and PT Mono developed by ParaType for the project ``Public
% Types of Russian Federation'', and released under an open user
% license. The fonts themselves are provided in both the TrueType
% and Type~1 formats, both created by ParaType). The fonts
% provide encodings OT1, T1, IL2, TS1, T2* and X2. The package
% provides a convenient replacement of the two packages ptsans
% and ptserif.
% \end{package}
% \begin{package}{PTSerif-CaptionUprightItalic-tlf-t1--base}{paratype}{{\LaTeX} support for free fonts by ParaType.}
% The package offers {\LaTeX} support for the fonts PT Sans, PT
% Serif and PT Mono developed by ParaType for the project ``Public
% Types of Russian Federation'', and released under an open user
% license. The fonts themselves are provided in both the TrueType
% and Type~1 formats, both created by ParaType). The fonts
% provide encodings OT1, T1, IL2, TS1, T2* and X2. The package
% provides a convenient replacement of the two packages ptsans
% and ptserif.
% \end{package}
\begin{package}{PTSerif-Italic-tlf-t1--base}{paratype}{{\LaTeX} support for free fonts by ParaType.}
The package offers {\LaTeX} support for the fonts PT Sans, PT
Serif and PT Mono developed by ParaType for the project ``Public
Types of Russian Federation'', and released under an open user
license. The fonts themselves are provided in both the TrueType
and Type~1 formats, both created by ParaType). The fonts
provide encodings OT1, T1, IL2, TS1, T2* and X2. The package
provides a convenient replacement of the two packages ptsans
and ptserif.
\end{package}
\begin{package}{PTSerif-Regular-tlf-t1--base}{paratype}{{\LaTeX} support for free fonts by ParaType.}
The package offers {\LaTeX} support for the fonts PT Sans, PT
Serif and PT Mono developed by ParaType for the project ``Public
Types of Russian Federation'', and released under an open user
license. The fonts themselves are provided in both the TrueType
and Type~1 formats, both created by ParaType). The fonts
provide encodings OT1, T1, IL2, TS1, T2* and X2. The package
provides a convenient replacement of the two packages ptsans
and ptserif.
\end{package}
\begin{package}{PTSerif-Slanted-tlf-t1--base}{paratype}{{\LaTeX} support for free fonts by ParaType.}
The package offers {\LaTeX} support for the fonts PT Sans, PT
Serif and PT Mono developed by ParaType for the project ``Public
Types of Russian Federation'', and released under an open user
license. The fonts themselves are provided in both the TrueType
and Type~1 formats, both created by ParaType). The fonts
provide encodings OT1, T1, IL2, TS1, T2* and X2. The package
provides a convenient replacement of the two packages ptsans
and ptserif.
\end{package}
\begin{package}{PTSerif-UprightItalic-tlf-t1--base}{paratype}{{\LaTeX} support for free fonts by ParaType.}
The package offers {\LaTeX} support for the fonts PT Sans, PT
Serif and PT Mono developed by ParaType for the project ``Public
Types of Russian Federation'', and released under an open user
license. The fonts themselves are provided in both the TrueType
and Type~1 formats, both created by ParaType). The fonts
provide encodings OT1, T1, IL2, TS1, T2* and X2. The package
provides a convenient replacement of the two packages ptsans
and ptserif.
\end{package}

\begin{package}{phaistos}{phaistos}{Disk of Phaistos font.}
A font that contains all the symbols of the famous Disc of
Phaistos, together with a {\LaTeX} package. The disc was 'printed'
by stamping the wet clay with some sort of punches, probably
around 1700 BCE. The font is available in Adobe Type~1 and
OpenType formats % (the latter using the Unicode positions for
% the symbols). There are those who believe that this Cretan
% script was used to 'write' Greek (it is known, for example,
% that the rather later Cretan Linear B script was used to write
% Greek), but arguments for other languages have been presented.
\end{package}

% \begin{package}{}{phonetic}{MetaFont Phonetic fonts, based on Computer Modern.}
% The fonts are based on Computer Modern, and specified in
% MetaFont. Macros for the fonts' use are provided, both for
% {\LaTeX} 2.09 and for current {\LaTeX}.
% \end{package}
% \begin{package}{pigpen}{pigpen}{A font for the pigpen (or masonic) cipher.}
% The Pigpen cipher package provides the font and the necessary
% wrappers (style file, etc.) in order to write Pigpen ciphers, a
% simple substitution cipher. The package provides a font
% (available both as MetaFont source, and as an Adobe Type~1
% file), and macros for its use.
% \end{package}

\begin{package}{ec-anttb}{poltawski}{Antykwa Poltawskiego Family of Fonts.}
The package contains the Antykwa Poltawskiego family of fonts
in the PostScript Type~1 and OpenType formats. The original
font was designed in the twenties of the XX century by the
Polish typographer Adam Poltawski (1881--1952). Following the
route set out by the Latin Modern and {\TeX} Gyre projects
(http://www.gust.org.pl/projects/e-foundry), the Antykwa
Poltawskiego digitisation project aims at providing a rich
collection of diacritical characters in the attempt to cover as
many Latin-based scripts as possible. To our knowledge, the
repertoire of characters covers all European languages as well
as some other Latin-based alphabets such as Vietnamese and
Navajo; at the request of users, recent extensions (following
the enhancement of the Latin Modern collection) provide glyphs
sufficient for typesetting of romanized transliterations of
Arabic and Sanskrit scripts. % The Antykwa Poltawskiego family
% consists of 4 weights (light, normal, medium, bold), each
% having upright and italic forms and one of 5 design sizes: 6,
% 8, 10, 12 and 17pt (in the OTF lingo: extended, semiextended,
% normal, semicondensed, and condensed, respectively).
% Altogether, the collection comprises 40 font files, containing
% the same repertoire of 1126 characters. The preliminary version
% of Antykwa Poltawskiego (antp package) released in 2000 is
% rendered obsolete by this package.
\end{package}
% \begin{package}{ec-anttbi}{poltawski}{Antykwa Poltawskiego Family of Fonts.}
% The package contains the Antykwa Poltawskiego family of fonts
% in the PostScript Type~1 and OpenType formats. The original
% font was designed in the twenties of the XX century by the
% Polish typographer Adam Poltawski (1881--1952). Following the
% route set out by the Latin Modern and {\TeX} Gyre projects
% (http://www.gust.org.pl/projects/e-foundry), the Antykwa
% Poltawskiego digitisation project aims at providing a rich
% collection of diacritical characters in the attempt to cover as
% many Latin-based scripts as possible. To our knowledge, the
% repertoire of characters covers all European languages as well
% as some other Latin-based alphabets such as Vietnamese and
% Navajo; at the request of users, recent extensions (following
% the enhancement of the Latin Modern collection) provide glyphs
% sufficient for typesetting of romanized transliterations of
% Arabic and Sanskrit scripts. % The Antykwa Poltawskiego family
% % consists of 4 weights (light, normal, medium, bold), each
% % having upright and italic forms and one of 5 design sizes: 6,
% % 8, 10, 12 and 17pt (in the OTF lingo: extended, semiextended,
% % normal, semicondensed, and condensed, respectively).
% % Altogether, the collection comprises 40 font files, containing
% % the same repertoire of 1126 characters. The preliminary version
% % of Antykwa Poltawskiego (antp package) released in 2000 is
% % rendered obsolete by this package.
% \end{package}
\begin{package}{ec-anttcb}{poltawski}{Antykwa Poltawskiego Family of Fonts.}
The package contains the Antykwa Poltawskiego family of fonts
in the PostScript Type~1 and OpenType formats. The original
font was designed in the twenties of the XX century by the
Polish typographer Adam Poltawski (1881--1952). Following the
route set out by the Latin Modern and {\TeX} Gyre projects
(http://www.gust.org.pl/projects/e-foundry), the Antykwa
Poltawskiego digitisation project aims at providing a rich
collection of diacritical characters in the attempt to cover as
many Latin-based scripts as possible. To our knowledge, the
repertoire of characters covers all European languages as well
as some other Latin-based alphabets such as Vietnamese and
Navajo; at the request of users, recent extensions (following
the enhancement of the Latin Modern collection) provide glyphs
sufficient for typesetting of romanized transliterations of
Arabic and Sanskrit scripts. % The Antykwa Poltawskiego family
% consists of 4 weights (light, normal, medium, bold), each
% having upright and italic forms and one of 5 design sizes: 6,
% 8, 10, 12 and 17pt (in the OTF lingo: extended, semiextended,
% normal, semicondensed, and condensed, respectively).
% Altogether, the collection comprises 40 font files, containing
% the same repertoire of 1126 characters. The preliminary version
% of Antykwa Poltawskiego (antp package) released in 2000 is
% rendered obsolete by this package.
\end{package}
% \begin{package}{ec-anttcbi}{poltawski}{Antykwa Poltawskiego Family of Fonts.}
% The package contains the Antykwa Poltawskiego family of fonts
% in the PostScript Type~1 and OpenType formats. The original
% font was designed in the twenties of the XX century by the
% Polish typographer Adam Poltawski (1881--1952). Following the
% route set out by the Latin Modern and {\TeX} Gyre projects
% (http://www.gust.org.pl/projects/e-foundry), the Antykwa
% Poltawskiego digitisation project aims at providing a rich
% collection of diacritical characters in the attempt to cover as
% many Latin-based scripts as possible. To our knowledge, the
% repertoire of characters covers all European languages as well
% as some other Latin-based alphabets such as Vietnamese and
% Navajo; at the request of users, recent extensions (following
% the enhancement of the Latin Modern collection) provide glyphs
% sufficient for typesetting of romanized transliterations of
% Arabic and Sanskrit scripts. % The Antykwa Poltawskiego family
% % consists of 4 weights (light, normal, medium, bold), each
% % having upright and italic forms and one of 5 design sizes: 6,
% % 8, 10, 12 and 17pt (in the OTF lingo: extended, semiextended,
% % normal, semicondensed, and condensed, respectively).
% % Altogether, the collection comprises 40 font files, containing
% % the same repertoire of 1126 characters. The preliminary version
% % of Antykwa Poltawskiego (antp package) released in 2000 is
% % rendered obsolete by this package.
% \end{package}
\begin{package}{ec-anttcl}{poltawski}{Antykwa Poltawskiego Family of Fonts.}
The package contains the Antykwa Poltawskiego family of fonts
in the PostScript Type~1 and OpenType formats. The original
font was designed in the twenties of the XX century by the
Polish typographer Adam Poltawski (1881--1952). Following the
route set out by the Latin Modern and {\TeX} Gyre projects
(http://www.gust.org.pl/projects/e-foundry), the Antykwa
Poltawskiego digitisation project aims at providing a rich
collection of diacritical characters in the attempt to cover as
many Latin-based scripts as possible. To our knowledge, the
repertoire of characters covers all European languages as well
as some other Latin-based alphabets such as Vietnamese and
Navajo; at the request of users, recent extensions (following
the enhancement of the Latin Modern collection) provide glyphs
sufficient for typesetting of romanized transliterations of
Arabic and Sanskrit scripts. % The Antykwa Poltawskiego family
% consists of 4 weights (light, normal, medium, bold), each
% having upright and italic forms and one of 5 design sizes: 6,
% 8, 10, 12 and 17pt (in the OTF lingo: extended, semiextended,
% normal, semicondensed, and condensed, respectively).
% Altogether, the collection comprises 40 font files, containing
% the same repertoire of 1126 characters. The preliminary version
% of Antykwa Poltawskiego (antp package) released in 2000 is
% rendered obsolete by this package.
\end{package}
% \begin{package}{ec-anttcli}{poltawski}{Antykwa Poltawskiego Family of Fonts.}
% The package contains the Antykwa Poltawskiego family of fonts
% in the PostScript Type~1 and OpenType formats. The original
% font was designed in the twenties of the XX century by the
% Polish typographer Adam Poltawski (1881--1952). Following the
% route set out by the Latin Modern and {\TeX} Gyre projects
% (http://www.gust.org.pl/projects/e-foundry), the Antykwa
% Poltawskiego digitisation project aims at providing a rich
% collection of diacritical characters in the attempt to cover as
% many Latin-based scripts as possible. To our knowledge, the
% repertoire of characters covers all European languages as well
% as some other Latin-based alphabets such as Vietnamese and
% Navajo; at the request of users, recent extensions (following
% the enhancement of the Latin Modern collection) provide glyphs
% sufficient for typesetting of romanized transliterations of
% Arabic and Sanskrit scripts. % The Antykwa Poltawskiego family
% % consists of 4 weights (light, normal, medium, bold), each
% % having upright and italic forms and one of 5 design sizes: 6,
% % 8, 10, 12 and 17pt (in the OTF lingo: extended, semiextended,
% % normal, semicondensed, and condensed, respectively).
% % Altogether, the collection comprises 40 font files, containing
% % the same repertoire of 1126 characters. The preliminary version
% % of Antykwa Poltawskiego (antp package) released in 2000 is
% % rendered obsolete by this package.
% \end{package}
\begin{package}{ec-anttcm}{poltawski}{Antykwa Poltawskiego Family of Fonts.}
The package contains the Antykwa Poltawskiego family of fonts
in the PostScript Type~1 and OpenType formats. The original
font was designed in the twenties of the XX century by the
Polish typographer Adam Poltawski (1881--1952). Following the
route set out by the Latin Modern and {\TeX} Gyre projects
(http://www.gust.org.pl/projects/e-foundry), the Antykwa
Poltawskiego digitisation project aims at providing a rich
collection of diacritical characters in the attempt to cover as
many Latin-based scripts as possible. To our knowledge, the
repertoire of characters covers all European languages as well
as some other Latin-based alphabets such as Vietnamese and
Navajo; at the request of users, recent extensions (following
the enhancement of the Latin Modern collection) provide glyphs
sufficient for typesetting of romanized transliterations of
Arabic and Sanskrit scripts. % The Antykwa Poltawskiego family
% consists of 4 weights (light, normal, medium, bold), each
% having upright and italic forms and one of 5 design sizes: 6,
% 8, 10, 12 and 17pt (in the OTF lingo: extended, semiextended,
% normal, semicondensed, and condensed, respectively).
% Altogether, the collection comprises 40 font files, containing
% the same repertoire of 1126 characters. The preliminary version
% of Antykwa Poltawskiego (antp package) released in 2000 is
% rendered obsolete by this package.
\end{package}
% \begin{package}{ec-anttcmi}{poltawski}{Antykwa Poltawskiego Family of Fonts.}
% The package contains the Antykwa Poltawskiego family of fonts
% in the PostScript Type~1 and OpenType formats. The original
% font was designed in the twenties of the XX century by the
% Polish typographer Adam Poltawski (1881--1952). Following the
% route set out by the Latin Modern and {\TeX} Gyre projects
% (http://www.gust.org.pl/projects/e-foundry), the Antykwa
% Poltawskiego digitisation project aims at providing a rich
% collection of diacritical characters in the attempt to cover as
% many Latin-based scripts as possible. To our knowledge, the
% repertoire of characters covers all European languages as well
% as some other Latin-based alphabets such as Vietnamese and
% Navajo; at the request of users, recent extensions (following
% the enhancement of the Latin Modern collection) provide glyphs
% sufficient for typesetting of romanized transliterations of
% Arabic and Sanskrit scripts. % The Antykwa Poltawskiego family
% % consists of 4 weights (light, normal, medium, bold), each
% % having upright and italic forms and one of 5 design sizes: 6,
% % 8, 10, 12 and 17pt (in the OTF lingo: extended, semiextended,
% % normal, semicondensed, and condensed, respectively).
% % Altogether, the collection comprises 40 font files, containing
% % the same repertoire of 1126 characters. The preliminary version
% % of Antykwa Poltawskiego (antp package) released in 2000 is
% % rendered obsolete by this package.
% \end{package}
\begin{package}{ec-anttcr}{poltawski}{Antykwa Poltawskiego Family of Fonts.}
The package contains the Antykwa Poltawskiego family of fonts
in the PostScript Type~1 and OpenType formats. The original
font was designed in the twenties of the XX century by the
Polish typographer Adam Poltawski (1881--1952). Following the
route set out by the Latin Modern and {\TeX} Gyre projects
(http://www.gust.org.pl/projects/e-foundry), the Antykwa
Poltawskiego digitisation project aims at providing a rich
collection of diacritical characters in the attempt to cover as
many Latin-based scripts as possible. To our knowledge, the
repertoire of characters covers all European languages as well
as some other Latin-based alphabets such as Vietnamese and
Navajo; at the request of users, recent extensions (following
the enhancement of the Latin Modern collection) provide glyphs
sufficient for typesetting of romanized transliterations of
Arabic and Sanskrit scripts. % The Antykwa Poltawskiego family
% consists of 4 weights (light, normal, medium, bold), each
% having upright and italic forms and one of 5 design sizes: 6,
% 8, 10, 12 and 17pt (in the OTF lingo: extended, semiextended,
% normal, semicondensed, and condensed, respectively).
% Altogether, the collection comprises 40 font files, containing
% the same repertoire of 1126 characters. The preliminary version
% of Antykwa Poltawskiego (antp package) released in 2000 is
% rendered obsolete by this package.
\end{package}
\begin{package}{ec-anttcri}{poltawski}{Antykwa Poltawskiego Family of Fonts.}
The package contains the Antykwa Poltawskiego family of fonts
in the PostScript Type~1 and OpenType formats. The original
font was designed in the twenties of the XX century by the
Polish typographer Adam Poltawski (1881--1952). Following the
route set out by the Latin Modern and {\TeX} Gyre projects
(http://www.gust.org.pl/projects/e-foundry), the Antykwa
Poltawskiego digitisation project aims at providing a rich
collection of diacritical characters in the attempt to cover as
many Latin-based scripts as possible. To our knowledge, the
repertoire of characters covers all European languages as well
as some other Latin-based alphabets such as Vietnamese and
Navajo; at the request of users, recent extensions (following
the enhancement of the Latin Modern collection) provide glyphs
sufficient for typesetting of romanized transliterations of
Arabic and Sanskrit scripts. % The Antykwa Poltawskiego family
% consists of 4 weights (light, normal, medium, bold), each
% having upright and italic forms and one of 5 design sizes: 6,
% 8, 10, 12 and 17pt (in the OTF lingo: extended, semiextended,
% normal, semicondensed, and condensed, respectively).
% Altogether, the collection comprises 40 font files, containing
% the same repertoire of 1126 characters. The preliminary version
% of Antykwa Poltawskiego (antp package) released in 2000 is
% rendered obsolete by this package.
\end{package}
\begin{package}{ec-anttl}{poltawski}{Antykwa Poltawskiego Family of Fonts.}
The package contains the Antykwa Poltawskiego family of fonts
in the PostScript Type~1 and OpenType formats. The original
font was designed in the twenties of the XX century by the
Polish typographer Adam Poltawski (1881--1952). Following the
route set out by the Latin Modern and {\TeX} Gyre projects
(http://www.gust.org.pl/projects/e-foundry), the Antykwa
Poltawskiego digitisation project aims at providing a rich
collection of diacritical characters in the attempt to cover as
many Latin-based scripts as possible. To our knowledge, the
repertoire of characters covers all European languages as well
as some other Latin-based alphabets such as Vietnamese and
Navajo; at the request of users, recent extensions (following
the enhancement of the Latin Modern collection) provide glyphs
sufficient for typesetting of romanized transliterations of
Arabic and Sanskrit scripts. % The Antykwa Poltawskiego family
% consists of 4 weights (light, normal, medium, bold), each
% having upright and italic forms and one of 5 design sizes: 6,
% 8, 10, 12 and 17pt (in the OTF lingo: extended, semiextended,
% normal, semicondensed, and condensed, respectively).
% Altogether, the collection comprises 40 font files, containing
% the same repertoire of 1126 characters. The preliminary version
% of Antykwa Poltawskiego (antp package) released in 2000 is
% rendered obsolete by this package.
\end{package}
% \begin{package}{ec-anttli}{poltawski}{Antykwa Poltawskiego Family of Fonts.}
% The package contains the Antykwa Poltawskiego family of fonts
% in the PostScript Type~1 and OpenType formats. The original
% font was designed in the twenties of the XX century by the
% Polish typographer Adam Poltawski (1881--1952). Following the
% route set out by the Latin Modern and {\TeX} Gyre projects
% (http://www.gust.org.pl/projects/e-foundry), the Antykwa
% Poltawskiego digitisation project aims at providing a rich
% collection of diacritical characters in the attempt to cover as
% many Latin-based scripts as possible. To our knowledge, the
% repertoire of characters covers all European languages as well
% as some other Latin-based alphabets such as Vietnamese and
% Navajo; at the request of users, recent extensions (following
% the enhancement of the Latin Modern collection) provide glyphs
% sufficient for typesetting of romanized transliterations of
% Arabic and Sanskrit scripts. % The Antykwa Poltawskiego family
% % consists of 4 weights (light, normal, medium, bold), each
% % having upright and italic forms and one of 5 design sizes: 6,
% % 8, 10, 12 and 17pt (in the OTF lingo: extended, semiextended,
% % normal, semicondensed, and condensed, respectively).
% % Altogether, the collection comprises 40 font files, containing
% % the same repertoire of 1126 characters. The preliminary version
% % of Antykwa Poltawskiego (antp package) released in 2000 is
% % rendered obsolete by this package.
% \end{package}
\begin{package}{ec-anttm}{poltawski}{Antykwa Poltawskiego Family of Fonts.}
The package contains the Antykwa Poltawskiego family of fonts
in the PostScript Type~1 and OpenType formats. The original
font was designed in the twenties of the XX century by the
Polish typographer Adam Poltawski (1881--1952). Following the
route set out by the Latin Modern and {\TeX} Gyre projects
(http://www.gust.org.pl/projects/e-foundry), the Antykwa
Poltawskiego digitisation project aims at providing a rich
collection of diacritical characters in the attempt to cover as
many Latin-based scripts as possible. To our knowledge, the
repertoire of characters covers all European languages as well
as some other Latin-based alphabets such as Vietnamese and
Navajo; at the request of users, recent extensions (following
the enhancement of the Latin Modern collection) provide glyphs
sufficient for typesetting of romanized transliterations of
Arabic and Sanskrit scripts. % The Antykwa Poltawskiego family
% consists of 4 weights (light, normal, medium, bold), each
% having upright and italic forms and one of 5 design sizes: 6,
% 8, 10, 12 and 17pt (in the OTF lingo: extended, semiextended,
% normal, semicondensed, and condensed, respectively).
% Altogether, the collection comprises 40 font files, containing
% the same repertoire of 1126 characters. The preliminary version
% of Antykwa Poltawskiego (antp package) released in 2000 is
% rendered obsolete by this package.
\end{package}
% \begin{package}{ec-anttmi}{poltawski}{Antykwa Poltawskiego Family of Fonts.}
% The package contains the Antykwa Poltawskiego family of fonts
% in the PostScript Type~1 and OpenType formats. The original
% font was designed in the twenties of the XX century by the
% Polish typographer Adam Poltawski (1881--1952). Following the
% route set out by the Latin Modern and {\TeX} Gyre projects
% (http://www.gust.org.pl/projects/e-foundry), the Antykwa
% Poltawskiego digitisation project aims at providing a rich
% collection of diacritical characters in the attempt to cover as
% many Latin-based scripts as possible. To our knowledge, the
% repertoire of characters covers all European languages as well
% as some other Latin-based alphabets such as Vietnamese and
% Navajo; at the request of users, recent extensions (following
% the enhancement of the Latin Modern collection) provide glyphs
% sufficient for typesetting of romanized transliterations of
% Arabic and Sanskrit scripts. % The Antykwa Poltawskiego family
% % consists of 4 weights (light, normal, medium, bold), each
% % having upright and italic forms and one of 5 design sizes: 6,
% % 8, 10, 12 and 17pt (in the OTF lingo: extended, semiextended,
% % normal, semicondensed, and condensed, respectively).
% % Altogether, the collection comprises 40 font files, containing
% % the same repertoire of 1126 characters. The preliminary version
% % of Antykwa Poltawskiego (antp package) released in 2000 is
% % rendered obsolete by this package.
% \end{package}
\begin{package}{ec-anttr}{poltawski}{Antykwa Poltawskiego Family of Fonts.}
The package contains the Antykwa Poltawskiego family of fonts
in the PostScript Type~1 and OpenType formats. The original
font was designed in the twenties of the XX century by the
Polish typographer Adam Poltawski (1881--1952). Following the
route set out by the Latin Modern and {\TeX} Gyre projects
(http://www.gust.org.pl/projects/e-foundry), the Antykwa
Poltawskiego digitisation project aims at providing a rich
collection of diacritical characters in the attempt to cover as
many Latin-based scripts as possible. To our knowledge, the
repertoire of characters covers all European languages as well
as some other Latin-based alphabets such as Vietnamese and
Navajo; at the request of users, recent extensions (following
the enhancement of the Latin Modern collection) provide glyphs
sufficient for typesetting of romanized transliterations of
Arabic and Sanskrit scripts. % The Antykwa Poltawskiego family
% consists of 4 weights (light, normal, medium, bold), each
% having upright and italic forms and one of 5 design sizes: 6,
% 8, 10, 12 and 17pt (in the OTF lingo: extended, semiextended,
% normal, semicondensed, and condensed, respectively).
% Altogether, the collection comprises 40 font files, containing
% the same repertoire of 1126 characters. The preliminary version
% of Antykwa Poltawskiego (antp package) released in 2000 is
% rendered obsolete by this package.
\end{package}
\begin{package}{ec-anttri}{poltawski}{Antykwa Poltawskiego Family of Fonts.}
The package contains the Antykwa Poltawskiego family of fonts
in the PostScript Type~1 and OpenType formats. The original
font was designed in the twenties of the XX century by the
Polish typographer Adam Poltawski (1881--1952). Following the
route set out by the Latin Modern and {\TeX} Gyre projects
(http://www.gust.org.pl/projects/e-foundry), the Antykwa
Poltawskiego digitisation project aims at providing a rich
collection of diacritical characters in the attempt to cover as
many Latin-based scripts as possible. To our knowledge, the
repertoire of characters covers all European languages as well
as some other Latin-based alphabets such as Vietnamese and
Navajo; at the request of users, recent extensions (following
the enhancement of the Latin Modern collection) provide glyphs
sufficient for typesetting of romanized transliterations of
Arabic and Sanskrit scripts. % The Antykwa Poltawskiego family
% consists of 4 weights (light, normal, medium, bold), each
% having upright and italic forms and one of 5 design sizes: 6,
% 8, 10, 12 and 17pt (in the OTF lingo: extended, semiextended,
% normal, semicondensed, and condensed, respectively).
% Altogether, the collection comprises 40 font files, containing
% the same repertoire of 1126 characters. The preliminary version
% of Antykwa Poltawskiego (antp package) released in 2000 is
% rendered obsolete by this package.
\end{package}

% \begin{package}{prodint}{prodint}{A font that provides the product integral symbol.}
% Product integrals are to products, as integrals are to sums.
% They have been around for more than a hundred years, they have
% not become part of the standard toolbox, possibly because no-%
% one invented the right mathematical symbol for them. The
% authors remedied that situation by proposing the symbol and
% providing this font.
% \end{package}
% \begin{package}{}{punk}{Donald Knuth's punk font.}
% A response to the assertion in a lecture that ``typography tends
% to lag behind other stylistic changes by about 10 years''. Knuth
% felt it was (in 1988) time to design a replacement for his
% designs of the 1970s, and came up with this font! The fonts are
% distributed as Metafont source. The package offers {\LaTeX}
% support by Rohit Grover, from an original by Sebastian Rahtz,
% which is slightly odd in claiming that the fonts are T1-
% encoded. A (possibly) more rational support package is to be
% found in punk-latex
% \end{package}
% \begin{package}{}{punk-latex}{{\LaTeX} support for punk fonts.}
% The package and .fd file provide support for Knuth's punk
% fonts. That bundle also offers support within {\LaTeX}; the
% present package is to be preferred.
% \end{package}
% \begin{package}{}{punknova}{OpenType version of Knuth's Punk font.}
% The font was generated from a MetaPost version of the sources
% of the 'original' punk font. Knuth's original fonts generated
% different shapes at random. This isn't actually possible in an
% OpenType font; rather, the font contains several variants of
% each glyph, and uses the OpenType randomize function to select
% a variant for each invocation.
% \end{package}
% \begin{package}{}{pxtxalfa}{Virtual maths alphabets based on pxfonts and txfonts.}
% The package provides virtual math alphabets based on pxfonts
% and txfonts, with {\LaTeX} support files and adjusted metrics. The
% mathalfa package offers support for this collection.
% \end{package}

\begin{package}{Quattrocento-Bold-tlf-t1--base}{quattrocento}{{\LaTeX} support for Quattrocento and Quattrocento Sans fonts.}
The package provides {\LaTeX}, {\pdfLaTeX}, {\XeLaTeX} and {\LuaLaTeX}
support for the Quattrocento and Quattrocento Sans families of
fonts, designed by Pablo Impallari; the fonts themselves are
also provided, in both Type~1 and OpenType format. Quattrocento
is a classic typeface with wide and open letterforms, and great
x-height, which makes it very legible for body text at small
sizes. Tiny details that only show up at bigger sizes make it
also great for display use. Quattrocento Sans is the perfect
sans-serif companion for Quattrocento.
\end{package}
% \begin{package}{Quattrocento-BoldItalic-tlf-t1--base}{quattrocento}{{\LaTeX} support for Quattrocento and Quattrocento Sans fonts.}
% The package provides {\LaTeX}, {\pdfLaTeX}, {\XeLaTeX} and {\LuaLaTeX}
% support for the Quattrocento and Quattrocento Sans families of
% fonts, designed by Pablo Impallari; the fonts themselves are
% also provided, in both Type~1 and OpenType format. Quattrocento
% is a classic typeface with wide and open letterforms, and great
% x-height, which makes it very legible for body text at small
% sizes. Tiny details that only show up at bigger sizes make it
% also great for display use. Quattrocento Sans is the perfect
% sans-serif companion for Quattrocento.
% \end{package}
\begin{package}{QuattrocentoItalic-tlf-t1--base}{quattrocento}{{\LaTeX} support for Quattrocento and Quattrocento Sans fonts.}
The package provides {\LaTeX}, {\pdfLaTeX}, {\XeLaTeX} and {\LuaLaTeX}
support for the Quattrocento and Quattrocento Sans families of
fonts, designed by Pablo Impallari; the fonts themselves are
also provided, in both Type~1 and OpenType format. Quattrocento
is a classic typeface with wide and open letterforms, and great
x-height, which makes it very legible for body text at small
sizes. Tiny details that only show up at bigger sizes make it
also great for display use. Quattrocento Sans is the perfect
sans-serif companion for Quattrocento.
\end{package}
\begin{package}{QuattrocentoRegular-tlf-t1--base}{quattrocento}{{\LaTeX} support for Quattrocento and Quattrocento Sans fonts.}
The package provides {\LaTeX}, {\pdfLaTeX}, {\XeLaTeX} and {\LuaLaTeX}
support for the Quattrocento and Quattrocento Sans families of
fonts, designed by Pablo Impallari; the fonts themselves are
also provided, in both Type~1 and OpenType format. Quattrocento
is a classic typeface with wide and open letterforms, and great
x-height, which makes it very legible for body text at small
sizes. Tiny details that only show up at bigger sizes make it
also great for display use. Quattrocento Sans is the perfect
sans-serif companion for Quattrocento.
\end{package}
\begin{package}{QuattrocentoSans-Bold-tlf-t1--base}{quattrocento}{{\LaTeX} support for Quattrocento and Quattrocento Sans fonts.}
The package provides {\LaTeX}, {\pdfLaTeX}, {\XeLaTeX} and {\LuaLaTeX}
support for the Quattrocento and Quattrocento Sans families of
fonts, designed by Pablo Impallari; the fonts themselves are
also provided, in both Type~1 and OpenType format. Quattrocento
is a classic typeface with wide and open letterforms, and great
x-height, which makes it very legible for body text at small
sizes. Tiny details that only show up at bigger sizes make it
also great for display use. Quattrocento Sans is the perfect
sans-serif companion for Quattrocento.
\end{package}
% \begin{package}{QuattrocentoSans-BoldItalic-tlf-t1--base}{quattrocento}{{\LaTeX} support for Quattrocento and Quattrocento Sans fonts.}
% The package provides {\LaTeX}, {\pdfLaTeX}, {\XeLaTeX} and {\LuaLaTeX}
% support for the Quattrocento and Quattrocento Sans families of
% fonts, designed by Pablo Impallari; the fonts themselves are
% also provided, in both Type~1 and OpenType format. Quattrocento
% is a classic typeface with wide and open letterforms, and great
% x-height, which makes it very legible for body text at small
% sizes. Tiny details that only show up at bigger sizes make it
% also great for display use. Quattrocento Sans is the perfect
% sans-serif companion for Quattrocento.
% \end{package}
\begin{package}{QuattrocentoSans-Italic-tlf-t1--base}{quattrocento}{{\LaTeX} support for Quattrocento and Quattrocento Sans fonts.}
The package provides {\LaTeX}, {\pdfLaTeX}, {\XeLaTeX} and {\LuaLaTeX}
support for the Quattrocento and Quattrocento Sans families of
fonts, designed by Pablo Impallari; the fonts themselves are
also provided, in both Type~1 and OpenType format. Quattrocento
is a classic typeface with wide and open letterforms, and great
x-height, which makes it very legible for body text at small
sizes. Tiny details that only show up at bigger sizes make it
also great for display use. Quattrocento Sans is the perfect
sans-serif companion for Quattrocento.
\end{package}
\begin{package}{QuattrocentoSans-tlf-t1--base}{quattrocento}{{\LaTeX} support for Quattrocento and Quattrocento Sans fonts.}
The package provides {\LaTeX}, {\pdfLaTeX}, {\XeLaTeX} and {\LuaLaTeX}
support for the Quattrocento and Quattrocento Sans families of
fonts, designed by Pablo Impallari; the fonts themselves are
also provided, in both Type~1 and OpenType format. Quattrocento
is a classic typeface with wide and open letterforms, and great
x-height, which makes it very legible for body text at small
sizes. Tiny details that only show up at bigger sizes make it
also great for display use. Quattrocento Sans is the perfect
sans-serif companion for Quattrocento.
\end{package}

\begin{package}{Raleway-Black-tlf-t1--base}{raleway}{Use Raleway with {\TeX}(-alike) systems.}
The package provides the Raleway family in an easy to use way.
For {\XeLaTeX} and {\LuaLaTeX} users the original OpenType fonts are
used. The entire font family is included.
\end{package}
% \begin{package}{Raleway-BlackItalic-tlf-t1--base}{raleway}{Use Raleway with {\TeX}(-alike) systems.}
% The package provides the Raleway family in an easy to use way.
% For {\XeLaTeX} and {\LuaLaTeX} users the original OpenType fonts are
% used. The entire font family is included.
% \end{package}
\begin{package}{Raleway-Bold-tlf-t1--base}{raleway}{Use Raleway with {\TeX}(-alike) systems.}
The package provides the Raleway family in an easy to use way.
For {\XeLaTeX} and {\LuaLaTeX} users the original OpenType fonts are
used. The entire font family is included.
\end{package}
% \begin{package}{Raleway-BoldItalic-tlf-t1--base}{raleway}{Use Raleway with {\TeX}(-alike) systems.}
% The package provides the Raleway family in an easy to use way.
% For {\XeLaTeX} and {\LuaLaTeX} users the original OpenType fonts are
% used. The entire font family is included.
% \end{package}
\begin{package}{Raleway-ExtraBold-tlf-t1--base}{raleway}{Use Raleway with {\TeX}(-alike) systems.}
The package provides the Raleway family in an easy to use way.
For {\XeLaTeX} and {\LuaLaTeX} users the original OpenType fonts are
used. The entire font family is included.
\end{package}
% \begin{package}{Raleway-ExtraBoldItalic-tlf-t1--base}{raleway}{Use Raleway with {\TeX}(-alike) systems.}
% The package provides the Raleway family in an easy to use way.
% For {\XeLaTeX} and {\LuaLaTeX} users the original OpenType fonts are
% used. The entire font family is included.
% \end{package}
\begin{package}{Raleway-ExtraLight-tlf-t1--base}{raleway}{Use Raleway with {\TeX}(-alike) systems.}
The package provides the Raleway family in an easy to use way.
For {\XeLaTeX} and {\LuaLaTeX} users the original OpenType fonts are
used. The entire font family is included.
\end{package}
% \begin{package}{Raleway-ExtraLightItalic-tlf-t1--base}{raleway}{Use Raleway with {\TeX}(-alike) systems.}
% The package provides the Raleway family in an easy to use way.
% For {\XeLaTeX} and {\LuaLaTeX} users the original OpenType fonts are
% used. The entire font family is included.
% \end{package}
\begin{package}{Raleway-Italic-tlf-t1--base}{raleway}{Use Raleway with {\TeX}(-alike) systems.}
The package provides the Raleway family in an easy to use way.
For {\XeLaTeX} and {\LuaLaTeX} users the original OpenType fonts are
used. The entire font family is included.
\end{package}
\begin{package}{Raleway-Light-tlf-t1--base}{raleway}{Use Raleway with {\TeX}(-alike) systems.}
The package provides the Raleway family in an easy to use way.
For {\XeLaTeX} and {\LuaLaTeX} users the original OpenType fonts are
used. The entire font family is included.
\end{package}
% \begin{package}{Raleway-LightItalic-tlf-t1--base}{raleway}{Use Raleway with {\TeX}(-alike) systems.}
% The package provides the Raleway family in an easy to use way.
% For {\XeLaTeX} and {\LuaLaTeX} users the original OpenType fonts are
% used. The entire font family is included.
% \end{package}
\begin{package}{Raleway-Medium-tlf-t1--base}{raleway}{Use Raleway with {\TeX}(-alike) systems.}
The package provides the Raleway family in an easy to use way.
For {\XeLaTeX} and {\LuaLaTeX} users the original OpenType fonts are
used. The entire font family is included.
\end{package}
% \begin{package}{Raleway-MediumItalic-tlf-t1--base}{raleway}{Use Raleway with {\TeX}(-alike) systems.}
% The package provides the Raleway family in an easy to use way.
% For {\XeLaTeX} and {\LuaLaTeX} users the original OpenType fonts are
% used. The entire font family is included.
% \end{package}
\begin{package}{Raleway-Regular-tlf-t1--base}{raleway}{Use Raleway with {\TeX}(-alike) systems.}
The package provides the Raleway family in an easy to use way.
For {\XeLaTeX} and {\LuaLaTeX} users the original OpenType fonts are
used. The entire font family is included.
\end{package}
\begin{package}{Raleway-SemiBold-tlf-t1--base}{raleway}{Use Raleway with {\TeX}(-alike) systems.}
The package provides the Raleway family in an easy to use way.
For {\XeLaTeX} and {\LuaLaTeX} users the original OpenType fonts are
used. The entire font family is included.
\end{package}
% \begin{package}{Raleway-SemiBoldItalic-tlf-t1--base}{raleway}{Use Raleway with {\TeX}(-alike) systems.}
% The package provides the Raleway family in an easy to use way.
% For {\XeLaTeX} and {\LuaLaTeX} users the original OpenType fonts are
% used. The entire font family is included.
% \end{package}
\begin{package}{Raleway-Thin-tlf-t1--base}{raleway}{Use Raleway with {\TeX}(-alike) systems.}
The package provides the Raleway family in an easy to use way.
For {\XeLaTeX} and {\LuaLaTeX} users the original OpenType fonts are
used. The entire font family is included.
\end{package}
\begin{package}{Raleway-ThinItalic-tlf-t1--base}{raleway}{Use Raleway with {\TeX}(-alike) systems.}
The package provides the Raleway family in an easy to use way.
For {\XeLaTeX} and {\LuaLaTeX} users the original OpenType fonts are
used. The entire font family is included.
\end{package}

% \begin{package}{}{recycle}{A font providing the ``recyclable'' logo.}
% This single-character font is provided as MetaFont source, and
% in Adobe Type~1 format. It is accompanied by a trivial {\LaTeX}
% package to use the logo at various sizes.
% \end{package}

% \begin{package}{t1-romandeadf-yrdr}{romande}{Romande ADF fonts and {\LaTeX} support.}
% Romande ADF is a serif font family with oldstyle figures,
% designed as a substitute for Times, Tiffany or Caslon. The
% family currently includes upright, italic and small-caps shapes
% in each of regular and demi-bold weights and an italic script
% in regular. The support package renames the fonts according to
% the Karl Berry fontname scheme and defines four families. Two
% of these primarily provide access to the ``standard'' or default
% characters while the ``alternate'' families support alternate
% characters, additional ligatures and the long s. The included
% package files provide access to these features in {\LaTeX} as
% explained in the documentation. The {\LaTeX} support requires the
% nfssext-cfr and the xkeyval packages.
% \end{package}

% \begin{package}{rrsfso10}{rsfso}{A mathematical calligraphic font based on rsfs.}
% The package provides virtual fonts and {\LaTeX} support files for
% mathematical calligraphic fonts based on the rsfs Adobe Type~1
% fonts (which must also be present for successful installation,
% with the slant substantially reduced. The output is quite
% similar to that from the Adobe Mathematical Pi script font.
% \end{package}

\begin{package}{mathkerncmssi10}{sansmathaccent}{Correct placement of accents in sans-serif maths.}
Sans serif maths (produced by the beamer class or the sfmath
package) often has accents positioned incorrectly. The package
fixes the positioning of such accents when the default font
(cmssi) is used for sans serif maths.
\end{package}

\begin{package}{cmsscsc10}{sansmathfonts}{Correct placement of accents in sans-serif maths.}
Sans serifsmall caps and math fonts for use with Computer
Modern.
\end{package}
% \begin{package}{}{sauter}{Wide range of design sizes for CM fonts.}
% Extensions, originally to the CM fonts, providing a
% parameterization scheme to build MetaFont fonts at true design
% sizes, for a large range of sizes. The scheme has now been
% extended to a range of other fonts, including the AMS fonts,
% bbm, bbold, rsfs and wasy fonts.
% \end{package}
% \begin{package}{}{sauterfonts}{Use sauter fonts in {\LaTeX}.}
% A package providing font definition files (plus a replacement
% for the package exscale) to access many of the fonts in
% Sauter's collection. These fonts are available in all point
% sizes and look nicer for such ``intermediate'' document sizes as
% 11pt. Also included is the package sbbm, an alternative to
% access the bbm fonts, a nice collection of blackboard bold
% symbols.
% \end{package}
% \begin{package}{}{schulschriften}{German ``school scripts'' from Suetterlin to the present day.}
% Das Paket enthalt im wesentlichen die METAFONT-Quellfiles fur
% die folgenden Schulausgangsschriften: Suetterlinschrift,
% Deutsche Normalschrift, Lateinische Ausgangsschrift,
% Schulausgangsschrift, Vereinfachte Ausgangsschrift. Damit ist
% es moglich, beliebige deutsche Texte in diesen Schreibschriften
% zu schreiben.
% \end{package}

% \begin{package}{smfer10}{semaphor}{Semaphore alphabet font.}
% These fonts represent semaphore in a highly schematic, but very
% clear, fashion. The fonts are provided as MetaFont source, and
% in both OpenType and Adobe Type~1 formats.
% \end{package}
\begin{package}{smfpr10}{semaphor}{Semaphore alphabet font.}
These fonts represent semaphore in a highly schematic, but very
clear, fashion. The fonts are provided as MetaFont source, and
in both OpenType and Adobe Type~1 formats.
\end{package}
\begin{package}{smfr10}{semaphor}{Semaphore alphabet font.}
These fonts represent semaphore in a highly schematic, but very
clear, fashion. The fonts are provided as MetaFont source, and
in both OpenType and Adobe Type~1 formats.
\end{package}

% \begin{package}{}{skull}{A font to draw a skull.}
% The font (defined in Metafont) defines a single character, a
% black solid skull. A package is supplied to make this character
% available as a symbol in maths mode.
% \end{package}

\begin{package}{SourceCodePro-Black-tlf-t1--base}{sourcecodepro}{Use SourceCodePro with {\TeX}(-alike) systems.}
The font is an open-source Monospaced development from Adobe.
The package provides fonts (in both Adobe Type~1 and OpenType
formats) and macros supporting their use in {\LaTeX} (Type~1) and
{\XeLaTeX}/{\LuaLaTeX} (OTF).
\end{package}
\begin{package}{SourceCodePro-Bold-tlf-t1--base}{sourcecodepro}{Use SourceCodePro with {\TeX}(-alike) systems.}
The font is an open-source Monospaced development from Adobe.
The package provides fonts (in both Adobe Type~1 and OpenType
formats) and macros supporting their use in {\LaTeX} (Type~1) and
{\XeLaTeX}/{\LuaLaTeX} (OTF).
\end{package}
\begin{package}{SourceCodePro-ExtraLight-tlf-t1--base}{sourcecodepro}{Use SourceCodePro with {\TeX}(-alike) systems.}
The font is an open-source Monospaced development from Adobe.
The package provides fonts (in both Adobe Type~1 and OpenType
formats) and macros supporting their use in {\LaTeX} (Type~1) and
{\XeLaTeX}/{\LuaLaTeX} (OTF).
\end{package}
\begin{package}{SourceCodePro-Light-tlf-t1--base}{sourcecodepro}{Use SourceCodePro with {\TeX}(-alike) systems.}
The font is an open-source Monospaced development from Adobe.
The package provides fonts (in both Adobe Type~1 and OpenType
formats) and macros supporting their use in {\LaTeX} (Type~1) and
{\XeLaTeX}/{\LuaLaTeX} (OTF).
\end{package}
\begin{package}{SourceCodePro-Medium-tlf-t1--base}{sourcecodepro}{Use SourceCodePro with {\TeX}(-alike) systems.}
The font is an open-source Monospaced development from Adobe.
The package provides fonts (in both Adobe Type~1 and OpenType
formats) and macros supporting their use in {\LaTeX} (Type~1) and
{\XeLaTeX}/{\LuaLaTeX} (OTF).
\end{package}
\begin{package}{SourceCodePro-Regular-tlf-t1--base}{sourcecodepro}{Use SourceCodePro with {\TeX}(-alike) systems.}
The font is an open-source Monospaced development from Adobe.
The package provides fonts (in both Adobe Type~1 and OpenType
formats) and macros supporting their use in {\LaTeX} (Type~1) and
{\XeLaTeX}/{\LuaLaTeX} (OTF).
\end{package}
\begin{package}{SourceCodePro-Semibold-tlf-t1--base}{sourcecodepro}{Use SourceCodePro with {\TeX}(-alike) systems.}
The font is an open-source Monospaced development from Adobe.
The package provides fonts (in both Adobe Type~1 and OpenType
formats) and macros supporting their use in {\LaTeX} (Type~1) and
{\XeLaTeX}/{\LuaLaTeX} (OTF).
\end{package}

\begin{package}{SourceSansPro-Black-tlf-t1--base}{sourcesanspro}{Use SourceSansPro with {\TeX}(-alike) systems.}
The font is an open-source Sans-Serif development from Adobe.
The package provides fonts (in both Adobe Type~1 and OpenType
formats) and macros supporting their use in {\LaTeX} (Type~1) and
{\XeLaTeX}/{\LuaLaTeX} (OTF).
\end{package}
% \begin{package}{SourceSansPro-BlackIt-tlf-t1--base}{sourcesanspro}{Use SourceSansPro with {\TeX}(-alike) systems.}
% The font is an open-source Sans-Serif development from Adobe.
% The package provides fonts (in both Adobe Type~1 and OpenType
% formats) and macros supporting their use in {\LaTeX} (Type~1) and
% {\XeLaTeX}/{\LuaLaTeX} (OTF).
% \end{package}
\begin{package}{SourceSansPro-Bold-tlf-t1--base}{sourcesanspro}{Use SourceSansPro with {\TeX}(-alike) systems.}
The font is an open-source Sans-Serif development from Adobe.
The package provides fonts (in both Adobe Type~1 and OpenType
formats) and macros supporting their use in {\LaTeX} (Type~1) and
{\XeLaTeX}/{\LuaLaTeX} (OTF).
\end{package}
% \begin{package}{SourceSansPro-BoldIt-tlf-t1--base}{sourcesanspro}{Use SourceSansPro with {\TeX}(-alike) systems.}
% The font is an open-source Sans-Serif development from Adobe.
% The package provides fonts (in both Adobe Type~1 and OpenType
% formats) and macros supporting their use in {\LaTeX} (Type~1) and
% {\XeLaTeX}/{\LuaLaTeX} (OTF).
% \end{package}
\begin{package}{SourceSansPro-ExtraLight-tlf-t1--base}{sourcesanspro}{Use SourceSansPro with {\TeX}(-alike) systems.}
The font is an open-source Sans-Serif development from Adobe.
The package provides fonts (in both Adobe Type~1 and OpenType
formats) and macros supporting their use in {\LaTeX} (Type~1) and
{\XeLaTeX}/{\LuaLaTeX} (OTF).
\end{package}
% \begin{package}{SourceSansPro-ExtraLightIt-tlf-t1--base}{sourcesanspro}{Use SourceSansPro with {\TeX}(-alike) systems.}
% The font is an open-source Sans-Serif development from Adobe.
% The package provides fonts (in both Adobe Type~1 and OpenType
% formats) and macros supporting their use in {\LaTeX} (Type~1) and
% {\XeLaTeX}/{\LuaLaTeX} (OTF).
% \end{package}
\begin{package}{SourceSansPro-It-tlf-t1--base}{sourcesanspro}{Use SourceSansPro with {\TeX}(-alike) systems.}
The font is an open-source Sans-Serif development from Adobe.
The package provides fonts (in both Adobe Type~1 and OpenType
formats) and macros supporting their use in {\LaTeX} (Type~1) and
{\XeLaTeX}/{\LuaLaTeX} (OTF).
\end{package}
\begin{package}{SourceSansPro-Light-tlf-t1--base}{sourcesanspro}{Use SourceSansPro with {\TeX}(-alike) systems.}
The font is an open-source Sans-Serif development from Adobe.
The package provides fonts (in both Adobe Type~1 and OpenType
formats) and macros supporting their use in {\LaTeX} (Type~1) and
{\XeLaTeX}/{\LuaLaTeX} (OTF).
\end{package}
% \begin{package}{SourceSansPro-LightIt-tlf-t1--base}{sourcesanspro}{Use SourceSansPro with {\TeX}(-alike) systems.}
% The font is an open-source Sans-Serif development from Adobe.
% The package provides fonts (in both Adobe Type~1 and OpenType
% formats) and macros supporting their use in {\LaTeX} (Type~1) and
% {\XeLaTeX}/{\LuaLaTeX} (OTF).
% \end{package}
\begin{package}{SourceSansPro-Regular-tlf-t1--base}{sourcesanspro}{Use SourceSansPro with {\TeX}(-alike) systems.}
The font is an open-source Sans-Serif development from Adobe.
The package provides fonts (in both Adobe Type~1 and OpenType
formats) and macros supporting their use in {\LaTeX} (Type~1) and
{\XeLaTeX}/{\LuaLaTeX} (OTF).
\end{package}
\begin{package}{SourceSansPro-Semibold-tlf-t1--base}{sourcesanspro}{Use SourceSansPro with {\TeX}(-alike) systems.}
The font is an open-source Sans-Serif development from Adobe.
The package provides fonts (in both Adobe Type~1 and OpenType
formats) and macros supporting their use in {\LaTeX} (Type~1) and
{\XeLaTeX}/{\LuaLaTeX} (OTF).
\end{package}
% \begin{package}{SourceSansPro-SemiboldIt-tlf-t1--base}{sourcesanspro}{Use SourceSansPro with {\TeX}(-alike) systems.}
% The font is an open-source Sans-Serif development from Adobe.
% The package provides fonts (in both Adobe Type~1 and OpenType
% formats) and macros supporting their use in {\LaTeX} (Type~1) and
% {\XeLaTeX}/{\LuaLaTeX} (OTF).
% \end{package}

\begin{package}{fsts8x}{starfont}{The StarFont Sans astrological font.}
The package contains StarFontSans and StarFontSerif, two
public-domain astrological fonts designed by Anthony I.P. Owen,
and the appropriate macros to use them with {\TeX} and {\LaTeX}. The
fonts are supplied in the original TrueType Format and as
PostScript font files.
\end{package}

% \begin{package}{icelandic}{staves}{Typeset Icelandic staves and runic letters.}
% This package contains all the necessary tools to typeset the
% ``magical'' Icelandic staves plus the runic letters used in
% Iceland. Included are a font in Adobe Type~1 format and {\LaTeX}
% support.
% \end{package}

% \begin{package}{}{stix}{OpenType Unicode maths fonts.}
% The STIX fonts are a suite of unicode OpenType fonts containing
% a complete set of mathematical glyphs. The CTAN copy is a
% mirror of their official release, organised as specified by the
% TeX Directory Structure, for inclusion in standard {\TeX}
% distributions.
% \end{package}
% \begin{package}{libertinesups}{superiors}{Attach superior figures to a font family.}
% The package allows the attachment of an arbitrary superior
% figures font to a font family that lacks one. (Superior figures
% are commonly used as footnote markers.) Two superior figures
% fonts are provided--one matching Times, the other matching
% Libertine.
% \end{package}
% \begin{package}{}{tapir}{A simple geometrical font.}
% Tapir is a simple geometrical font mostly created of line and
% circular segments with constant thickness. The font is
% available as MetaFont source and in Adobe Type~1 format. The
% character set contains all characters in the range 0-127 (as in
% cmr10), accented characters used in the Czech, Slovak and
% Polish languages.
% \end{package}
% \begin{package}{}{tengwarscript}{{\LaTeX} support for using Tengwar fonts.}
% The package provides ``mid-level'' access to tengwar fonts,
% providing good quality output. Each tengwar sign is represented
% by a command, which will place the sign nicely in relation to
% previous signs. A transcription package is available from the
% package's home page: writing all those tengwar commands would
% quickly become untenable. The package supports the use of a
% wide variety of tengwar fonts that are available from the net;
% metric and map files are provided for all the supported fonts.
% \end{package}
\begin{package}{tfrupee}{tfrupee}{A font offering the new (Indian) Rupee symbol.}
The package provides {\LaTeX} support for the (Indian) Rupee
symbol font, created by TechFat. The original font has been
converted to Adobe Type~1 format, and simple {\LaTeX} support
written for its use.
\end{package}
% \begin{package}{}{tpslifonts}{A {\LaTeX} package for configuring presentation fonts.}
% This package aims to improve presentations in terms of font
% readability, especially with maths. The standard cm math fonts
% at large design sizes are difficult to read from far away,
% especially at low resolutions and low contrast color choice.
% Using this package leads to much better overall readability of
% some font combinations. The package offers a couple of
% 'harmonising' combinations of text and math fonts from the
% (distant) relatives of computer modern fonts, with a couple of
% extras for optimising readability. Text fonts from computer
% modern roman, computer modern sans serif, SliTeX computer
% modern sans serif, computer modern bright, or concrete roman
% are available, in addition to math fonts from computer modern
% math, computer modern bright math, or Euler fonts. The package
% is part of the {\TeX}Power bundle.
% \end{package}

\begin{package}{trjnr10}{TRAJAN}{\uppercase{Fonts from the Trajan column in Rome.}}
\uppercase{Provides fonts (in both MetaFont and Adobe Type~1 format) based
on the capitals carved on the Trajan column in Rome in 114 AD,
together with macros to access the fonts. Many typographers
think these rank first among the Roman's artistic legacy. The
font is uppercase letters together with some punctuation and
analphabetics; no lowercase or digits.}
\end{package}

\begin{package}{gtimesg6r}{txfontsb}{Extensions to txfonts, using GNU Freefont.}
A set of fonts that extend the txfonts bundle with small caps
and old style numbers, together with Greek support. The
extensions are made with modifications of the GNU Freefont.
\end{package}
% \begin{package}{}{umtypewriter}{Fonts to typeset with the xgreek package.}
% The UMTypewriter font family is a monospaced font family that
% was built from glyphs from the CB Greek fonts, the CyrTUG
% Cyrillic alphabet fonts (``LH''), and the standard Computer
% Modern font family. It contains four OpenType fonts which are
% required for use of the xgreek package for {\XeLaTeX}.
% \end{package}
% \begin{package}{}{universa}{Herbert Bayer's 'universal' font.}
% An implementation of the universal font by Herbert Bayer of the
% Bauhaus school. The MetaFont sources of the fonts, and their
% {\LaTeX} support, are all supplied in a {\LaTeX} documented source
% (.dtx) file.
% \end{package}
% \begin{package}{}{urwchancal}{Use URW's clone of Zapf Chancery as a maths alphabet.}
% The package allows (the URW clone of) Zapf Chancery to function
% as a maths alphabet, the target of mathcal or mathscr, with
% accents appearing where they should, and other spacing
% parameters set to reasonable (not very tight) values.
% \end{package}

% \begin{package}{t1-yv1r}{venturisadf}{Venturis ADF fonts collection.}
% Serif and sans serif complete text font families, in both Adobe
% Type~1 and OpenType formats for publication. The family is
% based on Utopia family, and has been modified and developed by
% the Arkandis Digital foundry. Support for using the fonts, in
% {\LaTeX}, is also provided (and makes use of the nfssext-cfr
% package).
% \end{package}

% \begin{package}{dvng10}{velthuis: Velthuis}{Typeset Devanagari.}
% Frans Velthuis' preprocessor for Devanagari text, and fonts and
% macros to use when typesetting the processed text. The macros
% provide features that support Sanskrit, Hindi, Marathi, Nepali,
% and other languages typically printed in the Devanagari script.
% The fonts are available both in Metafont and Type 1 format.
% \end{package}
\begin{package}{dvnn10}{velthuis: VelthuisNepali}{Typeset Devanagari.}
Frans Velthuis' preprocessor for Devanagari text, and fonts and
macros to use when typesetting the processed text. The macros
provide features that support Sanskrit, Hindi, Marathi, Nepali,
and other languages typically printed in the Devanagari script.
The fonts are available both in Metafont and Type 1 format.
\end{package}

% \begin{package}{}{wsuipa}{International Phonetic Alphabet fonts.}
% The package provides a 7-bit IPA font, as Metafont source, and
% macros for support under {\TeX}t1 and {\LaTeX}. The fonts (and
% macros) are now largely superseded by the tipa fonts.
% \end{package}

\begin{package}{XCharter-Bold-tlf-t1--base}{xcharter}{Extension of Bitstream Charter fonts.}
The package presents an extension of Bitstream Charter, which
provides small caps, oldstyle figures and superior figures in
all four styles, accompanied by {\LaTeX} font support files. The
fonts themselves are provided in both Adobe Type~1 and OTF
formats, with supporting files as necessary.
\end{package}
\begin{package}{XCharter-Italic-tlf-t1--base}{xcharter}{Extension of Bitstream Charter fonts.}
The package presents an extension of Bitstream Charter, which
provides small caps, oldstyle figures and superior figures in
all four styles, accompanied by {\LaTeX} font support files. The
fonts themselves are provided in both Adobe Type~1 and OTF
formats, with supporting files as necessary.
\end{package}
\begin{package}{XCharter-Roman-tlf-t1--base}{xcharter}{Extension of Bitstream Charter fonts.}
The package presents an extension of Bitstream Charter, which
provides small caps, oldstyle figures and superior figures in
all four styles, accompanied by {\LaTeX} font support files. The
fonts themselves are provided in both Adobe Type~1 and OTF
formats, with supporting files as necessary.
\end{package}

% \begin{package}{}{xits}{A Scientific Times-like font with support for mathematical typesetting.}
% XITS is a Times-like font for scientific typesetting with
% proper mathematical support for modern, Unicode and OpenType
% capable {\TeX} engines, namely LuaTeX and {\XeTeX}. For use with
% {\LuaLaTeX} or {\XeLaTeX}, support is available from the fontspec and
% unicode-math packages.
% \end{package}
% \begin{package}{}{yfonts}{Support for old German fonts.}
% A {\LaTeX} interface to the old-german fonts designed by Yannis
% Haralambous: Gothic, Schwabacher, Fraktur and the baroque
% initials.
% \end{package}

\end{document}

# collection-basic
amsfonts
cm

# collection-fontsrecommended
avantgar
bookman
charter
cm-super
cmextra
courier
euro
euro-ce
eurosym
fpl
helvetic
lm
lm-math
marvosym
mathpazo
ncntrsbk
palatino
pxfonts
rsfs
symbol
tex-gyre
tex-gyre-math
times
tipa
txfonts
utopia
wasy
wasysym
zapfchan
zapfding

# collection-fontsextra
Asana-Math
accanthis
adforn
adfsymbols
aecc
alegreya
allrunes
anonymouspro
antiqua
antt
archaic
arev
ascii-font
aspectratio
astro
augie
auncial-new
aurical
b1encoding
barcodes
baskervald
bbding
bbm
bbm-macros
bbold
bbold-type1
belleek
bera
berenisadf
bguq
blacklettert1
boisik
bookhands
boondox
braille
brushscr
cabin
calligra
calligra-type1
cantarell
carolmin-ps
ccicons
cfr-lm
cherokee
cm-lgc
cm-unicode
cmbright
cmll
cmpica
cmtiup
comfortaa
concmath-fonts
cookingsymbols
countriesofeurope
courier-scaled
cryst
cyklop
dancers
dejavu
dice
dictsym
dingbat
doublestroke
dozenal
droid
duerer
duerer-latex
dutchcal
ean
ebgaramond
ecc
eco
eiad
eiad-ltx
electrum
elvish
epigrafica
epsdice
esstix
esvect
eulervm
euxm
fbb
fdsymbol
feyn
fge
foekfont
fonetika
fontawesome
fourier
fouriernc
frcursive
genealogy
gentium-tug
gfsartemisia
gfsbodoni
gfscomplutum
gfsdidot
gfsneohellenic
gfssolomos
gillcm
gillius
gnu-freefont
gothic
greenpoint
grotesq
hacm
hands
hfbright
hfoldsty
ifsym
inconsolata
initials
ipaex-type1
iwona
jablantile
jamtimes
junicode
kixfont
knuthotherfonts
kpfonts
kurier
lato
lfb
libertine
librebaskerville
librecaslon
libris
linearA
lxfonts
ly1
mathabx
mathabx-type1
mathdesign
mdputu
mdsymbol
merriweather
mintspirit
mnsymbol
newpx
newtx
nkarta
ocherokee
ocr-b
ocr-b-outline
ogham
oinuit
oldlatin
oldstandard
opensans
orkhun
pacioli
paratype
phaistos
phonetic
pigpen
poltawski
prodint
punk
punk-latex
punknova
pxtxalfa
quattrocento
raleway
recycle
romande
rsfso
sansmathaccent
sansmathfonts
sauter
sauterfonts
schulschriften
semaphor
skull
sourcecodepro
sourcesanspro
starfont
staves
stix
superiors
tapir
tengwarscript
tfrupee
tpslifonts
trajan
txfontsb
umtypewriter
universa
urwchancal
venturisadf
wsuipa
xcharter
xits
yfonts

